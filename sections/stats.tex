\chapter{Statistics}
\label{chap:stats}

%%%%%%%%%%%%%%%%%%%%%%%%%%%%%%%%%%%%%%%%%%%%%%%%%%%%%%%%
\section{Bayes' Theorem}
\label{stats:Bayes}

Bayes' theorem follows from the probability of the intersection of two events $A$ and $B$:

\begin{equation}\label{eq:stats:intersection}
P\left(A \cap B\right) = P\left(A \mid B\right) P\left(B\right) = P\left(B \mid A\right) P\left(A\right).
\end{equation}

\noindent Dividing by $P\left(B\right)$ we have:

\begin{equation}\label{eq:stats:Bayes}
\begin{split}
P\left(A \mid B\right) &= \frac{P\left(B \mid A\right) P\left(A\right)}{P\left(B\right)}\,, \\
&= \frac{P\left(B \mid A_{i}\right) P\left(A_{i}\right)}{\sum_{j} P\left(B \mid A_{j}\right)P\left(A_{j}\right)}\,, \\
\text{Posterior} &= \frac{\text{Likelihood} \times \text{Prior}}{\text{Normalization}}\,.
\end{split}
\end{equation}

Example: Testing for disease with a \SI{2}{\percent} incidence rate in the wider population.
The test has a \SI{99}{\percent} true positive rate and a \SI{15}{\percent} false positive rate.
What is the probability an individual has the disease if their test is positive?

\begin{equation}\label{eq:stats:BayesEx}
\begin{split}
P\left(\text{Infected} \mid +\right) &= \frac{P\left(+ \mid \text{Infected}\right) P\left(\text{Infected}\right)}{P\left(+\right)}\,, \\
 &= \frac{P\left(+ \mid \text{Infected}\right) P\left(\text{Infected}\right)}{
P\left(+ \mid \text{Infected}\right)P\left(\text{Infected}\right) + P\left(+ \mid \text{Healthy}\right)P\left(\text{Healthy}\right)}\,, \\
&= \frac{\num{0.99} \times \num{0.02}}{\num{0.99} \times \num{0.02} + \num{0.15} \times \left(1-\num{0.02}\right)}\,, \\
&\approx \num{0.57}\,.
\end{split}
\end{equation}

\noindent And if we run a second, independent, test which also comes back positive?

\begin{equation}\label{eq:stats:BayesEx2}
\begin{split}
P\left(\text{Infected} \mid ++\right) &= \frac{P\left(++ \mid \text{Infected}\right) P\left(\text{Infected}\right)}{P\left(++\right)}\,, \\
&= \frac{\num{0.99}^{2} \times \num{0.02}}{\num{0.99}^{2} \times \num{0.02} + \num{0.15}^{2} \times \left(1-\num{0.02}\right)}\,, \\
&\approx \num{0.90}\,.
\end{split}
\end{equation}

%%%%%%%%%%%%%%%%%%%%%%%%%%%%%%%%%%%%%%%%%%%%%%%%%%%%%%%%
\section{Gaussian Distribution}
\label{stats:gaus}
% TODO

%%%%%%%%%%%%%%%%%%%%%%%%%%%%%%%%%%%%%%%%%%%%%%%%%%%%%%%%
\section{Binomial Distribution}
\label{stats:binomial}
% TODO

%%%%%%%%%%%%%%%%%%%%%%%%%%%%%%%%%%%%%%%%%%%%%%%%%%%%%%%%
\section{Poisson Distribution}
\label{stats:poisson}
% TODO

\subsubsection{Bernoulli Distribution}
\label{stats:poisson:bernoulli}
% TODO

%%%%%%%%%%%%%%%%%%%%%%%%%%%%%%%%%%%%%%%%%%%%%%%%%%%%%%%%
\section{Student's \texorpdfstring{$t$}{t}-Distribution}
\label{stats:t_dist}
% TODO

%%%%%%%%%%%%%%%%%%%%%%%%%%%%%%%%%%%%%%%%%%%%%%%%%%%%%%%%
\section{Student's \texorpdfstring{$t$}{t}-Test}
\label{stats:t_test}
% TODO

%%%%%%%%%%%%%%%%%%%%%%%%%%%%%%%%%%%%%%%%%%%%%%%%%%%%%%%%
\section{Maximum Likelihood Estimation (MLE)}
\label{stats:MLE}
% TODO

% TODO under ``regularity assumptions'', MLE converges to the true optimal values as $n \to \inft$, is usually biased but bias is reduced as $n \to \inft$, efficient - variance approximates Cramer-Rao lower bound for larger samples

%%%%%%%%%%%%%%%%%%%%%%%%%%%%%%%%%%%%%%%%%%%%%%%%%%%%%%%%
\section{Principle Component Analysis (PCA)}
\label{stats:PCA}
% TODO

%%%%%%%%%%%%%%%%%%%%%%%%%%%%%%%%%%%%%%%%%%%%%%%%%%%%%%%%
\section{Time Series Analysis}
\label{stats:time_series_ana}
% TODO

