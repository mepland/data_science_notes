%%%%%%%%%%%%%%%%%%%%%%%%%%%%%%%%%%%%%%%%%%%%%%%%%%%%%%%%
%%%%%%%%%%%%%%%%%%%%%%%%%%%%%%%%%%%%%%%%%%%%%%%%%%%%%%%%
\chapter{Statistics}
\label{chap:stats}

%%%%%%%%%%%%%%%%%%%%%%%%%%%%%%%%%%%%%%%%%%%%%%%%%%%%%%%%
%%%%%%%%%%%%%%%%%%%%%%%%%%%%%%%%%%%%%%%%%%%%%%%%%%%%%%%%
\section{Bayes' Theorem}
\label{stats:Bayes}

Bayes' theorem follows from the probability of the intersection of two events $A$ and $B$:

\begin{equation}\label{eq:stats:intersection}
P\left(A \cap B\right) = P\left(A \mid B\right) P\left(B\right) = P\left(B \mid A\right) P\left(A\right).
\end{equation}

\noindent Dividing by $P\left(B\right)$ we have:

\begin{equation}\label{eq:stats:Bayes}
\begin{split}
P\left(A \mid B\right) &= \frac{P\left(B \mid A\right) P\left(A\right)}{P\left(B\right)}\,, \\
&= \frac{P\left(B \mid A_{i}\right) P\left(A_{i}\right)}{\sum_{j} P\left(B \mid A_{j}\right)P\left(A_{j}\right)}\,, \\
\text{Posterior} &= \frac{\text{Likelihood} \times \text{Prior}}{\text{Normalization}}\,.
\end{split}
\end{equation}

%%%%%%%%%%%%%%%%%%%%%%%%%%%%%%%%%%%%%%%%%%%%%%%%%%%%%%%%
\subsection{Example: Medical Testing}
\label{stats:Bayes:medical_test}

Example: Testing for disease with a \SI{2}{\percent} incidence rate in the wider population.
The test has a \SI{99}{\percent} true positive rate and a \SI{5}{\percent} false positive rate.
What is the probability an individual has the disease if their test is positive?

% https://www.wolframalpha.com/input/?i=(0.99*0.02)%2F((0.99*0.02)%2B(0.05*(1%E2%88%920.02)))
\begin{equation}\label{eq:stats:Bayes:medical_test_1}
\begin{split}
P\left(\text{Infected} \mid +\right) &= \frac{P\left(+ \mid \text{Infected}\right) P\left(\text{Infected}\right)}{P\left(+\right)}\,, \\
 &= \frac{P\left(+ \mid \text{Infected}\right) P\left(\text{Infected}\right)}{
P\left(+ \mid \text{Infected}\right)P\left(\text{Infected}\right) + P\left(+ \mid \text{Healthy}\right)P\left(\text{Healthy}\right)}\,, \\
&= \frac{\num{0.99} \times \num{0.02}}{\num{0.99} \times \num{0.02} + \num{0.05} \times \left(1-\num{0.02}\right)}\,, \\
&\approx \num{0.288}\,.
\end{split}
\end{equation}

\noindent And if we then run a second, independent, test which also comes back positive?

% https://www.wolframalpha.com/input/?i=(0.99*0.288)%2F((0.99*0.288)%2B(0.05*(1%E2%88%920.288)))
\begin{equation}\label{eq:stats:Bayes:medical_test_2}
\begin{split}
P\left(\text{Infected} \mid ++\right) &= \frac{P\left(+ \mid \text{Infected}\right) P\left(\text{Infected} \mid +\right)}{
P\left(+ \mid \text{Infected}\right)P\left(\text{Infected} \mid +\right) + P\left(+ \mid \text{Healthy}\right)P\left(\text{Healthy} \mid +\right)}\,, \\
&= \frac{\num{0.99} \times \num{0.288}}{\num{0.99} \times \num{0.288} + \num{0.05} \times \left(1-\num{0.288}\right)}\,, \\
&\approx \num{0.889}\,.
\end{split}
\end{equation}

\noindent Note that if we ran both tests the first time we would still have:

% https://www.wolframalpha.com/input/?i=(0.99%5E2*0.02)%2F((0.99%5E2*0.02)%2B(0.05%5E2*(1%E2%88%920.02)))
\begin{equation}\label{eq:stats:Bayes:medical_test_3}
\begin{split}
P\left(\text{Infected} \mid ++\right) &= \frac{P\left(++ \mid \text{Infected}\right) P\left(\text{Infected}\right)}{P\left(++\right)}\,, \\
&= \frac{\num{0.99}^{2} \times \num{0.02}}{\num{0.99}^{2} \times \num{0.02} + \num{0.05}^{2} \times \left(1-\num{0.02}\right)}\,, \\
&\approx \num{0.889}\,.
\end{split}
\end{equation}

%%%%%%%%%%%%%%%%%%%%%%%%%%%%%%%%%%%%%%%%%%%%%%%%%%%%%%%%
\subsection{Example: Unfair Coin}
\label{stats:Bayes:unfair_coin}

Consider the case of a bag of $n$ fair coins and $m$ unfair coins.
Let $P\left(H \mid \stcomp{F}\right) \equiv p_{H}$ be the \apriori probability of heads $H$ for an unfair coin $\stcomp{F}$.
Drawing one coin from the bag, you flip it multiple times recording $h$ heads and $t$ tails.
What is the probability you have drawn an unfair coin, $P\left(\stcomp{F} \mid h,t\right)$?

\begin{equation}\label{eq:stats:Bayes:unfair_coin_setup}
\begin{gathered}
P\left(F\right) = \frac{n}{n+m}\,,\quad P\left(\stcomp{F}\right) = \frac{m}{n+m}\,, \\
P\left(H \mid F\right) = \frac{1}{2}\,,\quad P\left(h,t \mid F\right) = \left(\frac{1}{2}\right)^{h}\,\left(\frac{1}{2}\right)^{t} = \frac{1}{2^{h+t}}\,, \\
P\left(h,t \mid \stcomp{F}\right) = P\left(H \mid \stcomp{F}\right)^{h} P\left(\stcomp{H} \mid \stcomp{F}\right)^{t} = p_{H}^{h} \left(1-p_{H}\right)^{t}.
\end{gathered}
\end{equation}

\begin{equation}\label{eq:stats:Bayes:unfair_coin_solution}
\begin{split}
P\left(\stcomp{F} \mid h,t\right) &= \frac{
P\left(h,t \mid \stcomp{F}\right) P\left(\stcomp{F}\right)}{
P\left(h,t \mid \stcomp{F}\right) P\left(\stcomp{F}\right) + P\left(h,t \mid F\right) P\left(F\right)} \\
&= \frac{
m\,p_{H}^{h} \left(1-p_{H}\right)^{t}}{
m\,p_{H}^{h} \left(1-p_{H}\right)^{t} + n\,2^{-h-t}}\,.
\end{split}
\end{equation}

Some example values are provided in \cref{tab:coin_table}.

\begin{table}[H]
\centering
\begingroup
\renewcommand*{\arraystretch}{1}
% $m = 50$, $n = 50$

\begin{tabular}{c c c c}
\hline
$h$ & $t$ & $p_{H}$ & $P\left(\stcomp{F} \mid h,t\right)$ \\
\hline
\hline
10 & 0 & 1.00 & 0.99902 \\
10 & 0 & 0.80 & 0.99099 \\
10 & 0 & 0.60 & 0.86095 \\
\hline
7 & 3 & 0.99 & 0.00095 \\
7 & 3 & 0.80 & 0.63208 \\
7 & 3 & 0.60 & 0.64722 \\
\hline
5 & 5 & 0.99 & 0.00000 \\
5 & 5 & 0.80 & 0.09696 \\
5 & 5 & 0.60 & 0.44915 \\
\hline
75 & 25 & 0.99 & 0.00000 \\
75 & 25 & 0.80 & 1.00000 \\
75 & 25 & 0.60 & 0.99970 \\
\hline
\end{tabular}
\endgroup
\caption{
$P\left(\stcomp{F} \mid h,t\right)$ for various values of $h$, $t$, and $p_{H}$ when $m = 50$, $n = 50$.
}
\label{tab:coin_table}
\end{table}

%%%%%%%%%%%%%%%%%%%%%%%%%%%%%%%%%%%%%%%%%%%%%%%%%%%%%%%%
%%%%%%%%%%%%%%%%%%%%%%%%%%%%%%%%%%%%%%%%%%%%%%%%%%%%%%%%
\section{Gaussian Distribution}
\label{stats:gaus}
% TODO

%%%%%%%%%%%%%%%%%%%%%%%%%%%%%%%%%%%%%%%%%%%%%%%%%%%%%%%%
%%%%%%%%%%%%%%%%%%%%%%%%%%%%%%%%%%%%%%%%%%%%%%%%%%%%%%%%
\section{Binomial Distribution}
\label{stats:binomial}
% TODO

%%%%%%%%%%%%%%%%%%%%%%%%%%%%%%%%%%%%%%%%%%%%%%%%%%%%%%%%
%%%%%%%%%%%%%%%%%%%%%%%%%%%%%%%%%%%%%%%%%%%%%%%%%%%%%%%%
\section{Poisson Distribution}
\label{stats:poisson}
% TODO

\subsubsection{Bernoulli Distribution}
\label{stats:poisson:bernoulli}
% TODO

%%%%%%%%%%%%%%%%%%%%%%%%%%%%%%%%%%%%%%%%%%%%%%%%%%%%%%%%
%%%%%%%%%%%%%%%%%%%%%%%%%%%%%%%%%%%%%%%%%%%%%%%%%%%%%%%%
\section{Student's \texorpdfstring{$t$}{t}-Distribution}
\label{stats:t_dist}
% TODO

%%%%%%%%%%%%%%%%%%%%%%%%%%%%%%%%%%%%%%%%%%%%%%%%%%%%%%%%
%%%%%%%%%%%%%%%%%%%%%%%%%%%%%%%%%%%%%%%%%%%%%%%%%%%%%%%%
\section{Student's \texorpdfstring{$t$}{t}-Test}
\label{stats:t_test}
% TODO

%%%%%%%%%%%%%%%%%%%%%%%%%%%%%%%%%%%%%%%%%%%%%%%%%%%%%%%%
%%%%%%%%%%%%%%%%%%%%%%%%%%%%%%%%%%%%%%%%%%%%%%%%%%%%%%%%
\section{Maximum Likelihood Estimation (MLE)}
\label{stats:MLE}
% TODO

% TODO under ``regularity assumptions'', MLE converges to the true optimal values as $n \to \inft$, is usually biased but bias is reduced as $n \to \inft$, efficient - variance approximates Cramer-Rao lower bound for larger samples

%%%%%%%%%%%%%%%%%%%%%%%%%%%%%%%%%%%%%%%%%%%%%%%%%%%%%%%%
%%%%%%%%%%%%%%%%%%%%%%%%%%%%%%%%%%%%%%%%%%%%%%%%%%%%%%%%
\section{Principle Component Analysis (PCA)}
\label{stats:PCA}
% TODO

