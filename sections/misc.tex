%%%%%%%%%%%%%%%%%%%%%%%%%%%%%%%%%%%%%%%%%%%%%%%%%%%%%%%%
%%%%%%%%%%%%%%%%%%%%%%%%%%%%%%%%%%%%%%%%%%%%%%%%%%%%%%%%
%%%%%%%%%%%%%%%%%%%%%%%%%%%%%%%%%%%%%%%%%%%%%%%%%%%%%%%%
\chapter{Miscellaneous}
\label{chap:misc}

%%%%%%%%%%%%%%%%%%%%%%%%%%%%%%%%%%%%%%%%%%%%%%%%%%%%%%%%
%%%%%%%%%%%%%%%%%%%%%%%%%%%%%%%%%%%%%%%%%%%%%%%%%%%%%%%%
\section{Feature Importance}
\label{misc:feature_importance}
% TODO

%%%%%%%%%%%%%%%%%%%%%%%%%%%%%%%%%%%%%%%%%%%%%%%%%%%%%%%%
%%%%%%%%%%%%%%%%%%%%%%%%%%%%%%%%%%%%%%%%%%%%%%%%%%%%%%%%
\section{Dimensionality Reduction \& Feature Selection}
\label{misc:m_reduction}
% TODO
% TODO Can look at correlation or mutual information between variables
% TODO could also use a chi2 test for independence between input variables and the dependent variable, see which ones might be useful. See https://scikit-learn.org/stable/modules/generated/sklearn.feature_selection.chi2.html

%%%%%%%%%%%%%%%%%%%%%%%%%%%%%%%%%%%%%%%%%%%%%%%%%%%%%%%%
%%%%%%%%%%%%%%%%%%%%%%%%%%%%%%%%%%%%%%%%%%%%%%%%%%%%%%%%
\section{Term Frequency-Inverse Document Frequency (TF-IDF)}
\label{misc:tfidf}

Term frequency-inverse document frequency (TF-IDF) is a statistic
used in natural language processing (NLP) to quantify
the importance, or uniqueness, of a term $t$ in a document $d$
with respect to a wider set of documents $D$\footnote{Note that $d$ is not necessarily an element of $D$, we can compare novel $d$ to a reference corpus $D$.}.
As the name suggests, TF-IDF is the product of two components,
one representing the frequency of the term under consideration within the document of interest,
and the other the inverse of the frequency of the term in all documents of the broader corpus.
There are a handful of definitions available for each term, but we will only describe
one of the more standard forms in this section:

\begin{subequations}\label{eq:unsupervised:tfidf}
\begin{align}
\text{tf}\left(t,d\right) &= \frac{n_{t,d}}{\sum_{t' \in d} n_{t',d}}, \label{eq:unsupervised:tfidf:tf} \\
\text{idf}\left(t,D\right) &= \log\left(\frac{\abs{D}}{1 + \abs{\left\{d' \in D \, | \, t \in d'\right\}}}\right), \label{eq:unsupervised:tfidf:idf} \\
\text{tf-idf}\left(t,d\right) &= \text{tf}\left(t,d\right) \times \text{idf}\left(t,D\right). \label{eq:unsupervised:tfidf:tfidf}
\end{align}
\end{subequations}

\noindent Here $n_{t,d}$ is number of times $t$ appears in $d$,
$\abs{D}$ is the number of documents in $D$,
and $\abs{\left\{d \in D \, | \, t \in d\right\}}$ is the number of number of documents in $D$ which contain $t$.
We take the natural log in the $\text{idf}\left(t,D\right)$ component to better accommodate large corpora of documents,
and include the constant $1+$ term to avoid divide-by-zero issues.

%%%%%%%%%%%%%%%%%%%%%%%%%%%%%%%%%%%%%%%%%%%%%%%%%%%%%%%%
%%%%%%%%%%%%%%%%%%%%%%%%%%%%%%%%%%%%%%%%%%%%%%%%%%%%%%%%
\section{Prevalence Ratio}
\label{misc:PrevalenceRatio}

The prevalence ratio (PR) is a similar concept to TF-IDF,
allowing us to identify important characteristics of a population of interest versus the wider population.
For example, we can use a PR to assess the association of hypertension with heart failure by comparing
the prevalence of hypertension in a cohort of heart failure patients
to the prevalence of hypertension in the general population,
or at least a representative sample\footnote{Defining an acceptable representative sample is often the hardest part of the analysis and may require stratification, \eg stratifying by age when investigating disease.} of it.

The prevalence of a characteristic $c$ in a population $P$ is $\abs{\{p' \in P \, | \, c \in p'\}} \, / \, \abs{p' \in P}$,
note the similarity to $\text{tf}\left(t,d\right)$ \cref{eq:unsupervised:tfidf:tf}.
The prevalence ratio is then

\begin{equation}\label{eq:unsupervised:PR}
\text{PR}\left(t,d\right) = \frac{ \abs{\{p' \in P \, | \, c \in p'\}} \, / \, \abs{p' \in P} }{ \abs{\{p' \in P_{0} \, | \, c \in p'\}} \, / \, \abs{p' \in P_{0}} },
\end{equation}

\noindent where $P$ is the population of interest and $P_{0}$ is the wider population.

Note that the PR is mathematically identical to the relative risk, or hazard ratio (HR) discussed in \cref{chap:survival}.
There is a similar discussion in the literature \cite{pmid27460748,10.3389/fvets.2017.00193}
on prevalence ratios versus odds ratios as is described in \cref{survival:additional:odds}.

% See also: https://sph.unc.edu/wp-content/uploads/sites/112/2015/07/nciph_ERIC8.pdf

%%%%%%%%%%%%%%%%%%%%%%%%%%%%%%%%%%%%%%%%%%%%%%%%%%%%%%%%
%%%%%%%%%%%%%%%%%%%%%%%%%%%%%%%%%%%%%%%%%%%%%%%%%%%%%%%%
\section{Factor Analysis}
\label{misc:factor_ana}
% TODO

%%%%%%%%%%%%%%%%%%%%%%%%%%%%%%%%%%%%%%%%%%%%%%%%%%%%%%%%
%%%%%%%%%%%%%%%%%%%%%%%%%%%%%%%%%%%%%%%%%%%%%%%%%%%%%%%%
\section{Information Theory \& Entropy}
\label{misc:info_theory}
% TODO
% TODO Entropy, significance of bits

%%%%%%%%%%%%%%%%%%%%%%%%%%%%%%%%%%%%%%%%%%%%%%%%%%%%%%%%
%%%%%%%%%%%%%%%%%%%%%%%%%%%%%%%%%%%%%%%%%%%%%%%%%%%%%%%%
\section{Modulo Operations}
\label{misc:modulo}

\begin{subequations}\label{eq:misc:modulo}
\begin{align}
\left(A~\text{mod}~C\right)~\text{mod}~C &= A~\text{mod}~C, \label{eq:misc:modulo:basic} \\
\left(A \pm B\right)~\text{mod}~C &= \left(A~\text{mod}~C \pm B~\text{mod}~C\right)~\text{mod}~C, \label{eq:misc:modulo:pm} \\
\left(A \times B\right)~\text{mod}~C &= \left(A~\text{mod}~C \times B~\text{mod}~C\right)~\text{mod}~C, \label{eq:misc:modulo:multiplication} \\
A^{B}~\text{mod}~C &= \left(\left(A~\text{mod}~C\right)^{B}\right)~\text{mod}~C. \label{eq:misc:modulo:exp}
\end{align}
\end{subequations}

%%%%%%%%%%%%%%%%%%%%%%%%%%%%%%%%%%%%%%%%%%%%%%%%%%%%%%%%
%%%%%%%%%%%%%%%%%%%%%%%%%%%%%%%%%%%%%%%%%%%%%%%%%%%%%%%%
\section{Mathematical Relations}
\label{misc:math}

\begin{subequations}\label{eq:misc:math}
\begin{align}
e^{x} &= \sum_{k=0}^{\infty} \frac{x^{k}}{k!} \label{eq:misc:math:ex} \\
e^{x} &= \lim_{n \to \infty} \left(1 + \frac{x}{n}\right)^{n} \label{eq:misc:math:ex_limit} \\
\frac{1}{1-r} &= \sum_{k=0}^{\infty} r^{k}, \quad \abs{r} < 1 \label{eq:misc:math:geometric} \\
e^{-i x} &= \cos\left(x\right) + i \sin\left(x\right) \label{eq:misc:math:eulers_formula} \\
2 \cos\left(x\right) &= e^{i x} + e^{-i x} \label{eq:misc:math:cos_exp} \\
2i \sin\left(x\right) &= e^{i x} - e^{-i x} \label{eq:misc:math:sin_exp} \\
\abs{\mathbf{u} + \mathbf{v}} &\leq \abs{\mathbf{u}} + \abs{\mathbf{v}} \label{eq:misc:math:triangle_inequality} \\
\abs{\braket{u}{v}}^{2} &\leq \braket{u} \braket{v} \label{eq:misc:math:cauchy_schwarz_inequality} \\
n^{k} &= N\left(\text{Permutations Length $k$, with Repetition}\right) \label{eq:misc:math:permutations_with_repetition} \\
\frac{n!}{\left(n-k\right)!} &= N\left(\text{Permutations Length $k$, without Repetition}\right) \label{eq:misc:math:permutations_without_repetition} \\
\frac{n!}{k!\left(n-k\right)!} &= {n \choose k} = N\left(\text{Combinations of Length $k$ from $n$}\right) \label{eq:misc:math:binomial_coefficient} \\
\overline{A \cap B} &= \overline{A} \cup \overline{B},\quad \overline{A \cup B} = \overline{A} \cap \overline{B} \label{eq:misc:math:de_morgan} \\
\dv{}{x}\, a^{x} &= a^{x} \ln\left(a\right), \quad 0 < a \label{eq:misc:math:diff_exp_gen} \\
\dv{}{x}\, \log_{a}\left(x\right) &= \frac{1}{x \ln a} \label{eq:misc:math:diff_log_gen}
\end{align}
\end{subequations}
