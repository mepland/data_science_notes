%%%%%%%%%%%%%%%%%%%%%%%%%%%%%%%%%%%%%%%%%%%%%%%%%%%%%%%%
%%%%%%%%%%%%%%%%%%%%%%%%%%%%%%%%%%%%%%%%%%%%%%%%%%%%%%%%
%%%%%%%%%%%%%%%%%%%%%%%%%%%%%%%%%%%%%%%%%%%%%%%%%%%%%%%%
\chapter{\pyspark}
\label{pyspark}

%%%%%%%%%%%%%%%%%%%%%%%%%%%%%%%%%%%%%%%%%%%%%%%%%%%%%%%%
%%%%%%%%%%%%%%%%%%%%%%%%%%%%%%%%%%%%%%%%%%%%%%%%%%%%%%%%
\section{Basic Commands}
\label{pyspark:basic}

\begin{lstlisting}[language=Python]
import pyspark.sql.functions as F
from pyspark.sql.window import Window
from pyspark.sql.types import IntegerType, DoubleType

# IO
df = spark.read.parquet('s3a://bucket/table')
df = df.withColumn('col', F.col('col').cast(IntegerType()))
df.write.save('s3a://bucket/table.parquet')

# descriptive commands
df.describe().show(); df.printSchema();
df.show(10); df.limit(10).toPandas();

# select columns
df_y = df.select('x', 'y')

# get distinct values
df_y_distinct = df.select('y').distinct()

# select by value
df_selection = df.where( ((F.col('x') == x_value) & (F.col('y') == y_value)) | (F.col('z') < z_value) )
df_selection = df.where( F.col('x').isNotNull() )

# select by value and assign new value
df = df.withColumn('y', F.when( F.col('x') == x_value, F.lit(y_value) ).otherwise(F.col('y')))

# select where isin, and not (~) isin, some_list
df_selection = df.where(F.col('x').isin(some_list))
df_selection = df.where(~F.col('x').isin(some_list))

# apply an arbitrary function - slow unless written in scala!
def func(x, y):
	return x*np.sin(y)
func_udf = F.udf(func, DoubleType())
df = df.withColumn('z', func_udf('x', 'y'))

# rename columns
df = df.withColumnRenamed('old', 'new')

# order by
df = df.orderBy(['x', 'y'], ascending=[True, False])

# group by, while counting rows and aggregating max date
df = df.groupBy('x', 'y', 'z').agg(F.count('*').alias('count'), F.max('date_col').alias('max_date'))

# group by, get mode of state per patient - deterministically (alphabetical order)
# can be expanded to additional columns, each with their own .join(df.groupBy()...) statements
df.select('patient').distinct().join(
df.where(F.col('state').isNotNull())
	.groupBy(['patient', 'state']).count().alias('c')
	.withColumn('row_num', F.row_number().over(Window().partitionBy('patient').orderBy(F.col('c').desc(), F.col('state'))))
	.where(F.col('row_num') == 1)
	.select('patient', 'state')
, 'patient', 'left')

# return duplicate rows
df.join(df, df.groupBy('x', 'y').agg(F.count('*').alias('c')).where(1 < F.('c')), ['x', 'y'], 'left_semi')

# drop columns
df = df.drop('col_to_drop1', 'col_to_drop2')

cols_to_drop = ['col1', 'col2']
df = df.drop(*cols_to_drop)

# fill nans, for all columns and per column - other syntaxes are also available
df = df.fillna(0.0)
df = df.fillna({'x': x_nan_value, 'z': y_nan_value})

# run a SQL query, note you must register the needed dataframes as tables first
df.registerTempTable('df')
spark.sql('select * from df limit 10').show(10)
\end{lstlisting}

%%%%%%%%%%%%%%%%%%%%%%%%%%%%%%%%%%%%%%%%%%%%%%%%%%%%%%%%
%%%%%%%%%%%%%%%%%%%%%%%%%%%%%%%%%%%%%%%%%%%%%%%%%%%%%%%%
\section{Joining}
\label{pyspark:join}

\noindent See the
\href{https://spark.apache.org/docs/latest/api/python/reference/api/pyspark.sql.DataFrame.join.html#pyspark-sql-dataframe-join}{documentation}
and this
\href{http://www.learnbymarketing.com/1100/pyspark-joins-by-example/}{useful guide}.
In addition to the standard types of joins \texttt{left\_semi}, or \texttt{leftsemi},
is very useful for filtering the left table to the matching rows in the join condition,
without actually joining any columns from the right table.

\begin{lstlisting}[language=Python]
df = df_l.join(df_r, df_l['id_left'] == df_r['id_right'], 'left')
\end{lstlisting}

%%%%%%%%%%%%%%%%%%%%%%%%%%%%%%%%%%%%%%%%%%%%%%%%%%%%%%%%
%%%%%%%%%%%%%%%%%%%%%%%%%%%%%%%%%%%%%%%%%%%%%%%%%%%%%%%%
\section{Pivoting}
\label{pyspark:pivoting}

\noindent \href{https://spark.apache.org/docs/latest/api/python/reference/api/pyspark.sql.GroupedData.pivot.html#pyspark-sql-groupeddata-pivot}{\texttt{pivot} documentation},
and
\href{https://sparkbyexamples.com/pyspark/pyspark-pivot-and-unpivot-dataframe/}{examples}.

\begin{lstlisting}[language=Python]
df.groupBy('product').pivot('state').sum('cost')
\end{lstlisting}
