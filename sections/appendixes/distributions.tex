%%%%%%%%%%%%%%%%%%%%%%%%%%%%%%%%%%%%%%%%%%%%%%%%%%%%%%%%
%%%%%%%%%%%%%%%%%%%%%%%%%%%%%%%%%%%%%%%%%%%%%%%%%%%%%%%%
\chapter{Probability Distributions}
\label{dist}

\begin{figure}
  \centering
  \savebox{\largestimage}{
    \includegraphics[width=0.47\textwidth]{figures/stats/dist/poisson_pmf.pdf}
  }% Store largest image in a box

  \begin{subfigure}[b]{\wd\largestimage}\centering
    \raisebox{\dimexpr.5\ht\largestimage-.5\height}{% Adjust vertical height of smaller image
      \includegraphics[width=\textwidth]{figures/stats/dist/binomial_pmf.pdf}}
  \caption{Binomial Distribution PMF}
  \label{fig:dist:binomial}
  \end{subfigure}
  ~
  \begin{subfigure}[b]{0.48\textwidth}\centering
    \usebox{\largestimage}
  \caption{Poisson Distribution PMF}
  \label{fig:dist:poisson}
  \end{subfigure}
\caption{
Binomial and Poisson distribution PMFs,
by \href{https://en.wikipedia.org/wiki/File:Binomial_distribution_pmf.svg}{Tayste}
and \href{https://en.wikipedia.org/wiki/File:Poisson_pmf.svg}{Skbkekas}.
Both plots have $k$ on the $x$-axis and $P$ on the $y$-axis, and curves for various $n$ and $p$, or $\lambda$.
\label{fig:dist:binomial_poisson}
}
\end{figure}

\begin{figure}
  \centering
  \savebox{\largestimage}{
    \includegraphics[width=0.47\textwidth]{figures/stats/dist/student_t_pdf.pdf}
  }% Store largest image in a box

  \begin{subfigure}[b]{0.48\textwidth}\centering
    \raisebox{\dimexpr.5\ht\largestimage-.5\height}{% Adjust vertical height of smaller image
      \includegraphics[width=\textwidth]{figures/stats/dist/gaussian_pdf.pdf}}
  \caption{Gaussian Distribution PDF}
  \label{fig:dist:gaus}
  \end{subfigure}
  ~
  \begin{subfigure}[b]{\wd\largestimage}\centering
    \usebox{\largestimage}
  \caption{Student's $t$-Distribution PDF}
  \label{fig:dist:student_t}
  \end{subfigure}
\caption{
Gaussian and Student's $t$-distribution PDFs,
by \href{https://en.wikipedia.org/wiki/File:Normal_Distribution_PDF.svg}{Inductiveload}
and \href{https://en.wikipedia.org/wiki/File:Student_t_pdf.svg}{Skbkekas}.
Note that in the limit $\nu \to \infty$ the Student's $t$-distribution
approaches the standard normal distribution shown in red.
\label{fig:dist:gaus_student_t}
}
\end{figure}


\begin{figure}
\centering
\includegraphics[width=0.7\textwidth]{figures/stats/dist/chi2_pdf.pdf}
\caption{
$\chi^{2}$-distribution PDF,
by \href{https://en.wikipedia.org/wiki/File:Chi-square_pdf.svg}{Geek3}.
Here $k$ is being used in lieu of $\nu$ for the number of degrees of freedom.
}
\label{fig:dist:chi2}
\end{figure}
