%%%%%%%%%%%%%%%%%%%%%%%%%%%%%%%%%%%%%%%%%%%%%%%%%%%%%%%%
%%%%%%%%%%%%%%%%%%%%%%%%%%%%%%%%%%%%%%%%%%%%%%%%%%%%%%%%
%%%%%%%%%%%%%%%%%%%%%%%%%%%%%%%%%%%%%%%%%%%%%%%%%%%%%%%%
\chapter{\sql}
\label{sql}

%%%%%%%%%%%%%%%%%%%%%%%%%%%%%%%%%%%%%%%%%%%%%%%%%%%%%%%%
%%%%%%%%%%%%%%%%%%%%%%%%%%%%%%%%%%%%%%%%%%%%%%%%%%%%%%%%
\section{Introduction}
\label{sql:intro}

\sql, or Structured Query Language, is way to
communicate with a Relational Database Management System (RDBMS).
There data is stored as collection of tables with
at least one common column to allow relational operators
to join information across tables.
In some \sql implementations, such as MySQL, each table must have an unique primary key for each row.
When a column in one table relates to the primary key of another, it is known as a foreign key.
A \sql query returns information from the database as a result set, and may contain subqueries.

When evaluating a \sql statement, each clause is
\href{https://cloud.google.com/bigquery/docs/reference/standard-sql/query-syntax#where_clause}{generally executed in this order}:

\begin{SQLcode}
FROM
WHERE
GROUP BY
HAVING
WINDOW
QUALIFY
DISTINCT
ORDER BY
LIMIT
\end{SQLcode}

%%%%%%%%%%%%%%%%%%%%%%%%%%%%%%%%%%%%%%%%%%%%%%%%%%%%%%%%
%%%%%%%%%%%%%%%%%%%%%%%%%%%%%%%%%%%%%%%%%%%%%%%%%%%%%%%%
\section{Basic Commands}
\label{sql:basic}

\begin{SQLcode}
-- create a new table, define columns & types
CREATE OR REPLACE TABLE t (id INT PRIMARY KEY
	, name VARCHAR(20) NOT NULL, state CHAR(2), dob DATE);

-- manually insert new rows
INSERT INTO t VALUES (0, 'Matt', 'NJ', '1990-1-2');
INSERT INTO t (id, name, state, dob) VALUES
                     (1, 'Jamie','NJ', '1990-3-4');
INSERT INTO t VALUES (2,'Mary','IL', '1970-5-6')
	, (3,'Eddie','IL','2010-7-8'),(4,'Bob','ND','1980-9-10');

-- insert rows from another table
INSERT INTO t SELECT * FROM t_other WHERE zip = 11111;

-- update row
UPDATE t SET name = 'Mar' WHERE id = 2;

/* SELECT with WHERE, ORDER BY, LIMIT
comparison ops: >, >=, =, <> or !=, BETWEEN, LIKE, IN
logical ops: AND, OR, NOT */
SELECT id,name FROM t WHERE 0 < id ORDER BY name LIMIT 10;
SELECT id,name FROM t WHERE id BETWEEN 1 AND 3;
SELECT id,name FROM t WHERE name LIKE 'M%';--starts with M
SELECT id,name FROM t WHERE state LIKE 'N_';-- N + 1 char
SELECT id,name FROM t WHERE (id IN (0,3)) OR (name='Bob');
SELECT name,dob FROM t WHERE 1985<YEAR(dob) ORDER BY dob;

-- aggregation commands
SELECT COUNT(*) AS n_rows FROM t;
SELECT COUNT(id) AS n_rows FROM t; -- only counts rows with non-null id values
SELECT COUNT(DISTINCT id) AS n_distinct_ids FROM t;
SELECT MIN(id) FROM t; -- AVG, MODE, SUM, ...

-- find unique / distinct values
SELECT DISTINCT name FROM t;

-- when finding nulls, use IS; can't use =, <>
SELECT id, name FROM t WHERE state IS NULL;

-- GROUP BY, use HAVING, not WHERE
SELECT state, COUNT(*) AS n_rows
FROM t
GROUP BY state
HAVING 1 < n_rows
ORDER BY n_rows DESC;

-- CTEs
WITH state_counts AS (
	SELECT COUNT(DISTINCT id) AS n_distinct_ids, state
	FROM t GROUP BY state
)
SELECT AVG(n_distinct_ids)
FROM state_counts WHERE state != 'NC';

-- CASE
SELECT name
	, CASE
		WHEN state = 'IL' THEN 'bears fan'
		WHEN state = 'ND' THEN 'bison fan'
		ELSE CONCAT(state, ' (unknown fan)')
	END AS fandom
FROM t;

-- fill nulls, can also use IFNULL
SELECT COALESCE(state, 'Unknown') AS state FROM t;

-- GROUP BY CUBE in Snowflake, not available in Google Big Query
-- GROUP BY ROLLUP is similar
-- https://cloud.google.com/bigquery/docs/reference/standard-sql/query-syntax#group_by_clause
-- https://stackoverflow.com/a/65590293
WITH b AS (
	-- taking DISTINCT before the GROUP BY can help reduce some duplicated computation later in COUNT(DISTINCT id), but is not required
	SELECT DISTINCT id
		, COALESCE(LEFT(UPPER(TRIM(name)), 1), ' ') AS first_letter
		-- must not have any nulls in the field to be GROUP BY CUBE(), or there will be multiple null rows in output
		, COALESCE(state, 'Unknown') AS state
	FROM t
)
-- format output
SELECT COALESCE(state, 'All') AS "State"
	, first_letter AS "First Letter"
	, n_patients AS "# Patients"
FROM (
	SELECT state, first_letter
		, COUNT(DISTINCT id) AS n_patients
	FROM b
	-- can mix regular GROUP BY and CUBE, but not ROLLUP
	GROUP BY CUBE (state), first_letter
)
ORDER BY "State", "First Letter";

-- LAG/LEAD
WITH b AS (
    SELECT state, sales_year, sales
    FROM t
)
, l AS (
    SELECT state, sales_year, sales
		, LAG(sales) OVER (PARTITION BY state ORDER BY sales_year ASC) AS sales_year_prior
    FROM b
)
SELECT state, sales_year, sales, sales_year_prior
	, sales - sales_year_prior AS delta_sales
FROM l
ORDER BY state, sales_year;

-- working with arrays, in Snowflake
WITH b0 AS (
	SELECT id, state
		, ARRAY_CONSTRUCT_COMPACT(D1,D2,D3,D4,D5) AS dx_array
	FROM c
	WHERE ARRAYS_OVERLAP(dx_array, ARRAY_CONSTRUCT('W5602', 'W5551XD', 'W5803XA'))
)
, b1 AS (
	SELECT id, state, v.VALUE AS dx
	FROM b0
	, LATERAL FLATTEN(INPUT => dx_array, OUTER => TRUE) AS v
)
SELECT state, dx, COUNT(DISTINCT id) AS n_patients
FROM b1
GROUP BY 1,2
ORDER BY 1,2;

-- working with arrays, in Google Big Query
-- https://cloud.google.com/bigquery/docs/arrays
-- https://cloud.google.com/bigquery/docs/reference/standard-sql/array_functions
WITH b0 AS (
	SELECT id, state
		-- will have nulls, unlike ARRAY_CONSTRUCT_COMPACT
		, [D1,D2,D3,D4,D5] AS dx_array
	FROM c
)
, b1 AS (
	SELECT *
	FROM b0
	-- no good ARRAYS_OVERLAP alternative
	WHERE 'W5602' IN UNNEST(dx_array)
)
, b2 AS (
	SELECT id, state, dx
	FROM b1
	CROSS JOIN UNNEST(dx_array) AS dx
)
SELECT state, dx, COUNT(DISTINCT id) AS n_patients
FROM b1
GROUP BY 1,2
ORDER BY 1,2;

-- generate rows, in Google Big Query
SELECT num
FROM UNNEST(GENERATE_ARRAY(1, 10)) AS num;
\end{SQLcode}

%%%%%%%%%%%%%%%%%%%%%%%%%%%%%%%%%%%%%%%%%%%%%%%%%%%%%%%%
%%%%%%%%%%%%%%%%%%%%%%%%%%%%%%%%%%%%%%%%%%%%%%%%%%%%%%%%
\section{Intermediate Recipes}
\label{ssql:intermediate_recipes}

\begin{SQLcode}
-- GROUP BY, get mode of state per person - deterministically (alphabetical order)
-- can be expanded to additional columns, each with their own g_state CTEs
WITH p AS ( SELECT DISTINCT patient FROM p0 )
, g_state AS (
	-- note COUNT(state) will ignore null values in state
	SELECT patient, state, COUNT(state) AS c
	FROM p0 GROUP BY patient, state
	-- QUALIFY is the WHERE statement for windows functions
	-- can use RANK() instead of ROW_NUMBER() to allow ties
	QUALIFY ROW_NUMBER() OVER (PARTITION BY patient ORDER BY c DESC, state ASC) = 1
)
SELECT p.patient, state
FROM p
LEFT JOIN g_state ON p.patient = g_state.patient;

-- return duplicate rows
SELECT a.*, n_dup_rows
FROM t AS a
INNER JOIN (
	SELECT state, COUNT(*) AS n_dup_rows
	FROM t
	GROUP BY state
	HAVING 1 < n_dup_rows
) AS b
	ON a.state = b.state;

-- return 5 examples each from the 20 most duplicated rows
WITH t_in AS (
	SELECT *
		, CONCAT(
			COALESCE(CAST(key_1_to_dedup_over AS STRING), ''),
			'|',
			COALESCE(CAST(key_2_to_dedup_over AS STRING), '')
		) AS dup_key
	FROM t_source
)
, g_0 AS (
	SELECT dup_key, COUNT(*) AS n_dup_rows
	FROM t_in
	GROUP BY dup_key
	HAVING 1 < n_dup_rows
)
, g AS (
	SELECT dup_key
		, ROW_NUMBER() OVER (ORDER BY n_dup_rows DESC, dup_key ASC) AS dup_id
		, n_dup_rows
	FROM g_0
)
, t_1 AS (
	SELECT g.dup_id, g.n_dup_rows, t_in.*
	FROM t_in
	INNER JOIN g
		ON t_in.dup_key = g.dup_key
)
, t_out AS (
	SELECT dup_id
		, ROW_NUMBER() OVER (PARTITION BY dup_key ORDER BY order_by_col ASC) AS dup_row_number
		, n_dup_rows, t_1.* EXCEPT(dup_id, n_dup_rows, dup_key)
	FROM t_1
)
SELECT *
FROM t_out
WHERE dup_id <= 20 AND dup_row_number <= 5
ORDER BY dup_id ASC, dup_row_number ASC;
\end{SQLcode}

%%%%%%%%%%%%%%%%%%%%%%%%%%%%%%%%%%%%%%%%%%%%%%%%%%%%%%%%
%%%%%%%%%%%%%%%%%%%%%%%%%%%%%%%%%%%%%%%%%%%%%%%%%%%%%%%%
\section{Joining}
\label{sql:join}

\begin{SQLcode}
SELECT col_1, col_2
FROM table1
LEFT JOIN table2
	ON table1.col_a = table2.col_b;

-- COALESCE appropriately when using a FULL OUTER JOIN
SELECT COALESCE(a.id, b.id) AS id
	, a.field AS field_a
	, b.field AS field_b
FROM a
FULL OUTER JOIN b
	ON a.id = b.id;
\end{SQLcode}

The three common join types are the standard
\texttt{INNER JOIN}, \texttt{LEFT JOIN},
and \texttt{FULL OUTER JOIN}\footnote{Note that
\texttt{FULL JOIN} can also be used as an equivalent to \texttt{FULL OUTER JOIN}.} as
shown in \cref{fig:sql:joins}.

\begin{figure}[H]
\centering
  \begin{subfigure}[c]{0.3\textwidth}\centering
  \includegraphics[width=\textwidth]{figures/sql/left_join}
  %\caption{}
  \label{fig:sql:joins:left_join}
  \end{subfigure}
  ~
  \begin{subfigure}[c]{0.3\textwidth}\centering
  \includegraphics[width=\textwidth]{figures/sql/inner_join}
  %\caption{}
  \label{fig:sql:joins:inner_join}
  \end{subfigure}
  ~
  \begin{subfigure}[c]{0.3\textwidth}\centering
  \includegraphics[width=\textwidth]{figures/sql/full_outer_join}
  %\caption{}
  \label{fig:sql:joins:full_outer_join}
  \end{subfigure}
\caption{
Illustration of common types of joins,
adapted from \href{http://stevestedman.com/2015/03/sql-server-join-types-poster-version-2}{Steve Stedman}.
}
\label{fig:sql:joins}
\end{figure}

%%%%%%%%%%%%%%%%%%%%%%%%%%%%%%%%%%%%%%%%%%%%%%%%%%%%%%%%
%%%%%%%%%%%%%%%%%%%%%%%%%%%%%%%%%%%%%%%%%%%%%%%%%%%%%%%%
\section{Pivoting}
\label{ssql:pivoting}

\noindent \href{https://docs.snowflake.com/en/sql-reference/constructs/pivot.html}{Snowflake \texttt{PIVOT} documentation}.

\begin{SQLcode}
WITH b AS (
	SELECT LEFT(name, 1) AS first_letter, state
		, DATEDIFF(DAY, dob, CURRENT_DATE)/365 AS age
	FROM t
)
SELECT first_letter AS "First Letter"
	, "'NJ'" AS "Avg Age NJ"
	, "'IL'" AS "Avg Age IL"
	, "'ND'" AS "Avg Age ND"
FROM b
PIVOT(AVG(age) FOR state IN ('NJ','IL','ND'))
ORDER BY first_letter;
\end{SQLcode}

%%%%%%%%%%%%%%%%%%%%%%%%%%%%%%%%%%%%%%%%%%%%%%%%%%%%%%%%
%%%%%%%%%%%%%%%%%%%%%%%%%%%%%%%%%%%%%%%%%%%%%%%%%%%%%%%%
\section{IO Commands (MySQL)}
\label{sql:io}

\begin{SQLcode}
-- admin, setup new user
GRANT ALL PRIVILEGES ON *.* TO 'user'@'localhost'
	IDENTIFIED BY 'pw';

-- create a new database
CREATE DATABASE mydb; USE mydb;

-- create a new schema
CREATE SCHEMA myschema; USE SCHEMA myschema;

-- load a SQL dump
SET AUTOCOMMIT=0; SOURCE dump.sql; COMMIT;

-- load csv file into table
LOAD DATA LOCAL INFILE 'in.csv' INTO TABLE t_new
	FIELDS TERMINATED BY ',' LINES TERMINATED BY '\n';

-- rename column, change type, add new column
ALTER TABLE t RENAME COLUMN old_col_name TO new_col_name;
ALTER TABLE t MODIFY col_name new_type;
ALTER TABLE t ADD new_col DOUBLE;

-- delete all rows of a table, keep structure
TRUNCATE TABLE t;

-- delete a row, column, table, database, schema, ...
DELETE FROM t WHERE id = 4;

SHOW COLUMNS FROM t; /* or, also for MySQL */ DESCRIBE t;
SHOW COLUMNS FROM t WHERE type ILIKE 'VARCHAR%';
ALTER TABLE t DROP COLUMN name;

SHOW TABLES;
DROP TABLE IF EXISTS 't';

SHOW DATABASES;
DROP DATABASE mydb;
\end{SQLcode}

\begin{lstlisting}[language=bash]
# export selection to csv (from shell, no file perms)
mysql -u user --password=pw --database=mydb
 --execute='SELECT ...;' -q -n -B -r > out.csv
 && sed -i '/\t/ s//,/g' out.csv

# load table from csv (from shell)
mysqlimport --ignore-lines=1 --fields-terminated-by=,
 --verbose --local -u user -p mydb /path/to/in.csv
\end{lstlisting}
