%%%%%%%%%%%%%%%%%%%%%%%%%%%%%%%%%%%%%%%%%%%%%%%%%%%%%%%%
%%%%%%%%%%%%%%%%%%%%%%%%%%%%%%%%%%%%%%%%%%%%%%%%%%%%%%%%
%%%%%%%%%%%%%%%%%%%%%%%%%%%%%%%%%%%%%%%%%%%%%%%%%%%%%%%%
\chapter{\sql}
\label{sql}

%%%%%%%%%%%%%%%%%%%%%%%%%%%%%%%%%%%%%%%%%%%%%%%%%%%%%%%%
%%%%%%%%%%%%%%%%%%%%%%%%%%%%%%%%%%%%%%%%%%%%%%%%%%%%%%%%
\section{Introduction}
\label{sql:intro}

\sql, or Structured Query Language, is way to
communicate with a RDBMS, or Relational Database Management System.
There data is stored as collection of tables with
at least one common column to allow relational operators
to join information across tables.
Each SQL table must have an unique primary key for each row.
When a column in one table relates to the primary key of another, it is known as a foreign key.
A SQL query returns information from the database as a result set, and may contain subqueries.
Note, in the next section for the ``select top 5 steam games\ldots'' query
see \href{https://www.databasejournal.com/features/mysql/selecting-the-top-n-results-by-group-in-mysql.html}{here},
and for more on \texttt{DATETIME}
see \href{https://www.techotopia.com/index.php/Working_with_Dates_and_Times_in_MySQL}{here}.

%%%%%%%%%%%%%%%%%%%%%%%%%%%%%%%%%%%%%%%%%%%%%%%%%%%%%%%%
%%%%%%%%%%%%%%%%%%%%%%%%%%%%%%%%%%%%%%%%%%%%%%%%%%%%%%%%
\section{Basic Commands (MySQL)}
\label{sql:basic}

\begin{lstlisting}[language=SQL]
-- create a new table, define columns & types
CREATE TABLE t (id INT PRIMARY KEY,
 name VARCHAR(20) NOT NULL, state CHAR(2), dob DATE);

-- manually insert new rows
INSERT INTO t VALUES (0, 'Matt', 'NJ', '1990-1-2');
INSERT INTO t (id, name, state) VALUES
                     (1, 'Jamie','NJ', '1990-3-4');
INSERT INTO t VALUES (2,'Marry','IL', '1970-5-6'),
 (3,'Eddie','IL','2010-7-8'),(4,'Bob','ND','1980-9-10');

-- insert rows from another table
INSERT INTO t SELECT * FROM other_t WHERE zip = 11111;

/* select with where, order by, limit
comparison ops: >, >=, =, <>, BETWEEN, LIKE, IN
logical ops: AND, OR, NOT */
SELECT id,name FROM t WHERE id > 0 ORDER BY name LIMIT 10;
SELECT id,name FROM t WHERE id BETWEEN 1 AND 3;
SELECT id,name FROM t WHERE name LIKE 'M%';--starts with M
SELECT id,name FROM t WHERE state LIKE 'N_';-- N+ 1 char
SELECT id,name FROM t WHERE (id IN (0,3)) OR (name='Bob');
SELECT name,dob FROM t WHERE YEAR(dob)>1985 ORDER BY dob;

-- aggregation commands
SELECT count(1) AS 'nrows' FROM t;
SELECT count(id) FROM t WHERE id NOT IN (0, 1);
SELECT MIN(id) FROM t;
-- plus MAX, AVG, and SUM

-- select unique / distinct values
SELECT DISTINCT name FROM t;

-- finding nulls, use IS, NOT IS; can't use =, <>
SELECT id,name FROM t WHERE state IS NULL;

-- GROUP BY
SELECT state,COUNT(id) FROM t GROUP BY state
 ORDER BY COUNT(id) DESC;

-- Use HAVING with an aggregate func, not WHERE
SELECT state,COUNT(id) FROM t GROUP BY state
 HAVING COUNT(id) > 5;

-- WITH / subqueries
WITH state_counts AS (
 SELECT count(id) AS 'count',state FROM t GROUP BY state )
 SELECT AVG(count) FROM state_counts WHERE state <> 'NC';

-- update row
UPDATE t SET name = 'Mar' WHERE id = 2;

-- CASE
SELECT name,
  CASE
    WHEN state IS 'IL' THEN 'bears fan'
    WHEN state IS 'ND' THEN 'bison fan'
    ELSE CONCAT(state, ' (unknown)')
  END
FROM t;

-- return duplicate rows using a join
SELECT  t1.*, totalCount AS Nduplicates
FROM    t t1
        INNER JOIN
        (
            SELECT  state, COUNT(*) totalCount
            FROM    t
            GROUP   BY state
            HAVING  COUNT(*) >= 2
        ) b ON a.state = b.state

-- Select the top 5 steam games per user, min of 2h play
SELECT steamid, appid, playtime_forever
 FROM
 (
   SELECT steamid, appid, playtime_forever,
   @steamid_rank := IF(@current_steamid = steamid,
                         @steamid_rank + 1,
                         1
                      ) AS steamid_rank,
   @current_steamid := steamid
   FROM Games_1
   WHERE playtime_forever > 120
   ORDER BY steamid, playtime_forever DESC
 ) ranked
 WHERE steamid_rank <= 5;
\end{lstlisting}

%%%%%%%%%%%%%%%%%%%%%%%%%%%%%%%%%%%%%%%%%%%%%%%%%%%%%%%%
%%%%%%%%%%%%%%%%%%%%%%%%%%%%%%%%%%%%%%%%%%%%%%%%%%%%%%%%
\section{Joining}
\label{sql:join}

\begin{lstlisting}[language=SQL]
SELECT column1, column2 FROM
 table1 LEFT JOIN table2 ON
 table1.column_nameA = table2.column_nameB
 WHERE table1.column_nameA > 0; -- WHERE is not required, but it can be included here
\end{lstlisting}

The three common join types are the standard
\texttt{INNER JOIN}, \texttt{LEFT JOIN},
and \texttt{FULL OUTER JOIN}\footnote{Note that \texttt{FULL JOIN} can also be used as an equivalent to \texttt{FULL OUTER JOIN}.} as
shown in \cref{fig:sql:joins}.


\begin{figure}[H]
\centering
  \begin{subfigure}[c]{0.3\textwidth}\centering
  \includegraphics[width=\textwidth]{figures/sql/left_join.pdf}
  %\caption{}
  \label{fig:sql:joins:left_join}
  \end{subfigure}
  ~
  \begin{subfigure}[c]{0.3\textwidth}\centering
  \includegraphics[width=\textwidth]{figures/sql/inner_join.pdf}
  %\caption{}
  \label{fig:sql:joins:inner_join}
  \end{subfigure}
  ~
  \begin{subfigure}[c]{0.3\textwidth}\centering
  \includegraphics[width=\textwidth]{figures/sql/full_outer_join.pdf}
  %\caption{}
  \label{fig:sql:joins:full_outer_join}
  \end{subfigure}
\caption{
Illustration of common types of joins, adapted from \href{http://stevestedman.com/2015/03/sql-server-join-types-poster-version-2}{Steve Stedman}.
}
\label{fig:sql:joins}
\end{figure}

%%%%%%%%%%%%%%%%%%%%%%%%%%%%%%%%%%%%%%%%%%%%%%%%%%%%%%%%
%%%%%%%%%%%%%%%%%%%%%%%%%%%%%%%%%%%%%%%%%%%%%%%%%%%%%%%%
\section{IO Commands}
\label{sql:io}

\begin{lstlisting}[language=SQL]
-- admin, setup new user
GRANT ALL PRIVILEGES ON *.* TO 'user'@'localhost'
 IDENTIFIED BY 'pw';

-- create a new database
CREATE DATABASE mydb; USE mydb;

-- load a SQL dump
SET autocommit=0; SOURCE dump.sql; COMMIT;

-- load csv file into table
LOAD DATA LOCAL INFILE 'in.csv' INTO TABLE tnew
 FIELDS TERMINATED BY ',' LINES TERMINATED BY '\n';

-- rename column, change type, add new column
ALTER TABLE t RENAME COLUMN old_col_name TO new_col_name;
ALTER TABLE t MODIFY col_name new_type;
ALTER TABLE t ADD new_col DOUBLE;

-- delete all rows of a table, keep structure
TRUNCATE TABLE t;

-- delete a row, column, table, database
DELETE FROM t WHERE id = 4;

SHOW COLUMNS FROM t; /* or, also for MySQL */ DESCRIBE t;
SHOW COLUMNS FROM t WHERE Type LIKE 'Varchar%';
ALTER TABLE t DROP COLUMN name;

SHOW TABLES;
DROP TABLE IF EXISTS 't';

SHOW DATABASES;
DROP DATABASE mydb;
\end{lstlisting}

\begin{lstlisting}[language=bash]
# export selection to csv (from shell, no file perms)
mysql -u user --password=pw --database=mydb
 --execute='SELECT ...;' -q -n -B -r > o.csv
 && sed -i '/\t/ s//,/g' o.csv

# load table from csv (from shell)
mysqlimport --ignore-lines=1 --fields-terminated-by=,
 --verbose --local -u user -p mydb /path/to/in.csv
\end{lstlisting}
