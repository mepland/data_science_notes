%%%%%%%%%%%%%%%%%%%%%%%%%%%%%%%%%%%%%%%%%%%%%%%%%%%%%%%%
%%%%%%%%%%%%%%%%%%%%%%%%%%%%%%%%%%%%%%%%%%%%%%%%%%%%%%%%
%%%%%%%%%%%%%%%%%%%%%%%%%%%%%%%%%%%%%%%%%%%%%%%%%%%%%%%%
\chapter{\pandas}
\label{pandas}

%%%%%%%%%%%%%%%%%%%%%%%%%%%%%%%%%%%%%%%%%%%%%%%%%%%%%%%%
%%%%%%%%%%%%%%%%%%%%%%%%%%%%%%%%%%%%%%%%%%%%%%%%%%%%%%%%
\section{Basic Commands}
\label{pandas:basic}

\begin{lstlisting}[language=Python]
# IO
dfp = pd.read_csv('file.csv', header=None)
dfp['col'] = dfp['col'].astype(int)
dfp.to_csv('out.csv')

# descriptive commands
dfp.describe(); dfp.columns; dfp.shape;

# aggregation commands
dfp.sum(); dfp.cumsum();
dfp.min(); dfp.max(); dfp.idxmin(); dfp.idxmax();
dfp.mean(); dfp.std(); dfp.median(); dfp.mode();

# extract all rows from one column as df, not series
dfp_y = dfp.loc[:, ['y']]

# select by value
dfp_selection = dfp.loc[( ((dfp['x'] == x_value) & (dfp['y'] == y_value)) | (dfp['z'] < z_value))]

# select by value and assign new value
dfp.loc[(dfp['x'] == x_value), 'y'] = y_value

# select with query
dfp_selection = dfp.query('0 < x')

# select with isin
dfp_selection = dfp.isin({'x': list2, 'y': list2})

# select row by index
series = dfp.iloc[0]

# apply an arbitrary function
def func(x, y):
	return x*np.sin(y)
dfp['z'] = np.vectorize(func)(dfp['x'], dfp['y'])

# iterate through rows - slow!
for index, row in dfp.iterrows():
	x_value = row['x']

# construct df from rows
rows_list = []
for nrow in range(nrows):
	rows_list.append({'x':x_value, 'y':y_value})
dfp = pd.DataFrame(rows_list)
dfp = dfp[['x', 'y']]

# rename columns
dfp = dfp.rename({'old': 'new'}, axis='columns')

# sort
dfp = dfp.sort_values(by=['x', 'y'], ascending=[True, False]).reset_index(drop=True)

# group by, while dropping new count column and duplicates
dfp = dfp.groupby(['x', 'y', 'z']).size().to_frame(name = 'count').reset_index().drop(['count'], axis=1).drop_duplicates()

# group by, aggregating multiple columns simultaneously
dfp = dfp.groupby(['x']).agg({'y': 'mean', 'z': 'max'}).reset_index()
dfp = dfp.groupby(['x']).agg({'y': 'mean', 'z': ['max', 'min']}) # needs multi-index

# return duplicate rows
columns_to_check_for_duplicates = ['x', 'y']
dfp_duplicates = dfp[dfp.duplicated(subset=columns_to_check_for_duplicates, keep=False)]

# shuffle rows
dfp = dfp.sample(frac=1., replace=False, random_state=rnd_seed).reset_index(drop=True)

# drop columns
dfp = dfp.drop(['col_to_drop1', 'col_to_drop2'], axis=1)

# union dataframes
dfp = pd.concat([dfp_1, dfp_2], ignore_index=True)

# fill nans, for all columns and per column
dfp = dfp.fillna(0.0)
dfp = dfp.fillna(value={'x': x_nan_val, 'z': y_nan_val})
\end{lstlisting}

\clearpage

%%%%%%%%%%%%%%%%%%%%%%%%%%%%%%%%%%%%%%%%%%%%%%%%%%%%%%%%
%%%%%%%%%%%%%%%%%%%%%%%%%%%%%%%%%%%%%%%%%%%%%%%%%%%%%%%%
\section{Joining}
\label{pandas:join}

\noindent See the \href{https://pandas.pydata.org/pandas-docs/stable/user_guide/merging.html#database-style-dataframe-or-named-series-joining-merging}{documentation}
and this
\href{https://chrisalbon.com/code/python/data_wrangling/pandas_join_merge_dataframe/}{useful guide}.

\begin{lstlisting}[language=Python]
dfp = pd.merge(dfp_l, dfp_r, left_on='id_left', right_on='id_right', how='left')
\end{lstlisting}

%%%%%%%%%%%%%%%%%%%%%%%%%%%%%%%%%%%%%%%%%%%%%%%%%%%%%%%%
%%%%%%%%%%%%%%%%%%%%%%%%%%%%%%%%%%%%%%%%%%%%%%%%%%%%%%%%
\section{Pivoting}
\label{pandas:pivoting}

\subsubsection{pivot}
\label{pandas:pivoting:pivot}

\noindent \href{http://pandas.pydata.org/pandas-docs/stable/reference/api/pandas.DataFrame.pivot.html}{\texttt{pivot} documentation}.

\begin{lstlisting}[language=Python]
dfp.pivot(index=None, columns=None, values=None)
\end{lstlisting}

\begin{figure}[H]
\centering
\includegraphics[width=0.85\textwidth]{figures/pandas/reshaping_pivot.png}
\caption{
Example \pandas \texttt{pivot} operation, from the package \href{http://pandas.pydata.org/pandas-docs/stable/user_guide/reshaping.html}{documentation}.
}
\label{fig:pandas:pivot}
\end{figure}

\subsubsection{pivot\_table}
\label{pandas:pivoting:pivot_table}

\noindent \href{https://pandas.pydata.org/pandas-docs/stable/reference/api/pandas.pivot_table.html}{\texttt{pivot\_table} documentation}.

\begin{lstlisting}[language=Python]
pd.pivot_table(dfp, values=None, index=None, columns=None, aggfunc='mean', fill_value=None, margins=False, dropna=True, margins_name='All')
\end{lstlisting}

\begin{figure}[H]
\centering
\includegraphics[width=0.95\textwidth]{figures/pandas/pivot-table-datasheet.png}
\caption{
Example \pandas \texttt{pivot\_table} operation, by \href{http://pbpython.com/pandas-pivot-table-explained.html}{Chris Moffitt}.
}
\label{fig:pandas:pivot_table}
\end{figure}
