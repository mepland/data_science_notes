%%%%%%%%%%%%%%%%%%%%%%%%%%%%%%%%%%%%%%%%%%%%%%%%%%%%%%%%
%%%%%%%%%%%%%%%%%%%%%%%%%%%%%%%%%%%%%%%%%%%%%%%%%%%%%%%%
\section{Survival Analysis}
\label{additional:Survival}

Survival analysis is a subfield of statistics which examines problems involving the timing of events
such as death, component failure, or exiting a line of therapy after an adverse effect, in a population of subjects under study.
Using the appropriate models, quantities such as the lifetime, or failure rate, of a subject
can be estimated, as well as their dependence on other independent variables.
A variety of common models are available,
each with their own assumptions, capabilities, and limitations,
which make them best suited for certain applications.

%%%%%%%%%%%%%%%%%%%%%%%%%%%%%%%%%%%%%%%%%%%%%%%%%%%%%%%%
\subsection{Nomenclature}
\label{additional:Survival:Nomenclature}

\begin{symbollist}
	\item[Event] The event of interest in a study, \eg death, component failure\ldots
	\item[$t$] Time from the start of observation to an event, the conclusion of the study, or the withdrawal of a subject.
	\item[$T$] Time that an event occurred
	\item[Censoring] Right\footnote{Left censored subjects enter the study after the event of interest has already occurred at an unknown $t < 0$.} censored subjects have no events during the observation period, either due to early withdrawal or the conclusion of the study. The true lifetime of these subjects is unavailable for analysis, \ie they have been censored.
	\item[$S\left(t\right)$] The survival function $S\left(t\right)$ is the probability that a subject survives longer than $t$, \ie $S\left(t\right) = P\left(T > t\right)$.
	\item[$F\left(t\right)$] The lifetime distribution function $F\left(t\right)$ is the complement of the survival function, \ie $F\left(t\right) = 1 - S\left(t\right) = P\left(T \leq t\right)$.
	\item[$f\left(t\right)$] The event density function $f\left(t\right)$ is the time derivative of the lifetime distribution function $F\left(t\right)$, $f\left(t\right) = \frac{dF}{dt} = -\frac{dS}{dt}$, if it exists.
	\item[$\lambda\left(t\right)$] The hazard function $\lambda\left(t\right)$ is the event rate at $t$ conditional on survival to time $t$, \ie $T \geq t$, see \cref{eq:Survival:hazard:def}. Any $\lambda\left(t\right)$ can be a hazard function, provided it follows \cref{eq:Survival:hazard:cond}\footnote{In nuclear physics $\lambda\left(t\right) = \lambda$ is the constant exponential decay rate, decays per unit time. $\frac{dN}{dt} = -\lambda N$, $N\left(t\right) = N_{0} e^{-\lambda t}$, half-life $t_{1/2} = \ln\left(2\right) / \lambda = \tau \ln\left(2\right)$, where $\tau$ is the time constant.}.
	\item[$\Lambda\left(t\right)$] The cumulative hazard function $\Lambda\left(t\right)$ is the integral of $\lambda\left(t\right)$ with respect to time \cref{eq:Survival:cum_hazard:def}. Also see the relations in \cref{eq:Survival:cum_hazard:to_lambda,eq:Survival:cum_hazard:to_S}.
	\item[HR] The hazard ratio (HR) compares the hazard between time $t$ and a reference time $t_{0}$, $\text{HR} = \lambda\left(t\right) / \lambda\left(t_{0}\right)$. % TODO different from relative risk, also talk about log hazard
\end{symbollist}

\begin{subequations}\label{eq:Survival:hazard}
\begin{gather}
\lambda\left(t\right)
= \lim_{dt \to 0} \frac{P\left(t \leq T < t + dt\right)}{dt \, S\left(t\right)}
= \frac{f\left(t\right)}{S\left(t\right)} = -\frac{1}{S} \frac{dS}{dt} \label{eq:Survival:hazard:def} \\
\forall t \geq 0, \, \lambda\left(t\right) \geq 0 \quad \text{and} \quad \int_{0}^{\infty} \lambda\left(t\right) \, \dif t = \infty \label{eq:Survival:hazard:cond}
\end{gather}
\end{subequations}

\begin{subequations}\label{eq:Survival:cum_hazard}
\begin{gather}
\Lambda\left(t\right) = \int_{0}^{t} \lambda\left(u\right) \, \dif u = - \ln\left(S\left(t\right)\right) \label{eq:Survival:cum_hazard:def} \\
\lambda\left(t\right) = \frac{d\Lambda}{dt} = -\frac{1}{S} \frac{dS}{dt} \label{eq:Survival:cum_hazard:to_lambda} \\
S\left(t\right) = \exp\left(-\Lambda\left(t\right)\right) \label{eq:Survival:cum_hazard:to_S}
\end{gather}
\end{subequations}

\subsubsection{Expected Future Lifetime}
\label{additional:Survival:Nomenclature:efl}
The expected future lifetime $\expval{T-t_{0}}$ is the
expected additional time of survival for a subject having already survived to $t_{0}$.
We can derive $\expval{T-t_{0}}$ \cref{eq:Survival:efl_der:efl}
from the probability of an event occurring between $t_{0}$ and $t_{0} + t$ \cref{eq:Survival:efl_der:P}
as follows, using integration by parts in the final step:

\begin{subequations}\label{eq:Survival:efl_der}
\begin{gather}
P\left(T \leq t_{0} + t \mid T > t_{0}\right)
= \frac{P\left(t_{0} < T \leq t_{0} + t\right)}{P\left(T > t_{0}\right)}
= \frac{F\left(t_{0} + t\right) - F\left(t_{0}\right)}{S\left(t_{0}\right)} \label{eq:Survival:efl_der:P} \\
\text{PDF}
= \frac{d}{dt} P\left(T \leq t_{0} + t \mid T > t_{0}\right)
= \frac{f\left(t_{0} + t\right)}{S\left(t_{0}\right)} \label{eq:Survival:efl_der:pdf} \\
\expval{T-t_{0}}
= \int_{0}^{\infty} t \, \text{PDF} \, \dif t
= \frac{1}{S\left(t_{0}\right)} \int_{0}^{\infty} S\left(t\right) \, \dif t \label{eq:Survival:efl_der:efl}
\end{gather}
\end{subequations}

%%%%%%%%%%%%%%%%%%%%%%%%%%%%%%%%%%%%%%%%%%%%%%%%%%%%%%%%
\subsection{Kaplan-Meier Model}
\label{additional:Survival:KM}

The Kaplan-Meier model $\hat{S}_{\text{KM}}\left(t\right)$ is a non-parametric estimate of $S\left(t\right)$ computed from empirical data.

\begin{equation}\label{eq:Survival:KM}
\hat{S}_{\text{KM}}\left(t\right) = \prod_{i:\,t_{i} < t} \left(1 - \frac{d_{i}}{n_{i}}\right)
\end{equation}

\noindent Here the product is over all times $t_{i}$ at which $d_{i}$ events occurred,
and $n_{i}$ is the number of subjects still under study at $t_{i}$,
\ie subjects who have not had an event or been censured.
Due to its construction, $\hat{S}_{\text{KM}}\left(t\right)$ remains steady
between $t_{i}$ but drops vertically at each data point.
Therefore the derivative does not exist and we can not estimate $f\left(t\right)$ or $\lambda\left(t\right)$.


We can apply the Kaplan-Meier estimator to different subsets of subjects partitioned by a categorical variable $x$
in order to understand the dependence of $S\left(t\right)$ on $x$.
Note that $x$ can not be a continuous variable, and must be binned to integer classes if that is the case.

A simple example in \R utilizing the built in Acute myeloid leukemia (aml) dataset is provided below.
The status variable represents the event, recurrence of leukemia, as $\text{status}=1$, and the censoring of a patient as $\text{status}=0$.
Two different treatment paths are present in the data, $x=$ Maintained and Nonmaintained.

\begin{figure}[H]
\centering
\includegraphics[width=0.7\textwidth]{figures/survival/aml_km.pdf}
\vspace{0.2cm}
\caption{
Kaplan-Meier plot of the aml dataset. Censoring events are displayed with tick marks.
}
\label{fig:aml_km}
\end{figure}

\begin{lstlisting}[language=R]
> library(survival)
> aml[with(aml, order(time)), ]
   time status             x
12    5      1 Nonmaintained
13    5      1 Nonmaintained
14    8      1 Nonmaintained
15    8      1 Nonmaintained
1     9      1    Maintained
16   12      1 Nonmaintained
2    13      1    Maintained
3    13      0    Maintained
17   16      0 Nonmaintained
4    18      1    Maintained
5    23      1    Maintained
18   23      1 Nonmaintained
19   27      1 Nonmaintained
6    28      0    Maintained
20   30      1 Nonmaintained
7    31      1    Maintained
21   33      1 Nonmaintained
8    34      1    Maintained
22   43      1 Nonmaintained
9    45      0    Maintained
23   45      1 Nonmaintained
10   48      1    Maintained
11  161      0    Maintained
> km.model <- survfit(Surv(time, status) ~ x, data=aml, type="kaplan-meier")
> summary(km.model)
Call: survfit(formula = Surv(time, status) ~ x, data = aml, type = "kaplan-meier")

                x=Maintained
 time n.risk n.event survival std.err lower 95% CI upper 95% CI
    9     11       1    0.909  0.0867       0.7541        1.000
   13     10       1    0.818  0.1163       0.6192        1.000
   18      8       1    0.716  0.1397       0.4884        1.000
   23      7       1    0.614  0.1526       0.3769        0.999
   31      5       1    0.491  0.1642       0.2549        0.946
   34      4       1    0.368  0.1627       0.1549        0.875
   48      2       1    0.184  0.1535       0.0359        0.944

                x=Nonmaintained
 time n.risk n.event survival std.err lower 95% CI upper 95% CI
    5     12       2   0.8333  0.1076       0.6470        1.000
    8     10       2   0.6667  0.1361       0.4468        0.995
   12      8       1   0.5833  0.1423       0.3616        0.941
   23      6       1   0.4861  0.1481       0.2675        0.883
   27      5       1   0.3889  0.1470       0.1854        0.816
   30      4       1   0.2917  0.1387       0.1148        0.741
   33      3       1   0.1944  0.1219       0.0569        0.664
   43      2       1   0.0972  0.0919       0.0153        0.620
   45      1       1   0.0000     NaN           NA           NA

> pdf('~/aml_km.pdf')
> plot(km.model, conf.int=F, xlab='Time', ylab='S(t)', col=c('black', 'blue'), las=1, mark.time=T, lty=1:2, las=1)
> legend(110, 1, c('Maintained', 'Nonmaintained'), col=c('black', 'blue'), lty=1:2, bty='n')
> dev.off()
\end{lstlisting}

%%%%%%%%%%%%%%%%%%%%%%%%%%%%%%%%%%%%%%%%%%%%%%%%%%%%%%%%
\subsection{Exponential Model}
\label{additional:Survival:exp}
% TODO

\subsubsection{Weibull Model}
\label{additional:Survival:weibull}
% TODO

%%%%%%%%%%%%%%%%%%%%%%%%%%%%%%%%%%%%%%%%%%%%%%%%%%%%%%%%
\subsection{Cox Proportional-Hazards Model}
\label{additional:Survival:cox}
% TODO

%%%%%%%%%%%%%%%%%%%%%%%%%%%%%%%%%%%%%%%%%%%%%%%%%%%%%%%%
\subsection{Method Comparison}
\label{additional:Survival:comp}
% TODO

\begin{table}[H]
\centering
\begin{tabular}{l|l|l}
\multicolumn{1}{c}{Kaplan-Meier} & \multicolumn{1}{c}{Exponential} & \multicolumn{1}{c}{Cox} \\
\cline{1-3}
\begin{tabular}[c]{p{0.3\textwidth}}
Pro
\begin{itemize}
	\item Simple to compute and understand
	\item Estimates $S$
\end{itemize}
\\
Con
\begin{itemize}
	\item No functional form
	\item Can not estimate the hazard ratio
	\item Only can handle a few categorical $x$
\end{itemize}
\end{tabular}

&

\begin{tabular}[c]{p{0.3\textwidth}}
Pro
\begin{itemize}
	\item Can estimate $S$ and hazard ratio
\end{itemize}
\\
Con
\begin{itemize}
	\item Limited $\lambda\left(t\right) = c \, \text{or} \propto t$
\end{itemize}
\end{tabular}

&

\begin{tabular}[c]{p{0.3\textwidth}}
Pro
\begin{itemize}
	\item Item 1
	\item Item 2
\end{itemize}
\\
Con
\begin{itemize}
	\item Item 1
	\item Item 2
\end{itemize}
\end{tabular}

\\
\end{tabular}
\end{table}



%%%%%%%%%%%%%%%%%%%%%%%%%%%%%%%%%%%%%%%%%%%%%%%%%%%%%%%%
\subsection{}
\label{additional:Survival:}
% TODO

% TODO survdiff(Surv(time, status) ~ x, data=aml)
% TODO common assumptions, censuring is non-informative, \ie uncorrelated with event outcome
