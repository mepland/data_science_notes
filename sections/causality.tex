%%%%%%%%%%%%%%%%%%%%%%%%%%%%%%%%%%%%%%%%%%%%%%%%%%%%%%%%
%%%%%%%%%%%%%%%%%%%%%%%%%%%%%%%%%%%%%%%%%%%%%%%%%%%%%%%%
%%%%%%%%%%%%%%%%%%%%%%%%%%%%%%%%%%%%%%%%%%%%%%%%%%%%%%%%
\chapter{Causal Inference}
\label{chap:causal_inference}

Causal inference is the application of prior knowledge, logic, and statistics
to estimate the causal effect of some intervention, action, treatment, variable, \etc, on a larger system;
moving beyond simple statements of correlation\footnote{``Everyone who confuses correlation with causation eventually ends up dead'' -- unknown}.
Causal effect is defined to be the difference between
the factual state of the world, where some intervention $D$ actually occurred,
and a counterfactual state of the world, where $D$ did not occur.
Using the notation of \cite{CausalMixtape}, which forms the basis of this chapter unless otherwise noted,
we can write out the causal effect $\delta$, \ie the unit-specific treatment effect, as

\begin{equation}\label{eq:causal_effect_delta}
\delta_{i} = Y_{i}^{1} - Y_{i}^{0}\,,
\end{equation}

\noindent where $i$ represents a particular individual,
and $Y^{D}$ is the observed outcome\footnote{We can represent
$Y_{i}$ under either $D_{i}=0,1$ with the switching equation,
$Y_{i} = D_{i} Y_{i}^{1} + \left(1-D_{i}\right) Y_{i}^{0}$.},
both with $D=1$, and without $D=0$, the intervention.
This leads us to the fundamental problem of causal inference:
there will always be part of the data which can not be observed,
\ie one path of $D_{i}$ will always be counterfactual\footnote{For example,
in medicine a patient can be treated, or not, but not both.}.
As we can not observe real data for both outcomes,
it is necessary to make assumptions for any and all measurements of causal effects.

%%%%%%%%%%%%%%%%%%%%%%%%%%%%%%%%%%%%%%%%%%%%%%%%%%%%%%%%
%%%%%%%%%%%%%%%%%%%%%%%%%%%%%%%%%%%%%%%%%%%%%%%%%%%%%%%%
\section{Measures of Causal Effect}
\label{causal_inference:Measures}
TODO

average treatment effect (ATE),
average treatment effect on the treated (ATT),
and the
average treatment effect on the untreated (ATU)

\begin{subequations} \label{eq:causal_measures}
\begin{align}
ATE &= \expvalE{\delta_{i}} = \expvalE{Y_{i}^{1} - Y_{i}^{0}} = \expvalE{Y_{i}^{1}} - \expvalE{Y_{i}^{0}} \label{eq:causal_measures:ATE} \\
ATT &= \expvalE{\delta_{i} \mid D_{i} = 1} = \expvalE{Y_{i}^{1} \mid D_{i} = 1} - \expvalE{Y_{i}^{0} \mid D_{i} = 1} \label{eq:causal_measures:ATT} \\
ATU &= \expvalE{\delta_{i} \mid D_{i} = 0} = \expvalE{Y_{i}^{1} \mid D_{i} = 0} - \expvalE{Y_{i}^{0} \mid D_{i} = 0} \label{eq:causal_measures:ATU}
\end{align}
\end{subequations}

%%%%%%%%%%%%%%%%%%%%%%%%%%%%%%%%%%%%%%%%%%%%%%%%%%%%%%%%
%%%%%%%%%%%%%%%%%%%%%%%%%%%%%%%%%%%%%%%%%%%%%%%%%%%%%%%%
\section{Directed Acyclic Graphs (DAG)}
\label{causal_inference:DAG}
TODO

% TODo flesh this out more
Unless we are in the trivial case with only a single input feature,
we must make at least one assumption\footnote{This
dovetails nicely with the introductory statement that
``it is necessary to make assumptions for any and all measurements of causal effects''.} about
the structure of the DAG to keep it acyclic.

%%%%%%%%%%%%%%%%%%%%%%%%%%%%%%%%%%%%%%%%%%%%%%%%%%%%%%%%
%%%%%%%%%%%%%%%%%%%%%%%%%%%%%%%%%%%%%%%%%%%%%%%%%%%%%%%%
\section{Difference in Differences (DID)}
\label{causal_inference:DID}

\begin{subequations} \label{eq:DID:2x2}
\begin{align}
\hat{\delta}_{kU}^{2 \times 2} &= \left( \expvalE{Y_{k} \mid \text{Post}} - \expvalE{Y_{k} \mid \text{Pre}} \right)
- \left( \expvalE{Y_{U} \mid \text{Post}} - \expvalE{Y_{U} \mid \text{Pre}} \right) \label{eq:DID:2x2:def} \\
&= \left( \expvalE{Y_{k}^{1} \mid \text{Post}} - \expvalE{Y_{k}^{0} \mid \text{Pre}} \right)
- \left( \expvalE{Y_{U}^{0} \mid \text{Post}} - \expvalE{Y_{U}^{0} \mid \text{Pre}} \right) \label{eq:DID:2x2:expand} \\
&\hphantom{=} + \left( \expvalE{Y_{k}^{0} \mid \text{Post}} - \expvalE{Y_{k}^{0} \mid \text{Pre}} \right) \label{eq:DID:2x2:add_zero} \\
&= \underbrace{\left( \expvalE{Y_{k}^{1} \mid \text{Post}} - \expvalE{Y_{k}^{0} \mid \text{Post}} \right)}_{\text{ATT}} \label{eq:DID:2x2:ATT} \\
&\hphantom{=} + \underbrace{
\left( \expvalE{Y_{k}^{0} \mid \text{Post}} - \expvalE{Y_{k}^{0} \mid \text{Pre}} \right)
- \left( \expvalE{Y_{U}^{0} \mid \text{Post}} - \expvalE{Y_{U}^{0} \mid \text{Pre}} \right)
}_{\text{Non-parallel trends bias in $2\times 2$ case}} \label{eq:DID:2x2:nonparallel_trends_bias}
\end{align}
\end{subequations}

TODO

%\begin{figure}
%\centering
%\includegraphics[width=0.6\textwidth]{figures/TODO}
%\caption{
%}
%\label{fig:TODO}
%\end{figure}

%\begin{subequations}\label{eq:causal_inference:TODO}
%\begin{align}
%y &= y_{t}, \label{eq:causal_inference:TODO:A} \\
%\delta &= 1. \label{eq:causal_inference:TODO:B}
%\end{align}
%\end{subequations}
