%-----------------------------------------------------------------------------%
% Set metadata for use with pdfx
%\begingroup\newif\ifmy
%\IfFileExists{\jobname.xmpdata}{}{\mytrue}
%\ifmy
%\begin{filecontents*}{\jobname.xmpdata}
%\Title{Data Science Notes}
%\Author{Matthew Epland, Ph.D.}
%\Keywords{Data Science\sep Machine Learning\sep Statistics}
%\Copyright{CC-BY-4.0}
%\end{filecontents*}
%\fi\endgroup

% \Subject{TODO}

%-----------------------------------------------------------------------------%
% Set documentclass
\documentclass[nogradschool,singlespace,nobind]{dukedissertation_modified}

%-----------------------------------------------------------------------------%
% usepackages
\usepackage{amsmath,amssymb,bbm,bm} % https://ctan.org/pkg/bm
\usepackage{mathtools}
\usepackage{cancel}
\usepackage{graphicx}
\usepackage{xcolor}
\usepackage{textcomp} % needed to fix \mico \textmu from siunitx to work with microtype: https://tex.stackexchange.com/questions/74670/microtype-siunitx-and-micro-mysterious-warnings
\usepackage[protrusion=true,expansion=true]{microtype} % make text flow nicely... might screw up duke dissertation template
\usepackage{verbatim} % verbatim text and comment environment
\usepackage{lmodern} % allowing font sizes at arbitrary sizes
\usepackage{notoccite} % fixes citation numbering in captions with respect to lof & lot, see https://ctan.org/pkg/notoccite
\usepackage[nocompress]{cite} % orders references numerically within one \cite{}, see https://tex.stackexchange.com/questions/69230/numbered-ordering-of-multiple-citations Also changes spacing after comma
\usepackage{fnpct} % make multiple footnotes at one point look nice, https://tex.stackexchange.com/questions/28465/multiple-footnotes-at-one-point
\usepackage[separate-uncertainty,multi-part-units=single,free-standing-units,product-units=repeat,use-xspace,binary-units]{siunitx} % units package, see https://www.ctan.org/pkg/siunitx
\sisetup{range-phrase={\text{--}},range-units=single}
\usepackage{physics}
\usepackage{booktabs,array,multirow,diagbox}
\renewcommand{\arraystretch}{1.5} % gives extra height to tabular rows for super/subscripts
%\usepackage{longtable}
\usepackage{enumitem}
\usepackage{lscape} % landscape https://ctan.org/pkg/lscape
\usepackage{moresize}
\usepackage{listings}
\usepackage[T1]{fontenc}
\usepackage{fontawesome}

\usepackage[top=1in, bottom=1.25in, left=1.25in, right=1.25in]{geometry}
\usepackage{fancyhdr}
\pagestyle{plain}

% use subcaption to get split figures, but the caption dependency doesn't know about the dukedissertation document class - turn off the warning with silence
% load caption explicitly first to set it's options, subcaption says it passes them through but it doesn't seem to work
% https://tex.stackexchange.com/questions/34579/is-there-really-something-wrong-with-using-the-caption-package-for-continuedflo
% https://en.wikibooks.org/wiki/LaTeX/Floats,_Figures_and_Captions#Subfloats
% https://ctan.org/pkg/caption
% https://ctan.org/pkg/subcaption
\usepackage{silence}
\WarningFilter{caption}{Unsupported document class}
\WarningFilter{hyperref}{The PDF version number could not be set}
\usepackage{setspace} % needed to specify, https://ctan.org/pkg/setspace
\usepackage[style=base,skip=2pt,width=\textwidth]{caption} % ,font={stretch=1.3}
\usepackage[skip=1pt]{subcaption}
\newsavebox{\largestimage} % see https://tex.stackexchange.com/questions/239128/subcaption-vertical-alignment-of-two-images-of-different-vertical-size

% Don't let floats get before the subsection where they're included
% https://tex.stackexchange.com/questions/32598/force-latex-image-to-appear-in-the-section-in-which-its-declared
% https://tex.stackexchange.com/questions/279/how-do-i-ensure-that-figures-appear-in-the-section-theyre-associated-with/235312#235312
% Also doesn't let a float go into a following subsection, results in a ton of blank space - probably better left off
% \usepackage{placeins}
%\let\Oldsubsection\subsection
%\renewcommand{\subsection}{\FloatBarrier\Oldsubsection}

\usepackage{float} % to allow for H option. Works better than \FloatBarrier from placeins, though it is more manual
\floatstyle{plaintop}
\restylefloat{table}

% keep footnotes from splitting, can still happen sometimes (10000 forces)
% https://tex.stackexchange.com/questions/32208/footnote-runs-onto-second-page
\interfootnotelinepenalty=9999

% keep inline equations from splitting, can still happen sometimes (10000 forces)
% https://tex.stackexchange.com/a/14243
\relpenalty=9999
\binoppenalty=9999

% center subfigure captions with multiple lines
\captionsetup[subfigure]{justification=centering}

%-----------------------------------------------------------------------------%
% Other possibly useful packages
% \usepackage{fancyvrb}
% \usepackage{ulem}
% \usepackage{overpic}
% \usepackage{amsfonts, amsthm}
% \usepackage{mathabx}

%-----------------------------------------------------------------------------%
% tweak listings
\definecolor{codegreen}{rgb}{0,0.6,0}
\definecolor{codegray}{rgb}{0.5,0.5,0.5}
\definecolor{codemauve}{rgb}{0.58,0,0.82}

\lstset{
  backgroundcolor=\color{white},     % choose the background color; you must add \usepackage{color} or \usepackage{xcolor}; should come as last argument
  % basicstyle=\ssmall,                % the size of the fonts that are used for the code
  basicstyle=\ttfamily,              % make font tt
  upquote=true,                      % make all single quotes straight up and down
  breakatwhitespace=false,           % sets if automatic breaks should only happen at whitespace
  breaklines=true,                   % sets automatic line breaking
  captionpos=b,                      % sets the caption-position to bottom
  commentstyle=\color{codegreen},    % comment style
  % deletekeywords={...},            % if you want to delete keywords from the given language
  % escapeinside={\%*}{*)},          % if you want to add LaTeX within your code
  extendedchars=true,                % lets you use non-ASCII characters; for 8-bits encodings only, does not work with UTF-8
  firstnumber=1,                     % start line enumeration with line 1000
  frame=none,                        % adds a frame around the code
  keepspaces=true,                   % keeps spaces in text, useful for keeping indentation of code (possibly needs columns=flexible)
  keywordstyle=\color{blue},         % keyword style
  % language=Octave,                 % the language of the code
  % morekeywords={*,...},            % if you want to add more keywords to the set
  numbers=left,                      % where to put the line-numbers; possible values are (none, left, right)
  numbersep=5pt,                     % how far the line-numbers are from the code
  numberstyle=\ttfamily\tiny\color{codegray}, % the style that is used for the line-numbers
  rulecolor=\color{black},           % if not set, the frame-color may be changed on line-breaks within not-black text (e.g. comments (green here))
  showspaces=false,                  % show spaces everywhere adding particular underscores; it overrides 'showstringspaces'
  showstringspaces=false,            % underline spaces within strings only
  showtabs=false,                    % show tabs within strings adding particular underscores
  stepnumber=1,                      % the step between two line-numbers. If it's 1, each line will be numbered
  stringstyle=\color{codemauve},     % string literal style
  tabsize=2,                         % sets default tabsize to 2 spaces
}

%-----------------------------------------------------------------------------%
% Include abbreviations
\usepackage{xspace}
 % note that \xspace properly handles the abbreviation period spacing - producing a single regular space, not the end of a sentence spacing

% https://tex.stackexchange.com/questions/7032/good-way-to-make-textcircled-numbers
% \usepackage{tikz}
% \DeclareRobustCommand{\circled}[1]{\tikz[baseline=(char.base)]{\node[shape=circle,draw,inner sep=1pt] (char) {#1};}}

% general, https://tex.stackexchange.com/questions/2229/is-a-period-after-an-abbreviation-the-same-as-an-end-of-sentence-period/2230#2230
\newcommand*{\ie}{\textit{i.e.}\@\xspace}
\newcommand*{\eg}{\textit{e.g.}\@\xspace}
%\newcommand*{\etc}{\textit{etc.}\@\xspace}
\makeatletter
\newcommand\etc{\textit{etc}\@ifnextchar.{}{.\@\xspace}}
\makeatother

% \newcommand{\orderof}{\ensuremath{\mathcal{O}}} % use \order from physics package instead
\newcommand{\lagr}{\ensuremath{\mathcal{L}}\xspace}
\newcommand{\transpose}{\ensuremath{^{\text{T}}}\xspace}
\newcommand{\stcomp}[1]{\overline{#1}}
\newcommand{\identity}{\ensuremath{I}\xspace} % note \mathbb{1} did not work
% \newcommand{\dif}{\mathop{}\!\mathrm{d}} % I think dt looks just fine
% \newcommand{\dif}{d} % I think dt looks just fine
\newcommand*{\dif}{\ensuremath{d}} % from ATLAS def

\DeclareMathOperator*{\sign}{sgn}

\DeclareMathOperator*{\argmin}{arg\,min} % thin space, limits underneath in displays, https://tex.stackexchange.com/questions/5223/command-for-argmin-or-argmax
\DeclareMathOperator*{\argmax}{arg\,max}

\newcommand*{\cov}[2]{\ensuremath{\text{cov}\left(#1,#2\right)}\xspace}
\newcommand*{\variance}[1]{\ensuremath{\text{var}\left(#1\right)}\xspace}

\newcommand*{\pvalue}{\ensuremath{p\text{-value}}\xspace}

\newcommand*{\insitu}{\text{\textit{in~situ}}\xspace}
\newcommand*{\Insitu}{\text{\textit{In~situ}}\xspace}
\newcommand*{\InSitu}{\text{\textit{In~Situ}}\xspace}

\newcommand*{\apriori}{\text{\textit{a~priori}}\xspace}

\newcommand*{\CLs}{\ensuremath{CL_{s}}\xspace}

\newcommand*{\yhatBDT}{\ensuremath{\hat{y}_{\text{BDT}}}\xspace}
\newcommand*{\yhat}{\ensuremath{\hat{y}}\xspace}

% Software packages
\newcommand*{\xgboost}{\textsc{XGBoost}\xspace}
\newcommand*{\uprootPackage}{\textsc{uproot}\xspace} % \uproot already exists
\newcommand*{\sql}{\textsc{SQL}\xspace}
\newcommand*{\pandas}{\textsc{Pandas}\xspace}
\newcommand*{\numpy}{\textsc{NumPy}\xspace}
\newcommand*{\scipy}{\textsc{SciPy}\xspace}
\newcommand*{\sklearn}{\textsc{Scikit-Learn}\xspace}
\newcommand*{\skopt}{\textsc{Scikit-Optimize}\xspace}
\newcommand*{\networkx}{\textsc{NetworkX}\xspace}
\newcommand*{\ROOT}{\textsc{ROOT}\xspace}
\newcommand*{\TMVA}{\textsc{TMVA}\xspace}
\newcommand*{\HF}{\textsc{HistFitter}\xspace}
\newcommand*{\hfactory}{\textsc{HistFactory}\xspace}
\newcommand*{\roostats}{\textsc{RooStats}\xspace}
\newcommand*{\roofit}{\textsc{RooFit}\xspace}


%-----------------------------------------------------------------------------%
% Include tikz figures which can not be made standalone, only if they have internal references / citations. Also must be manually added to Makefile if they use feynmp

%-----------------------------------------------------------------------------%
% Theorem, Lemma, etc. environments
%\newtheorem{theorem}{Theorem}%[section]
%\newtheorem{lemma}[theorem]{Lemma}
%\newtheorem{proposition}[theorem]{Proposition}
%\newtheorem{corollary}[theorem]{Corollary}
%\newtheorem{result}[theorem]{Result}

%-----------------------------------------------------------------------------%
% PREAMBLE
%-----------------------------------------------------------------------------%
\author{Matthew Epland, Ph.D.}
\title{Data Science Notes}
\date{\today}
%-----------------------------------------------------------------------------%

%-----------------------------------------------------------------------------%
% HYPERREF
%-----------------------------------------------------------------------------%
\usepackage[hyperpageref]{backref} % pages

% need to load in this order to get proper pdfx a-2b format!!!
\PassOptionsToPackage{hyperfootnotes,pagebackref}{hyperref}

% comment out \usepackage{hyperref} and \hypersetup if using pdfx
\usepackage{hyperref}
\makeatletter\hypersetup{
    breaklinks, baseurl=http://, pdfborder=0 0 0, pdfpagemode=UseNone, pdfstartpage=1, bookmarksopen=false, bookmarksdepth=2, % to show sections and subsections
    pdfauthor      = {Matthew Epland, Ph.D.}, %
    pdftitle       = {Data Science Notes}, %
    pdfsubject     = {Data Science, Machine Learning, Statistics}, %
    pdfkeywords    = {Data Science, Machine Learning, Statistics}
}\makeatother

% \usepackage[a-2b]{pdfx} % Note pdfx does not work with travis CI due to latest ubuntu image being from 2016, thus containing this bug https://bugs.debian.org/cgi-bin/bugreport.cgi?bug=877167 fixed in oct 2017. Can use locally if desired

\hypersetup{plainpages=false, bookmarksnumbered,
            % draft, % for printing
            colorlinks, linkcolor=blue, citecolor=blue, urlcolor=blue, % for web
            % breaklinks=true,
           }

% adapted from https://tex.stackexchange.com/questions/183702/formatting-back-references-in-bibliography-bibtex
\renewcommand*{\backrefalt}[4]{%
%    \ifcase #1 Not cited.%
    \ifcase #1% Not cited.%
          \or Cited on page~#2.%
          \else Cited on pages #2.%
    \fi%
    }

%-----------------------------------------------------------------------------%
% use cref and not ref, have to load last
\usepackage[capitalise]{cleveref} % https://ctan.org/pkg/cleveref see section 7.1, if redefining need to make them caps
\crefname{figure}{Figure}{Figures}
\Crefname{figure}{Figure}{Figures}
\crefname{tabular}{Table}{Tables}
\Crefname{tabular}{Table}{Tables}
\crefname{section}{Section}{Sections}
\Crefname{section}{Section}{Sections}
\crefname{chapter}{Chapter}{Chapters}
\Crefname{chapter}{Chapter}{Chapters}
\crefname{appchap}{Appendix}{Appendices}
\Crefname{appchap}{Appendix}{Appendices}
\crefformat{equation}{(#2#1#3)}

\newcommand\preface{%
   \nmchapter{Preface}
}

\begin{document}

%-----------------------------------------------------------------------------%
% TITLE PAGE
%-----------------------------------------------------------------------------%
\maketitle

%-----------------------------------------------------------------------------%
% ABSTRACT -- included file should start with '\abstract'.
%-----------------------------------------------------------------------------%
% \include{{sections/abstract}}

%-----------------------------------------------------------------------------%
% FRONTMATTER
%-----------------------------------------------------------------------------%
\tableofcontents % Automatically generated
\backrefsetup{disable}
\abbreviations

\section*{Symbols}

\begin{symbollist}
	\item[$P\left(X \mid Y\right)$] (Conditional) Probability of $X$ Given $Y$
	\item[$\expval{X}$ or $\expvalE{X}$] Expectation Value of $X$
	\item[$\sigma_{X}^{2}$ or $\variance{X}$] Variance of $X$
	\item[$\sigma_{u,v}^{2}$ or $\cov{u}{v}$] Covariance of $u$ and $v$
	\item[$\bm{\beta}$] Model Parameters, $n \times 1$ Column Vector
	\item[$\mathbf{X}$] Input Features, $m \times n$ Matrix
	\item[$y$] Dependent Feature
	\item[$m$] Number of Data Points or Rows
	\item[$n$] Number of Input Features or Columns
	\item[$\nu$] Number of Degrees of Freedom
	\item[$S\left(\bm{\beta}\right)$] Objective Function
	\item[$L\left(\bm{\beta}\right)$] Loss Function
	\item[$\Omega\left(\bm{\beta}\right)$] Regularization Function
	\item[$L$] Likelihood Function
	\item[$\yhat$] Estimated Dependent Feature or Classification Score, Prediction
	\item[$Z$] Significance
	\item[$S\left(t\right)$] Survival Function
	\item[$\lambda\left(t\right)$] Hazard Function
	% \item[\ZB] Significance, Incomplete Beta Function Approximation
	% \item[\CLs] Signal Confidence Level
\end{symbollist}

%\clearpage
\section*{Abbreviations}
% TODo keep updated

\begin{symbollist}
	\item[$k$-NN] $k$-Nearest Neighbors
	\item[ACF] Auto-Correlation Function
	\item[ADF] Augmented Dickey--Fuller Test for Stationarity
	\item[ANCOVA] Analysis of Covariance
	\item[ANOVA] Analysis of Variance
	\item[AR] Autoregressive (Models)
	\item[AUC] Area Under Curve
	\item[BDT] Boosted Decision Tree
	\item[BLUE] Best Linear Unbiased Estimator
	\item[CART] Classification and Regression Tree
	\item[CDF] Cumulative Distribution Function
	\item[CLT] Central Limit Theorem
	\item[CNN] Convolutional Neural Network
	\item[FWHM] Full Width at Half Maximum
	\item[GLM] Generalized Linear Model
	\item[GLS] Generalized Least Squares
	\item[GMM] Gaussian Mixture Model
	\item[GP] Gaussian Process
	\item[HR] Hazard Ratio
	\item[i.i.d.] Independent and Identically Distributed
	\item[LSTM] Long Short Term Memory
	\item[LVQ] Learning Vector Quantization
	\item[MA] Moving Average (Models)
	\item[MANOVA] Multivariate Analysis of Variance
	\item[MAP] Maximum \aposteriori (Estimation)
	\item[MAPE] Mean Absolute Percent Error
	\item[ML] Machine Learning
	\item[MLE] Maximum Likelihood Estimation (or Estimator)
	\item[MSE] Mean Squared Error
	\item[NN] Neural Network
	\item[OLS] Ordinary Least Squares
	\item[PACF] Partial Auto-Correlation Function
	\item[PCA] Principle Component Analysis
	\item[PCR] Principal Component Regression
	\item[PDF] Probability Density Function
	\item[PR] Prevalence Ratio
	\item[RDBMS] Relational Database Management System
	\item[RMSD] Root Mean Squared Deviation
	\item[RMSE] Root Mean Squared Error
	\item[RNN] Recurrent Neural Network
	\item[ROC] Receiver Operating Characteristic
	\item[SGD] Stochastic Gradient Descent
	\item[SMBO] Sequential Model-Based Optimization
	\item[SQL] Structured Query Language
	\item[SVM] Support Vector Machine
	\item[TPE] Tree-Structured Parzen Estimator
	\item[VAE] Variational Autoencoder
%	\item[] 
\end{symbollist}
%	\item[GAN] Generational Adversarial Network
} % List of Abbreviations. Start file with '\abbreviations'
\backrefsetup{enable}

%-----------------------------------------------------------------------------%
% PREFACE
%-----------------------------------------------------------------------------%
\preface

These notes were prepared while studying data science interviews and are somewhat incomplete and scattered.
They are intended for use as a quick reference rather than an introduction to the material.
If you find an error, please let the author know at \href{mailto:matthew.epland@duke.edu}{matthew.epland@duke.edu}.
You may also be interested in Doug Davis' \url{https://github.com/drdavis/aou} lecture notes on the analysis of uncertainties,
and \textit{The Elements of Statistical Learning} \cite{HastieTF09}.
}

%==============================================================================
%-----------------------------------------------------------------------------%
%
% MAIN BODY
%
%
%-----------------------------------------------------------------------------%
%%%%%%%%%%%%%%%%%%%%%%%%%%%%%%%%%%%%%%%%%%%%%%%%%%%%%%%%
%%%%%%%%%%%%%%%%%%%%%%%%%%%%%%%%%%%%%%%%%%%%%%%%%%%%%%%%
%%%%%%%%%%%%%%%%%%%%%%%%%%%%%%%%%%%%%%%%%%%%%%%%%%%%%%%%
\chapter{Statistics}
\label{chap:stats}

%%%%%%%%%%%%%%%%%%%%%%%%%%%%%%%%%%%%%%%%%%%%%%%%%%%%%%%%
%%%%%%%%%%%%%%%%%%%%%%%%%%%%%%%%%%%%%%%%%%%%%%%%%%%%%%%%
\section{Expectation Value and Variance}
\label{stats:expval_and_var}

The expectation value \cref{eq:stats:exp_relations} and variance \cref{eq:stats:var_relations} are introductory, yet essential, statistical measures.
Their definitions and interesting properties are reproduced here for reference.
Note that $s$ is the unbiased sample variance,
and the sample mean $\bar{x}$ has $\mu_{\bar{x}} = \mu$, and standard error $\sigma_{\bar{x}} = \sigma / \sqrt{n}$,
where $\mu$ and $\sigma$ are from the parent population.

\begin{subequations}\label{eq:stats:exp_relations}
\begin{align}
\expvalE{X} = \expval{X} &= \sum_{j=1}^{m} x_{j} \, p_{j} = \int_{-\infty}^{\infty} x f\left(x\right) \, \dd{x} \label{eq:stats:exp_relations:def} \\
\bar{x} = \mu &= \frac{1}{n} \sum_{j=1}^{m} x_{j}\,,\,\text{for uniform}~p_{j} \label{eq:stats:exp_relations:mean} \\
\expval{X+Y} &= \expval{X} + \expval{X} \label{eq:stats:exp_relations:add} \\
\expval{a X} &= a \expval{X} \label{eq:stats:exp_relations:mult} \\
\expval{a} &= a \,\, \implies \, \expval{\expval{X}} = \expval{X} \label{eq:stats:exp_relations:self} \\
\expval{X Y}^{2} &\leq \expval{X^{2}} \expval{Y^{2}} \label{eq:stats:exp_relations:cbs_ineq}
\end{align}
\end{subequations}

\begin{subequations}\label{eq:stats:var_relations}
\begin{align}
\sigma_{X}^{2} = \variance{X} &= \expval{\left(x-\expval{x}\right)^{2}} = \expval{X^{2}} - \expval{X}^{2} \label{eq:stats:var_relations:def} \\
s^{2} &= \frac{1}{n-1} \sum_{j=1}^{m} \left( x_{j} - \expval{x}\right)^{2} \label{eq:stats:var_relations:sample} \\
s_{\bar{x}} &= \frac{s}{\sqrt{n}} \label{eq:stats:var_relations:standard_error_of_mean} \\
\variance{X+a} &= \variance{X} \label{eq:stats:var_relations:add} \\
\variance{a X} &= a^{2} \, \variance{X} \label{eq:stats:var_relations:mult} \\
\variance{a X \pm b Y} &= a^{2} \, \variance{X} + b^{2} \, \variance{Y} \pm 2 \, ab \, \cov{X}{Y} \label{eq:stats:var_relations:linear} \\
\variance{X \mid Y} &= \expval{\left(X - \expval{X \mid Y}\right)^{2} \mid Y} \label{eq:stats:var_relations:conditional1} \\
\variance{X} &= \expval{\variance{X \mid Y}} + \variance{\expval{X \mid Y}} \label{eq:stats:var_relations:conditional2}
\end{align}
\end{subequations}

%%%%%%%%%%%%%%%%%%%%%%%%%%%%%%%%%%%%%%%%%%%%%%%%%%%%%%%%
%%%%%%%%%%%%%%%%%%%%%%%%%%%%%%%%%%%%%%%%%%%%%%%%%%%%%%%%
\section{Covariance and Correlation}
\label{stats:corr_covar}

%%%%%%%%%%%%%%%%%%%%%%%%%%%%%%%%%%%%%%%%%%%%%%%%%%%%%%%%
\subsection{Covariance}
\label{stats:corr_covar:covariance}

The covariance between two variables $u$ and $v$,

\begin{equation}\label{eq:stats:covar}
\begin{split}
\sigma_{u,v}^{2} = \cov{u}{v} &= \frac{1}{m}\sum_{j=1}^{m}\left(u_{j}-\expval{u}\right)\left(v_{j}-\expval{v}\right) \\
&= \expval{\left(u-\expval{u}\right)\left(v-\expval{v}\right)} \\
&= \expval{u v} - \expval{u}\expval{v}\,,
\end{split}
\end{equation}

\noindent is a measure of their joint variability,
\ie a measure of any linear relationship which may exist between them.
It is helpful to remember the following covariance relations:

\begin{subequations}\label{eq:stats:covar_relations}
\begin{align}
\cov{X}{X} &= \variance{X}, \label{eq:stats:covar_relations:var} \\
\cov{X + a}{Y + b} &= \cov{X}{Y}, \label{eq:stats:covar_relations:add} \\
\cov{a\,X}{b\,Y} &= ab\,\cov{X}{Y}, \label{eq:stats:covar_relations:mult} \\
\cov{X}{Y}^{2} &\leq \variance{X}\variance{Y}. \label{eq:stats:covar_relations:inequality}
\end{align}
\end{subequations}

%%%%%%%%%%%%%%%%%%%%%%%%%%%%%%%%%%%%%%%%%%%%%%%%%%%%%%%%
\subsection{Pearson Correlation}
\label{stats:corr_covar:pearson}

The Pearson correlation coefficient,

\begin{equation}\label{eq:stats:corr:pearson}
\rho_{u,v} = \corr{u}{v} = \frac{\sigma_{u,v}^{2}}{\sigma_{u}\sigma_{v}} = \frac{\cov{u}{v}}{\sigma_{u}\sigma_{v}}\,,
\end{equation}

\noindent is a convenient dimensionless version, normalized to $-1 \leq \rho \leq 1$.
Example distributions can be found in \cref{fig:stats:corr_ex:pearson}.

\begin{figure}
\centering
\includegraphics[width=0.95\textwidth]{figures/stats/corr_ex}
\caption{
Example distributions for
uncorrelated ($\rho \approx 0$),
correlated ($\rho \approx 1$),
and anticorrelated ($\rho \approx -1$)
variables $u$ and $v$ \cite{DougNotes}.
}
\label{fig:stats:corr_ex:pearson}
\end{figure}

%%%%%%%%%%%%%%%%%%%%%%%%%%%%%%%%%%%%%%%%%%%%%%%%%%%%%%%%
\subsection{Spearman Correlation}
\label{stats:corr_covar:spearman}

The Spearman rank correlation coefficient $r_{s}$ \cref{eq:stats:corr:spearman} is a non-parametric measure
used to quantify how well two variables are monotonically related,
\ie measure the degree of rank ordering between two variables across the available data points.
In contrast to the Pearson correlation coefficient,
a linear relationship between the variables is not assumed,
only that they increase or decrease in a consistent manner.
The Spearman correlation is simply the Pearson correlation of the variable's ranks,
where the rank function $R\left(x_{i}\right)$ returns the order,
$1, 2, 3, \ldots, m$, of a data point $x_{i}$ after being sorted\footnote{Ties
can be handled via dense ranking $1,2,2,3$, fractional ranking $1,2.5,2.5,4$, or arbitrarily broken with ordinal ranking $1,2,3,4$, \ie row numbering.}.
A comparison to the Pearson correlation coefficient can be found in \cref{fig:stats:corr_ex:spearman}.

\begin{equation}\label{eq:stats:corr:spearman}
r_{s} = \rho_{R\left(u\right),R\left(v\right)} = \frac{\cov{R\left(u\right)}{R\left(v\right)}}{\sigma_{R\left(u\right)}\sigma_{R\left(v\right)}}\,.
\end{equation}

\begin{figure}[H]
  \centering
  \begin{subfigure}[c]{0.48\textwidth}\centering
    \includegraphics[width=\textwidth]{figures/stats/spearman_corr_non_para}
  \caption{Non-Parametric}
  \label{fig:stats:corr_ex:spearman:non_para}
  \end{subfigure}
  ~
  \begin{subfigure}[c]{0.48\textwidth}\centering
    \includegraphics[width=\textwidth]{figures/stats/spearman_corr_outliers}
  \caption{Outliers}
  \label{fig:stats:corr_ex:spearman:outliers}
  \end{subfigure}
\caption{
Comparisons of the Spearman rank correlation and Pearson correlation coefficients,
adapted from
\href{https://en.wikipedia.org/wiki/File:Spearman_fig1.svg}{Skbkekas} and
\href{https://en.wikipedia.org/wiki/File:Spearman_fig3.svg}{Skbkekas}.
Note that non-parametric Spearman rank correlation
only measures the monotonic nature of the data, not its linearity,
and is more resilient to outliers.
\label{fig:stats:corr_ex:spearman}
}
\end{figure}

%%%%%%%%%%%%%%%%%%%%%%%%%%%%%%%%%%%%%%%%%%%%%%%%%%%%%%%%
\subsection{Covariance Matrix}
\label{stats:corr_covar:covar_matrix}

The covariance matrix,

\begin{align}\label{eq:covar_matrix}
  \mb{M} = \begin{pmatrix}
    \sigma_{1}^2 & \cov{1}{2}   & \cov{1}{3}   & \ldots \\
    \cov{1}{2}   & \sigma_{2}^2 & \cov{2}{3}   & \ldots \\
    \cov{1}{3}   & \cov{2}{3}   & \sigma_{3}^2 & \ldots \\
    \vdots       & \vdots       & \vdots       & \ddots
  \end{pmatrix}\,,
\end{align}

\noindent with elements $M_{ij} = \expval{\left(u_{i} - \expval{u}_{i}\right)\left(u_{j}-\expval{u}_{j}\right)}$
is the higher dimensional extension of the covariance.
We can visualize the covariance between variables with
Gaussian error ellipses given by the probability distribution

\begin{equation}\label{eq:stats:P_error_ellipse_k}
P\left(x_{1},x_{2},\ldots,x_{k}\right) = \frac{1}{(2\pi)^{k/2}}\frac{1}{\abs{\mb{M}}^{1/2}}\exp\left[-\frac{1}{2}\left(\vb{x}-\vb*{\mu}\right)^{\transpose}\mb{M}\left(\vb{x}-\vb*{\mu}\right)\right]\,,
\end{equation}

\noindent where the ellipse semi-axes are directed along the eigenvectors of $\mb{M}$.
In two dimensions it is easier to see the equation of the error ellipse itself:

\begin{equation}\label{eq:stats:P_error_ellipse_2}
\begin{split}
P\left(u,v\right) &= \frac{1}{2\pi\sigma_{u}\sigma_{v}}\frac{1}{\sqrt{1-\rho^{2}}}\exp\bigg\{-\frac{1}{2}\bigg[ \\
&\frac{1}{(1-\rho)^{2}}\left(\frac{\left(u-\expval{u}\right)^{2}}{\sigma_{u}^{2}}+\frac{\left(v-\expval{v}\right)^{2}}{\sigma_{v}^{2}}-\frac{2\rho \left(u-\expval{u}\right)\left(v-\expval{v}\right)}{\sigma_{u}\sigma_{v}}\right)\bigg]\bigg\}\,.
\end{split}
\end{equation}

%%%%%%%%%%%%%%%%%%%%%%%%%%%%%%%%%%%%%%%%%%%%%%%%%%%%%%%%
%%%%%%%%%%%%%%%%%%%%%%%%%%%%%%%%%%%%%%%%%%%%%%%%%%%%%%%%
\section{Central Limit Theorem (CLT)}
\label{stats:CLT}

The central limit theorem (CLT) states that
if we take samples of size $n$ from an independent random variable multiple times,
from any distribution\footnote{Where the mean is defined, \ie not the Cauchy or other pathological distributions.},
the sample means will tend to the normal distribution.
In practice, we typically require $\num{30} \lesssim n$ points per sample before we say the CLT applies.
The CLT is an important result in statistics as it lets us
treat many problems in a normally distributed framework,
in particular finding confidence intervals for the sample mean,
and hypothesis testing for sample means with a \ttest or ANOVA.

%%%%%%%%%%%%%%%%%%%%%%%%%%%%%%%%%%%%%%%%%%%%%%%%%%%%%%%%
%%%%%%%%%%%%%%%%%%%%%%%%%%%%%%%%%%%%%%%%%%%%%%%%%%%%%%%%
\section{Bayes' Theorem}
\label{stats:Bayes_rule}

Bayes' theorem follows from the probability of the intersection of two events $A$ and $B$:

\begin{equation}\label{eq:stats:intersection}
P\left(A \cap B\right) = P\left(A \mid B\right) P\left(B\right) = P\left(B \mid A\right) P\left(A\right).
\end{equation}

\noindent Dividing by $P\left(B\right)$ we have:

\begin{equation}\label{eq:stats:Bayes_rule}
\begin{split}
P\left(A \mid B\right) &= \frac{P\left(B \mid A\right) P\left(A\right)}{P\left(B\right)}\,, \\
&= \frac{P\left(B \mid A_{i}\right) P\left(A_{i}\right)}{\sum_{j} P\left(B \mid A_{j}\right)P\left(A_{j}\right)}\,, \\
\text{Posterior} &= \frac{\text{Likelihood} \times \text{Prior}}{\text{Normalization}}\,.
\end{split}
\end{equation}

\subsubsection{Example: Medical Testing}
\label{stats:Bayes_rule:medical_test}

Example: Testing for disease with a \SI{2}{\percent} incidence rate in the wider population.
The test has a \SI{99}{\percent} true positive rate and a \SI{5}{\percent} false positive rate.
What is the probability an individual has the disease if their test is positive?

% https://www.wolframalpha.com/input/?i=(0.99*0.02)%2F((0.99*0.02)%2B(0.05*(1%E2%88%920.02)))
\begin{equation}\label{eq:stats:Bayes_rule:medical_test_1}
\begin{split}
P\left(\text{Infected} \mid +\right) &= \frac{P\left(+ \mid \text{Infected}\right) P\left(\text{Infected}\right)}{P\left(+\right)}\,, \\
 &= \frac{P\left(+ \mid \text{Infected}\right) P\left(\text{Infected}\right)}{
P\left(+ \mid \text{Infected}\right)P\left(\text{Infected}\right) + P\left(+ \mid \text{Healthy}\right)P\left(\text{Healthy}\right)}\,, \\
&= \frac{\num{0.99} \times \num{0.02}}{\num{0.99} \times \num{0.02} + \num{0.05} \times \left(1-\num{0.02}\right)}\,, \\
&\approx \num{0.288}\,.
\end{split}
\end{equation}

\noindent And if we then run a second, independent, test which also comes back positive?

% https://www.wolframalpha.com/input/?i=(0.99*0.288)%2F((0.99*0.288)%2B(0.05*(1%E2%88%920.288)))
\begin{equation}\label{eq:stats:Bayes_rule:medical_test_2}
\begin{split}
P\left(\text{Infected} \mid ++\right) &= \frac{P\left(+ \mid \text{Infected}\right) P\left(\text{Infected} \mid +\right)}{
P\left(+ \mid \text{Infected}\right)P\left(\text{Infected} \mid +\right) + P\left(+ \mid \text{Healthy}\right)P\left(\text{Healthy} \mid +\right)}\,, \\
&= \frac{\num{0.99} \times \num{0.288}}{\num{0.99} \times \num{0.288} + \num{0.05} \times \left(1-\num{0.288}\right)}\,, \\
&\approx \num{0.889}\,.
\end{split}
\end{equation}

\noindent Note that if we ran both tests the first time we would still have:

% https://www.wolframalpha.com/input/?i=(0.99%5E2*0.02)%2F((0.99%5E2*0.02)%2B(0.05%5E2*(1%E2%88%920.02)))
\begin{equation}\label{eq:stats:Bayes_rule:medical_test_3}
\begin{split}
P\left(\text{Infected} \mid ++\right) &= \frac{P\left(++ \mid \text{Infected}\right) P\left(\text{Infected}\right)}{P\left(++\right)}\,, \\
&= \frac{\num{0.99}^{2} \times \num{0.02}}{\num{0.99}^{2} \times \num{0.02} + \num{0.05}^{2} \times \left(1-\num{0.02}\right)}\,, \\
&\approx \num{0.889}\,.
\end{split}
\end{equation}

\subsubsection{Example: Biased Coin}
\label{stats:Bayes_rule:biased_coin}

Consider the case of a bag of $n$ fair coins and $m$ biased coins.
Let $P\left(H \mid \stcomp{F}\right) \equiv p_{H}$ be the \apriori probability of heads $H$ for an biased coin $\stcomp{F}$.
Drawing one coin from the bag, you flip it multiple times recording $h$ heads and $t$ tails.
What is the probability you have drawn an biased coin, $P\left(\stcomp{F} \mid h,t\right)$?

\begin{equation}\label{eq:stats:Bayes_rule:biased_coin_setup}
\begin{gathered}
P\left(F\right) = \frac{n}{n+m}\,,\quad P\left(\stcomp{F}\right) = \frac{m}{n+m}\,, \\
P\left(H \mid F\right) = \frac{1}{2}\,,\quad P\left(h,t \mid F\right) = \left(\frac{1}{2}\right)^{h}\,\left(\frac{1}{2}\right)^{t} = \frac{1}{2^{h+t}}\,, \\
P\left(h,t \mid \stcomp{F}\right) = P\left(H \mid \stcomp{F}\right)^{h} P\left(\stcomp{H} \mid \stcomp{F}\right)^{t} = p_{H}^{h} \left(1-p_{H}\right)^{t}.
\end{gathered}
\end{equation}

\begin{equation}\label{eq:stats:Bayes_rule:biased_coin_solution}
\begin{split}
P\left(\stcomp{F} \mid h,t\right) &= \frac{
P\left(h,t \mid \stcomp{F}\right) P\left(\stcomp{F}\right)}{
P\left(h,t \mid \stcomp{F}\right) P\left(\stcomp{F}\right) + P\left(h,t \mid F\right) P\left(F\right)} \\
&= \frac{
m\,p_{H}^{h} \left(1-p_{H}\right)^{t}}{
m\,p_{H}^{h} \left(1-p_{H}\right)^{t} + n\,2^{-h-t}}\,.
\end{split}
\end{equation}

Some example values are provided in \cref{tab:Bayes_rule:biased_coin}.

\begin{table}[H]
\centering
\begingroup
\renewcommand*{\arraystretch}{1}
\input{tables/coin_examples/bayes_biased_coin/bayes_biased_coin_example.tex}
\endgroup
\caption{
$P\left(\stcomp{F} \mid h,t\right)$ for various values of $h$, $t$, and $p_{H}$ when $m = 50$, $n = 50$.
}
\label{tab:Bayes_rule:biased_coin}
\end{table}

%%%%%%%%%%%%%%%%%%%%%%%%%%%%%%%%%%%%%%%%%%%%%%%%%%%%%%%%
%%%%%%%%%%%%%%%%%%%%%%%%%%%%%%%%%%%%%%%%%%%%%%%%%%%%%%%%
\section{Coin Problems}
\label{stats:coin_problems}

%%%%%%%%%%%%%%%%%%%%%%%%%%%%%%%%%%%%%%%%%%%%%%%%%%%%%%%%
\subsection{Making an Biased Coin Fair}
\label{stats:coin_problems:biased_to_fair}
% https://fivethirtyeight.com/features/can-you-make-an-unfair-coin-fair/
% https://fivethirtyeight.com/features/can-you-snatch-defeat-from-the-jaws-of-victory/
% https://youtu.be/-SANBbv0-Hw
% https://math.stackexchange.com/a/146614 for additional references

If we are given an biased coin with $P\left(H\right) = p \neq \num{0.5}$
we can construct a fair coin\footnote{The solution is credited to John von Neumann, who really did work on everything!} by
flipping the biased coin $m$ times and selecting outcomes with an equal probability of occurring, while disregarding all other outcomes.
For example, with $m=2$ flips we have the outcomes listed in \cref{tab:coin_problems:biased_to_fair:m2_outcomes}:

\begin{table}[H]
\centering
\begingroup
\renewcommand*{\arraystretch}{1}
\begin{tabular}{c c c c}
\hline
Outcome & $P\left(\text{Outcome}\right)$ & Assignment \\
\hline
\hline
$HH$ & $p^{2}$ & Disregard \\
$HT$ & $p\left(1-p\right)$ & H \\
$TH$ & $\left(1-p\right)p$ & T \\
$TT$ & $\left(1-p\right)^{2}$ & Disregard \\
\end{tabular}

\endgroup
\caption{
Making an biased coin fair in $m=2$ flips.
}
\label{tab:coin_problems:biased_to_fair:m2_outcomes}
\end{table}

Note that we are disregarding $p^{2} + \left(1-p\right)^{2}$ percent of flips
and this method will become increasingly inefficient as $p$ moves away from $\num{0.5}$.
We can improve the efficiency by increasing $m$ and reassigning the
outcomes\footnote{For $m = \num{4}$, disregard $P\left(HHHH \parallel TTTT\right) = p^{4} + \left(1-p\right)^{4} < p^{2} + \left(1-p\right)^{2}$,\newline
$HT \parallel HHHT \parallel HHTT \parallel TTHT \to H$,\newline
$TH \parallel TTTH \parallel TTHH \parallel HHTH \to T = H^{-1}$.} such that
we decrease the average number of flips required.
If we need to simulate something with more outcomes, like a 6 sided die,
we can increase $m$ until we have enough equal outcomes, or combinations of outcomes.

We can also ask, for what values of $p$ is a fair outcome ensured in $m$ flips?
To solve, create $P\left(\text{Outcome}\right)$ for all possible outcomes of $m$ flips.
Then, sum all possible combinations\footnote{Excluding the combination with all outcomes where $\sum P\left(\text{Outcome}\right) = 1$.} of $P\left(\text{Outcome}\right)$
and try to solve $\sum P\left(\text{Outcome}\right) = \num{0.5}$.
This partitions the outcomes into two assignments without disregarding any outcomes, \ie no flips are wasted.
For example, with $m=2$ and \cref{tab:coin_problems:biased_to_fair:m2_outcomes}, we solve
$p^{2}$, $p\left(1-p\right)$, $\left(1-p\right)^{2}$,
$p^{2} + p\left(1-p\right)$, $p\left(1-p\right) + \left(1-p\right)^{2}$, and $p^{2} + \left(1-p\right)^{2} = \num{0.5}$
separately for $0<p<1$.
The valid $p$ are then
$p = 1/\sqrt{2}$ with $HH \to H$,
$p = \num{0.5}$ with $HH \parallel TT \to H$,
and $p = 1 - 1/\sqrt{2}$ with $TT \to H$;
everything else $\to T$.

%%%%%%%%%%%%%%%%%%%%%%%%%%%%%%%%%%%%%%%%%%%%%%%%%%%%%%%%
\subsection{Making an Fair Coin Biased}
\label{stats:coin_problems:fair_to_biased}

In the opposite direction, we can also make a fair coin biased
with $P\left(H\right) = p = 1/N$ by flipping it
$m = \text{ceil}\left(\log_{2}N\right)$ times
and reassigning the outcomes such that $H^{m} \to H$,
$N-1$ other outcomes $\to T$, and we disregard any remaining $m-N$ outcomes.

%%%%%%%%%%%%%%%%%%%%%%%%%%%%%%%%%%%%%%%%%%%%%%%%%%%%%%%%
%%%%%%%%%%%%%%%%%%%%%%%%%%%%%%%%%%%%%%%%%%%%%%%%%%%%%%%%
\section{Uniform Distribution}
\label{stats:uniform}

The uniform distribution, \cref{eq:stats:uniform:P} and \cref{fig:dist:uniform},
is the simplest probability distribution
having a constant probability $1/(a+b)$ over the interval $a$ to $b$.

\begin{equation}\label{eq:stats:uniform:P}
P\left(x;\,a,\,b\right) = \begin{cases}
\frac{1}{b-a} & a \leq x \leq b \,, \\
0 & \text{otherwise} \,,
\end{cases}
\end{equation}

It is illustrative to compute the mean and variance of the uniform distribution from $P(x)$ directly:

\begin{subequations}\label{eq:stats:uniform:mean_variance}
\begin{align}
\expval{x} &= \int_{-\infty}^{\infty} x P\left(x\right) \, \dd{x} = \frac{1}{b-a} \int_{a}^{b} x \, \dd{x} = \frac{1}{2(b-a)}\left(b^{2} - a^{2}\right) = \frac{1}{2}\left(a + b\right)\,, \label{eq:stats:uniform:mean_variance:mean} \\
\variance{x} &= \expval{x^{2}} - \expval{x}^{2} = \frac{1}{b-a} \int_{a}^{b} x^{2} \, \dd{x} - \expval{x}^{2} = \cdots = \frac{1}{12}\left(b-a\right)^{2}\,. \label{eq:stats:uniform:mean_variance:variance}
\end{align}
\end{subequations}

%%%%%%%%%%%%%%%%%%%%%%%%%%%%%%%%%%%%%%%%%%%%%%%%%%%%%%%%
%%%%%%%%%%%%%%%%%%%%%%%%%%%%%%%%%%%%%%%%%%%%%%%%%%%%%%%%
\section{Binomial Distribution}
\label{stats:binomial}

The binomial distribution, \cref{eq:stats:binomial:P} and \cref{fig:dist:binomial},
gives the probability of observing $k$ successes in $n$ independent Boolean trials,
when $p$ is the probability of success in any one trial.

\begin{subequations}\label{eq:stats:binomial}
\begin{align}
P\left(k;\,n,p\right) &= \binom{n}{k} p^{k} \left(1-p\right)^{n-k}, \label{eq:stats:binomial:P} \\
\binom{n}{k} &\equiv \frac{n!}{k!\left(n-k\right)!}\,. \label{eq:stats:binomial_coefficient}
\end{align}
\end{subequations}

\noindent Here \cref{eq:stats:binomial_coefficient} is the binomial coefficient,
representing the number of unordered combinations
which select $k$ elements from $n$ elements; $n$ choose $k$.

The mean and variance of the binomial distribution are:

\begin{subequations}\label{eq:stats:binomial:mean_variance}
\begin{align}
\expval{k} &= \sum_{k=0}^{n} k P\left(k;\,n,p\right) = n p\,, \label{eq:stats:binomial:mean} \\
\sigma^{2} &= n p\left(1-p\right). \label{eq:stats:binomial:variance}
\end{align}
\end{subequations}

\subsubsection{Bernoulli Distribution}
\label{stats:binomial:bernoulli}

For the special case when $n=1$, we have the Bernoulli distribution:

\begin{equation}\label{eq:stats:bernoulli}
P\left(k;\,p\right) = p^{k} \left(1-p\right)^{1-k}, \quad \expval{k} = p\,, \quad \sigma^{2} = p\left(1-p\right).
\end{equation}

\subsubsection{Negative Binomial Distribution}
\label{stats:binomial:negative}

If we are interested in the probability of
observing $k$ successes before we observe $r$ failures,
we can slightly modify the binomial distribution to be
the ``negative''\footnote{Negative as in $\binom{k+r-1}{k} = \left(-1\right)^{k} \binom{-r}{k}$.} binomial distribution:

\begin{subequations}\label{eq:stats:binomial:neg:P}
\begin{align}
P\left(k;\,r,p\right) = \binom{k+r-1}{k} p^{k} \left(1-p\right)^{r}\,,
\end{align}
\end{subequations}

\noindent where $p$ is still the probability of a success.
Note that since we stop on the $r^{\text{th}}$ failure,
we need to only arrange the $k$ successes in the first $k+r-1$ trials.

The mean and variance are then:

\begin{subequations}\label{eq:stats:binomial:neg:mean_variance}
\begin{align}
\expval{k} &= r p / \left(1-p\right)\,, \label{eq:stats:binomial:neg:mean} \\
\sigma^{2} &= r p / \left(1-p\right)^{2}. \label{eq:stats:binomial:neg:variance}
\end{align}
\end{subequations}

%%%%%%%%%%%%%%%%%%%%%%%%%%%%%%%%%%%%%%%%%%%%%%%%%%%%%%%%
%%%%%%%%%%%%%%%%%%%%%%%%%%%%%%%%%%%%%%%%%%%%%%%%%%%%%%%%
\section{Poisson Distribution}
\label{stats:poisson}

For rare processes with $p \ll 1$, and thus $\lambda \equiv n\,p \ll 1$,
the binomial distribution reduces\footnote{See
one \href{https://medium.com/@andrew.chamberlain/deriving-the-poisson-distribution-from-the-binomial-distribution-840cc1668239}{derivation here}.
In practice, $\num{20} \lesssim n$, $\lambda = n\,p \lesssim \num{10}$ is a reasonable standard.} to the
Poisson distribution:

\begin{equation}\label{eq:stats:poisson:P}
P\left(k;\,\lambda\right) = \frac{\lambda^{k}}{k!}\,e^{-\lambda}\,, \\
\end{equation}

\noindent where $\lambda$ is the expected number of events in a given interval,
$\lambda = \left(\text{event density}\right) \times \left(\text{interval}\right)$.
Note that $\lambda$ has dimensionless units of counts of events\footnote{Although
we are typically considering $\lambda$ events in a given time interval,
$\lambda_{\text{Poisson}}$ is not a rate, yet!} and does not have to be an integer.
When events occur independently and at a constant rate,
which is low enough that no events occur simultaneously within our measurement resolution,
the event count per time interval will follow the Poisson distribution.
Such a situation is known as a Poisson process.
A plot of the distribution can be found in \cref{fig:dist:poisson}.

Interestingly, the mean and variance of the Poisson distribution are identical and equal to $\lambda$:

\begin{equation}\label{eq:stats:poisson:mean_variance}
\expval{k} = \sigma^{2} = \lambda\,.
\end{equation}

%%%%%%%%%%%%%%%%%%%%%%%%%%%%%%%%%%%%%%%%%%%%%%%%%%%%%%%%
%%%%%%%%%%%%%%%%%%%%%%%%%%%%%%%%%%%%%%%%%%%%%%%%%%%%%%%%
\section{Exponential Distribution}
\label{stats:exp}

If we sample the time intervals between Poisson distributed events\footnote{Assuming the first event occurs at $T$,
$P\left(T > t\right) = P_{\text{Poisson}}\left(k=0;\,\lambda_{\text{Poisson}} \equiv \lambda\,t\right) = e^{-\lambda t}$,
$P\left(t\right) = \dv{t} P\left(T \leq t\right) = \dv{t} 1 - e^{-\lambda t} = \lambda e^{-\lambda t}$.
Here we have used the Poisson distribution to find the CDF,
then differentiated to find the PDF.
$\lambda$ is now a rate, unlike $\lambda_{\text{Poisson}}$!} we
arrive at the exponential distribution,

\begin{equation}\label{eq:stats:exp:P}
P\left(x;\,\lambda\right) = \begin{cases}
\lambda e^{-\lambda x} & x \geq 0 \,, \\
0 & x < 0 \,,
\end{cases}
\end{equation}

\noindent with rate parameter $\lambda$ events per unit $x$.
A plot of the distribution can be found in \cref{fig:dist:exp}.

The mean and standard deviation of the exponential distribution are identical and equal to $1/\lambda$:

\begin{equation}\label{eq:stats:exp:mean_variance}
\expval{x} = \sigma = \frac{1}{\lambda}\,, \quad \sigma^{2} = \frac{1}{\lambda^{2}}\,.
\end{equation}

The exponential distribution is memoryless\footnote{The exponential (geometric) distribution is the only real number (integer) memoryless probability distribution.} \cref{eq:stats:exp:memoryless},
\ie the time until a future event occurs does not depend on the time elapsed at the present.

\begin{equation}\label{eq:stats:exp:memoryless}
P\left(T > t + s \mid T > s\right) = P\left(T > t\right)\,, \quad \forall s,t \geq 0.
\end{equation}

%%%%%%%%%%%%%%%%%%%%%%%%%%%%%%%%%%%%%%%%%%%%%%%%%%%%%%%%
%%%%%%%%%%%%%%%%%%%%%%%%%%%%%%%%%%%%%%%%%%%%%%%%%%%%%%%%
\section{Example: Cars Driving By}
\label{stats:cars}

After watching a patch of road,
you notice that on average cars are spaced \SI{5}{\minute} apart.
Assume the cars are traveling independently\footnote{Not a good assumption
in real life as cars will clump up in traffic, \ie become correlated.
Better examples include
radioactive decay,
photon arrival from a distant star,
support tickets being created\ldots} of one another.

\subsubsection{Questions}
\label{stats:cars:questions}

\begin{enumerate}[noitemsep]
  \item What is $P\left(\text{Observing \num{5} cars in the next \SI{6}{\minute}}\right)$?\label{item:stats:cars:1}
  \item If \SI{20}{\percent} of the cars are red, what is $P\left(\text{Observing \num{5} red cars in the next \SI{6}{\minute}}\right)$?\label{item:stats:cars:2}
  \item What is $P\left(\text{Observing \num{2} or more cars in the next \SI{10}{\minute}}\right)$?\label{item:stats:cars:3}
  \item \num{100} cars have just driven by, how many are expected in the next hour?\label{item:stats:cars:4}
  \item If we start watching the road at a random time, how long should we expect to wait until we see our first car?\label{item:stats:cars:5}
\end{enumerate}

\subsubsection{Solutions}
\label{stats:cars:solutions}

\begin{itemize}[noitemsep]
  % https://www.wolframalpha.com/input/?i=%281.2%5E5+%2F+5%21%29*e%5E-1.2
  \item[\cref{item:stats:cars:1}.] $k = 5$, $\lambda = \frac{1\,\text{car observed}}{\SI{5}{\minute}} \times \SI{6}{\minute} = \num{1.2}$, $P_{\text{Poisson}} = \frac{\num{1.2}^{5}}{5!} e^{-1.2} = \num{0.006}$.
  % https://www.wolframalpha.com/input/?i=%280.24%5E5+%2F+5%21%29*e%5E-0.24
  \item[\cref{item:stats:cars:2}.] $\lambda = \frac{\num{0.2} \times 1\,\text{car observed}}{\SI{5}{\minute}} \times \SI{6}{\minute} = \num{0.24}$, $P_{\text{Poisson}} = \num{5.2e-6}$.
  % https://www.wolframalpha.com/input/?i=1+-+3e%5E-2
  \item[\cref{item:stats:cars:3}.] $\lambda = \frac{1}{\SI{5}{\minute}} \times \SI{10}{\minute} = \num{2}$, $P\left(2 \leq k \right) = 1 - P_{\text{Poisson}}\left(k < 2 \right) = 1 - \left( \frac{2^{0}}{0!} e^{-2} + \frac{2^{1}}{1!} e^{-2}\right) \newline = 1 - 3 e^{-2} = \num{0.6}$.
  \item[\cref{item:stats:cars:4}.] The cars are independent, so the prior $k$ cars are irrelevant -- gambler's fallacy. $\expval{k} = \lambda = \frac{1}{\SI{5}{\minute}} \times \SI{60}{\minute} = \num{12}$.
  \item[\cref{item:stats:cars:5}.] We can use the exponential distribution with $\lambda_{\text{Exp}} = \frac{1}{\SI{5}{\minute}}$, $\expval{t} = \frac{1}{\lambda} = \SI{5}{\minute}$,\newline as one would expect\ldots
\end{itemize}

%%%%%%%%%%%%%%%%%%%%%%%%%%%%%%%%%%%%%%%%%%%%%%%%%%%%%%%%
%%%%%%%%%%%%%%%%%%%%%%%%%%%%%%%%%%%%%%%%%%%%%%%%%%%%%%%%
\section{Gaussian Distribution}
\label{stats:gaus}

In the other direction, for common processes $\mu \equiv n\,p \gg 1$,
the binomial distribution becomes\footnote{See
one \href{http://scipp.ucsc.edu/~haber/ph116C/NormalApprox.pdf}{derivation here}.} the
well-known Gaussian, or normal, distribution,

\begin{equation}\label{eq:stats:gaus:P}
P\left(x;\,\mu,\sigma\right) = \frac{1}{\sqrt{2\pi}\,\sigma} \exp\left( -\frac{1}{2} \left(\frac{x-\mu}{\sigma}\right)^{2} \right)\,,
\end{equation}

\noindent with mean $\mu$ and variance $\sigma^{2}$.
A plot of the distribution can be found in \cref{fig:dist:gaus}.

It is helpful to remember that

\begin{equation}\label{eq:stats:gaus:sigmas}
\begin{split}
\mu \pm 1\sigma &\approx \SI{68}{\percent}\,, \\
\mu \pm 2\sigma &\approx \SI{95}{\percent}\,, \\
\mu \pm 3\sigma &\approx \SI{99}{\percent}\,,
\end{split}
\end{equation}

\noindent while the full width at half maximum (FWHM) $\Gamma = 2\sqrt{2 \ln{2}}\,\sigma \approx \num{2.355} \sigma$.
Note that the standard normal distribution has $\mu = 0$ and $\sigma =1$.

%%%%%%%%%%%%%%%%%%%%%%%%%%%%%%%%%%%%%%%%%%%%%%%%%%%%%%%%
%%%%%%%%%%%%%%%%%%%%%%%%%%%%%%%%%%%%%%%%%%%%%%%%%%%%%%%%
\section{Student's \texorpdfstring{$t$}{t}-Distribution}
\label{stats:t_dist}

If we take a sample of size $m$ from a normal distribution
we arrive at Student's \tdist.
Letting $\nu = m-1$ be the number of degrees of freedom,
Student's \tdist has the form:

\begin{equation}\label{eq:stats:t_dist:P}
P\left(t;\,\nu\right) = \frac{
\Gamma\left(\frac{\nu+1}{2}\right)
}{
\sqrt{\nu \pi}\,\Gamma\left(\frac{\nu}{2}\right)
} \left(1+\frac{t^{2}}{\nu}\right)^{-\frac{\nu+1}{2}}\,,
\end{equation}

\noindent where $\Gamma\left(z\right)$ is the gamma function\footnote{$\Gamma\left(z\right) = \int_{0}^{\infty} x^{z-1} e^{-x} \, \dd{x}$,
which simplifies to $\Gamma\left(n\right)=\left(n-1\right)!$ for integer $n$.}.
The distribution, \cref{fig:dist:student_t}, has a mean of \num{0} for $\nu > 1$,
and variance of $\infty$ for $\nu =2$ and $\nu / \left(\nu-2\right)$ for $\nu > 2$.

%%%%%%%%%%%%%%%%%%%%%%%%%%%%%%%%%%%%%%%%%%%%%%%%%%%%%%%%
%%%%%%%%%%%%%%%%%%%%%%%%%%%%%%%%%%%%%%%%%%%%%%%%%%%%%%%%
\section{\texorpdfstring{$\chi^{2}$-Distribution}{Chi-Squared Distribution}}
\label{stats:chi2_dist}

The \chiSqdist with $\nu$ degrees of freedom,
\cref{eq:stats:chi2_dist:P} and \cref{fig:dist:chi2},
is created by summing the squares of $\nu$ independent standard normal random variables.
It therefore has a mean of $\nu$ and variance of $2\nu$, and is
primary useful for conducting {\chiSqtest}s.

\begin{equation}\label{eq:stats:chi2_dist:P}
P\left(x;\,\nu\right) = \frac{
x^{\frac{\nu}{2} - 1} e^{-\frac{x}{2}}
}{
2^{\frac{\nu}{2}} \Gamma\left(\frac{\nu}{2}\right)}.
\end{equation}

%%%%%%%%%%%%%%%%%%%%%%%%%%%%%%%%%%%%%%%%%%%%%%%%%%%%%%%%
%%%%%%%%%%%%%%%%%%%%%%%%%%%%%%%%%%%%%%%%%%%%%%%%%%%%%%%%
\section{\texorpdfstring{$F$}{F}-Distribution}
\label{stats:F_dist}

The \Fdist with degrees of freedom $d_{1}$ and $d_{2}$,
\cref{eq:stats:F_dist:X,eq:stats:F_dist:P} and \cref{fig:dist:F},
is formed by dividing two $\chi^{2}$-distributed random variables
$S_{1}$ and $S_{2}$ with their respective degrees of freedom:

\begin{subequations}\label{eq:stats:F_dist}
\begin{align}
X_{\Fdist} &= \frac{S_{1}/d_{1}}{S_{2}/d_{2}}, \label{eq:stats:F_dist:X} \\
P\left(x;\,d_{1},d_{2}\right) &= \frac{1}{x B\left(d_{1}/2, d_{2}/2\right)} \sqrt{\frac{\left(d_{1} x\right)^{d_{1}} d_{2}^{d_{2}}}{\left(d_{1} x + d_{2}\right)^{d_{1}+d_{2}}}}, \label{eq:stats:F_dist:P}
\end{align}
\end{subequations}

\noindent where $B\left(x,y\right)$ is the beta function\footnote{$B\left(x,y\right) = \int_{0}^{1} t^{x-1} \left(1-t\right)^{y-1} \, \dd{t} = \Gamma\left(x\right)\Gamma\left(y\right) / \Gamma\left(x+y\right)$.}.

The \Fdist has a complicated
mean and variance\footnote{Mean: $d_{2}/\left(d_{2}-2\right)$ for $2 < d_{2}$.
Variance: $2 d_{2}^{2}\left(d_{1}+d_{2}-2\right)/\left(d_{1}\left(d_{2}-2\right)^{2}\left(d_{2}-4\right)\right)$ for $4 < d_{2}$.},
and is primary useful for conducting {\Ftest}s such as ANOVA.

%%%%%%%%%%%%%%%%%%%%%%%%%%%%%%%%%%%%%%%%%%%%%%%%%%%%%%%%
%%%%%%%%%%%%%%%%%%%%%%%%%%%%%%%%%%%%%%%%%%%%%%%%%%%%%%%%
\section{Geometric Distribution}
\label{stats:geo_dist}

The geometric distribution gives the probability of
requiring $k$ independent Boolean trials to observe one success\footnote{With the success occurring in trial $k$.},
\cref{eq:stats:geo:P_trials} and \cref{fig:dist:geometric_trials},
or the probability of observing $k$ failed trials before the first success,
\cref{eq:stats:geo:P_failures} and \cref{fig:dist:geometric_failures}.
As usual $p$ is the probability of success for any single trial.
Note that $k$ is indexed differently depending on the framing of the problem.

\begin{subequations}\label{eq:stats:geo}
\begin{align}
P_{\text{Trials}}\left(k;\,p\right) &= \left(1-p\right)^{k-1}\,p,\quad k \in \left\{1,2,3,\ldots\right\}, \label{eq:stats:geo:P_trials} \\
P_{\text{Failures}}\left(k;\,p\right) &= \left(1-p\right)^{k}\,p,\quad k \in \left\{0, 1,2,3,\ldots\right\}. \label{eq:stats:geo:P_failures}
\end{align}
\end{subequations}

The mean and standard deviation of the geometric distribution are:

\begin{subequations}\label{eq:stats:geo_dist:mean_variance}
\begin{gather}
\expval{k}_{\text{Trials}} = \frac{1}{p}\,, \quad \expval{k}_{\text{Failures}} = \frac{1-p}{p}\,, \label{eq:stats:geo_dist:mean} \\
\sigma^{2} = \frac{1-p}{p^{2}}\,. \label{eq:stats:geo_dist:variance}
\end{gather}
\end{subequations}

Like the exponential distribution, its continuous analogue, the geometric distribution is memoryless \cref{eq:stats:exp:memoryless}.

%%%%%%%%%%%%%%%%%%%%%%%%%%%%%%%%%%%%%%%%%%%%%%%%%%%%%%%%
%%%%%%%%%%%%%%%%%%%%%%%%%%%%%%%%%%%%%%%%%%%%%%%%%%%%%%%%
\section{Hypergeometric Distribution}
\label{stats:hypergeo_dist}

The hypergeometric distribution, \cref{eq:stats:hypergeometric:P} and \cref{fig:dist:hypergeometric},
gives the probability of obtaining $k$ ``successes'' in $n$ draws, {\em without replacement},
from a population of size $N$ which contains $K$ successes.
One example is getting $k$ red balls, \ie successes,
out of $n$ draws from a bag of $K$ red and $N-K$ white balls.

\begin{equation}\label{eq:stats:hypergeometric:P}
P\left(k;\,N,K,n\right) = \frac{\binom{K}{k}\binom{N-K}{n-k}}{\binom{N}{n}}\,.
\end{equation}

The mean and standard deviation of the hypergeometric distribution are:

\begin{subequations}\label{eq:stats:hypergeometric:neg:mean_variance}
\begin{gather}
\expval{k} = n K / N = n p, \label{eq:stats:hypergeometric:mean} \\
\sigma^{2} = n \frac{K}{N} \frac{N-K}{N} \frac{N-n}{N-1} = n p \left(1-p\right) \frac{N-n}{N-1}\,. \label{eq:stats:hypergeometric:variance}
\end{gather}
\end{subequations}

The hypergeometric distribution should not be confused with the geometric distribution,
and is far more similar to the binomial distribution where $p = K/N$ and we draw with replacement.

%%%%%%%%%%%%%%%%%%%%%%%%%%%%%%%%%%%%%%%%%%%%%%%%%%%%%%%%
%%%%%%%%%%%%%%%%%%%%%%%%%%%%%%%%%%%%%%%%%%%%%%%%%%%%%%%%
\section{Kurtosis and Skewness}
\label{stats:kurtosis_skewness}
% TODO

%%%%%%%%%%%%%%%%%%%%%%%%%%%%%%%%%%%%%%%%%%%%%%%%%%%%%%%%
%%%%%%%%%%%%%%%%%%%%%%%%%%%%%%%%%%%%%%%%%%%%%%%%%%%%%%%%
\section{Confidence Intervals}
\label{stats:CI}
% TODO

%%%%%%%%%%%%%%%%%%%%%%%%%%%%%%%%%%%%%%%%%%%%%%%%%%%%%%%%
\subsection{Hoeffding's Inequality}
\label{stats:CI:hoeffding}
% TODO

%%%%%%%%%%%%%%%%%%%%%%%%%%%%%%%%%%%%%%%%%%%%%%%%%%%%%%%%
%%%%%%%%%%%%%%%%%%%%%%%%%%%%%%%%%%%%%%%%%%%%%%%%%%%%%%%%
\section{Outliers}
\label{stats:outliers}
% TODO
% interquartile range (IQR)

\begin{equation}\label{eq:stats:outlier_IQR_def}
\expval{x} \pm \num{1.5}\, \text{IQR}
\end{equation}

%%%%%%%%%%%%%%%%%%%%%%%%%%%%%%%%%%%%%%%%%%%%%%%%%%%%%%%%
%%%%%%%%%%%%%%%%%%%%%%%%%%%%%%%%%%%%%%%%%%%%%%%%%%%%%%%%
\section{Bootstrapping}
\label{stats:bootstrapping}

The bootstrap method can be a great solution when presented with difficult problems
around measuring uncertainties or even conducting hypothesis tests.
With bootstrapping we can estimate the uncertainty on any well defined quantity,
even when the uncertainty itself does not have an explicit expression,
\eg finding the confidence intervals for the median
or coefficient of determination $R^{2}$ of linear regression.

We begin with an original sample of size $n$ drawn from the larger population,
and generate $1 \ll m$ bootstrap samples by resampling $n$ points from the sample with replacement,
as illustrated in \cref{fig:bootstrapping}.
The statistic of interest is computed on each of the bootstrapped samples
before being combined into an overall distribution.
Assuming the original population is independent and identically distributed (\iid)\footnote{Note
that the \iid assumption is less stringent than
the usual assumption of normality via the central limit theorem (CLT) of \cref{stats:CLT},
allowing bootstrapping to be used in cases were $n$ may be to small for other methods.},
the resulting bootstrapped distribution will approximate the true sampling distribution
and we can use it to estimate the uncertainty on the statistic.
The \SI{95}{\percent} confidence interval of the statistic in question
can be found by simply\footnote{There are
more advanced methods for determining the confidence interval not covered here,
particularly for asymmetric distributions.} identifying
the boundaries of the central \SI{95}{\percent} of bootstrapped sample values.
Likewise, the standard error of the statistic is the standard deviation of the bootstrapped distribution.

If the bootstrapped distribution is not symmetrical,
which may be the case when working with measures of variance like the standard deviation,
the bootstrapped result may be biased.
In these cases, one potential correction is to shift the bootstrapped distribution
such that the original sample statistic and mean of the bootstrapped distribution agree.

\begin{figure}
\centering
\includegraphics[width=0.7\textwidth]{figures/stats/bootstrapping.jpeg}
\caption{
Illustration of the bootstrapping method
by \href{https://towardsdatascience.com/bootstrapping-statistics-what-it-is-and-why-its-used-e2fa29577307}{Trist'n Joseph}.
Note that the $m$ resamples are done with replacement.
The final bootstrapped distribution is built by measuring the statistic in question on all of the bootstrapped samples,
and can be used to estimate the statistic's uncertainty.
}
\label{fig:bootstrapping}
\end{figure}

%%%%%%%%%%%%%%%%%%%%%%%%%%%%%%%%%%%%%%%%%%%%%%%%%%%%%%%%
\subsection{Hypothesis Testing}
\label{stats:bootstrapping:hypo}

As we can construct confidence intervals, bootstrapping can also be used for hypothesis testing.
N{a\"i}vely we can test the null hypothesis by checking if the value in question,
typically \num{0}, lands within the confidence interval or not.
We can also estimate the \pvalue of a parameter by shifting the original sample distribution such that
it satisfies the null hypothesis, \eg shift all the data points by $\delta$ such that the sample has $\expval{x} = \num{0}$.
We then bootstrap the shifted sample to produce the bootstrapped distribution as normal.
The \pvalue is then the proportion of the bootstrapped distribution
with values differing from the null hypothesis by the observed difference or more,
\eg $P\left(\delta \leq \abs{x}\right)$ in the prior example.

%%%%%%%%%%%%%%%%%%%%%%%%%%%%%%%%%%%%%%%%%%%%%%%%%%%%%%%%
\subsection{Poisson Bootstrapping}
\label{stats:bootstrapping:poisson}

For large sample sizes $n$, instead of drawing $m$ bootstrap samples with replacement,
which can be computationally expensive,
we can use the single sample,
but give each data point a weight $w$ drawn
from the Poisson distribution with $\lambda = 1$.
The set of weights is then generated $m$ times.
Formally, the methods are equivalent as for large $n$
the binomial probability of including a given data point approaches the Poisson probability with $\lambda = 1$.
Poisson Bootstrapping is commonly used in particle physics,
in particular see the direct balance $\gamma\text{+Jet}$ calibration
in the dissertation \cite{mepland_dissertation}, Appendix E.1.4.

%%%%%%%%%%%%%%%%%%%%%%%%%%%%%%%%%%%%%%%%%%%%%%%%%%%%%%%%
%%%%%%%%%%%%%%%%%%%%%%%%%%%%%%%%%%%%%%%%%%%%%%%%%%%%%%%%
\section{Bayesian Statistics}
\label{stats:Bayes}
% TODO

% \cref{stats:Bayes_rule}

%%%%%%%%%%%%%%%%%%%%%%%%%%%%%%%%%%%%%%%%%%%%%%%%%%%%%%%%
\subsection{Bayesian Conjugate Prior}
\label{stats:Bayes:conjugate_prior}
% TODO

%%%%%%%%%%%%%%%%%%%%%%%%%%%%%%%%%%%%%%%%%%%%%%%%%%%%%%%%
\subsection{Bayes Estimator}
\label{stats:Bayes:estimator}
% TODO

%%%%%%%%%%%%%%%%%%%%%%%%%%%%%%%%%%%%%%%%%%%%%%%%%%%%%%%%
\subsection{Approximate Bayesian Computation (ABC)}
\label{stats:Bayes:ABC}
% TODO

%%%%%%%%%%%%%%%%%%%%%%%%%%%%%%%%%%%%%%%%%%%%%%%%%%%%%%%%
%%%%%%%%%%%%%%%%%%%%%%%%%%%%%%%%%%%%%%%%%%%%%%%%%%%%%%%%
\section{Bias of a Predictor}
\label{stats:bias}
% TODO

\begin{equation}\label{eq:stats:bias}
\bias{\hat{f}\left(x\right)} = \expval{\hat{f}\left(x\right)} - f\left(x\right)
\end{equation}

%%%%%%%%%%%%%%%%%%%%%%%%%%%%%%%%%%%%%%%%%%%%%%%%%%%%%%%%
%%%%%%%%%%%%%%%%%%%%%%%%%%%%%%%%%%%%%%%%%%%%%%%%%%%%%%%%
\section{Kullback-Leibler Divergence}
\label{stats:kld}
% TODO

%%%%%%%%%%%%%%%%%%%%%%%%%%%%%%%%%%%%%%%%%%%%%%%%%%%%%%%%
%%%%%%%%%%%%%%%%%%%%%%%%%%%%%%%%%%%%%%%%%%%%%%%%%%%%%%%%
\section{Markov Chains}
\label{misc:markov_chains}
% TODO

%%%%%%%%%%%%%%%%%%%%%%%%%%%%%%%%%%%%%%%%%%%%%%%%%%%%%%%%
%%%%%%%%%%%%%%%%%%%%%%%%%%%%%%%%%%%%%%%%%%%%%%%%%%%%%%%%
\section{Markov Chain Monte Carlo (MCMC)}
\label{misc:MCMC}
% TODO

% Markov Chain Monte Carlo (MCMC) is a Monte Carlo (MC) method

%%%%%%%%%%%%%%%%%%%%%%%%%%%%%%%%%%%%%%%%%%%%%%%%%%%%%%%%
\subsection{\pymcThree}% would rather have the \textsc caps than italics
\label{misc:MCMC:PyMC3}
% TODO
% TODO \pymcThree

%%%%%%%%%%%%%%%%%%%%%%%%%%%%%%%%%%%%%%%%%%%%%%%%%%%%%%%%
%%%%%%%%%%%%%%%%%%%%%%%%%%%%%%%%%%%%%%%%%%%%%%%%%%%%%%%%
\section{Sampling from Probability Distributions}
\label{misc:sampling_prob_dist}

Most programming languages have built-in functions to generate
pseudo-random numbers from common probability distributions,
such as the Poisson \cref{eq:stats:poisson:P} or Gaussian \cref{eq:stats:gaus:P} distributions.
However, in some cases we may wish to sample from an unsupported esoteric function.
In these situations we can turn to inverse transform sampling and rejection sampling,
to give just two examples from many possible computational methods,
to construct the desired distribution from an existing random number generator.

%%%%%%%%%%%%%%%%%%%%%%%%%%%%%%%%%%%%%%%%%%%%%%%%%%%%%%%%
\subsection{Inverse Transform Sampling}
\label{misc:sampling_prob_dist:inverse}
% https://www.youtube.com/watch?v=9ixzzPQWuAY

If we know the explicit form of the target probability density function (PDF), $X = P\left(x\right)$,
can integrate it to find the cumulative distribution function (CDF), $F_{X}\left(x\right) = \int_{-\infty}^{x} P\left(t\right) \, \dd{t}$,
and furthermore can invert the CDF\footnote{The inverse CDF
is known as the percent point function, \texttt{ppf},
in \href{https://docs.scipy.org/doc/scipy/reference/stats.html}{\scipy}.}, $F^{-1}_{X}\left(u\right)$ for $0 \leq u \leq 1$,
we can explicitly transform the uniform distribution $U\left(x\right)$ into $P\left(x\right)$ as:

\begin{equation}\label{eq:stats:sampling_prob_dist:inverse}
F^{-1}_{X}\left(U\right) = P\left(x\right) = X\,.
\end{equation}

To prove the method, assuming that $F^{-1}_{X}$ exists, we can do:

\begin{subequations}\label{eq:stats:sampling_prob_dist:inverse_proof}
\begin{align}
P\left(F^{-1}_{X}\left(U\right) \leq x\right) &= P\left(U \leq F_{X}\left(x\right) \right) \label{eq:stats:sampling_prob_dist:inverse_proof:inverse} \\
&= F_{X}\left(x\right)\,,\label{eq:stats:sampling_prob_dist:inverse_proof:def} \\
P\left(U \leq y\right) &= y\,, \label{eq:stats:sampling_prob_dist:inverse_proof:U}
\end{align}
\end{subequations}

\noindent where in \cref{eq:stats:sampling_prob_dist:inverse_proof:inverse} we have applied $F$ to both sides of the inner inequality,
and in \cref{eq:stats:sampling_prob_dist:inverse_proof:def} we have used the definition of the uniform distribution \cref{eq:stats:sampling_prob_dist:inverse_proof:U}.
As $P\left(F^{-1}_{X}\left(U\right) \leq x\right) = F_{X}\left(x\right) = P\left(X \leq x\right)$
we can compare terms and see \cref{eq:stats:sampling_prob_dist:inverse}.

The inverse sampling method can be seen graphically \cref{fig:stats:sampling_prob_dist:inverse} being used to generate the normal distribution.

%\begin{figure}
%\centering
%\includegraphics[width=0.7\textwidth]{figures/stats/inverse_transform_sampling_normal_dist}
%\caption{
%Example application of the inverse sampling method to generate the normal distribution, adapted from \href{https://en.wikipedia.org/wiki/File:Inverse_transform_sampling.png}{Olivier Ricou}.
%}
%\label{fig:stats:sampling_prob_dist:inverse}
%\end{figure}

%%%%%%%%%%%%%%%%%%%%%%%%%%%%%%%%%%%%%%%%%%%%%%%%%%%%%%%%
\subsection{Rejection Sampling}
\label{misc:sampling_prob_dist:reject}
% https://www.youtube.com/watch?v=OXDqjdVVePY

While inverse transform sampling is a computationally efficient method
its assumptions will not be met by all interesting PDFs,
in particular if we do not have an explicit, invertible CDF.
In these cases we can turn to rejection sampling instead,
which can handle more general PDFs at the cost of a lower computational efficiency.

In rejection sampling, we assume we know the PDF of the random variable to be sampled from,
up to a normalization constant $A$ \cref{eq:stats:sampling_prob_dist:reject:X}.
We then choose a different PDF, $g\left(x\right)$ which is easy for us to sample from.
Any $g\left(x\right)$ covering the range of $x$ will do,
but the method is more efficient the closer we can get $g\left(x\right)$ to match $f\left(x\right)$.
$g\left(x\right)$ is then scaled by a known constant $M$
such that it is always larger than $f\left(x\right)$ \cref{eq:stats:sampling_prob_dist:reject:f_condition}
as illustrated in \cref{fig:stats:sampling_prob_dist:reject}.
Finally, we sample random $x$ values from $g\left(x\right)$ and accept them with probability \cref{eq:stats:sampling_prob_dist:reject:P_accept}.
The accepted $x$ values will have the same distribution as $X$.

\begin{subequations}\label{eq:stats:sampling_prob_dist:reject}
\begin{align}
X = P\left(x\right) &= \frac{1}{A} f\left(x\right), \label{eq:stats:sampling_prob_dist:reject:X} \\
\forall x, \quad f\left(x\right) & \leq M g\left(x\right), \label{eq:stats:sampling_prob_dist:reject:f_condition} \\
P\left(\text{Accept} \mid x\right) &= \frac{f\left(x\right)}{M g\left(x\right)}\,. \label{eq:stats:sampling_prob_dist:reject:P_accept}
\end{align}
\end{subequations}

\begin{figure}[H]
  \centering
  \savebox{\largestimage}{
    \includegraphics[width=0.47\textwidth,trim={3.0cm 0.5cm 3.0cm 0.1cm},clip]{figures/stats/inverse_transform_sampling_normal_dist}% trim={<left> <lower> <right> <upper>}
  }% Store largest image in a box

  \begin{subfigure}[b]{0.48\textwidth}\centering
    \usebox{\largestimage}
    \vspace{0.01cm}
  \caption{Inverse Sampling}
  \label{fig:stats:sampling_prob_dist:inverse}
  \end{subfigure}
  ~
  \begin{subfigure}[b]{\wd\largestimage}\centering
    \raisebox{\dimexpr.5\ht\largestimage-.5\height}{% Adjust vertical height of smaller image
      \includegraphics[width=\textwidth]{figures/stats/rejection_sampling}}
  \caption{Rejection Sampling}
  \label{fig:stats:sampling_prob_dist:reject}
  \end{subfigure}
\caption{
Illustrations of the inverse sampling and rejection sampling methods, adapted from \href{https://en.wikipedia.org/wiki/File:Inverse_transform_sampling.png}{Olivier Ricou} and \href{https://www.data-blogger.com/2016/01/24/the-mathematics-behind-rejection-sampling/}{Kevin Jacobs}.
  \label{fig:stats:sampling_prob_distr}
}
\end{figure}

For a proof of why the rejection sampling method works see \cref{eq:stats:sampling_prob_dist:reject:proof} below
which hinges on Bayes' Theorem from \cref{stats:Bayes_rule} in \cref{eq:stats:sampling_prob_dist:reject:proof:x_given_Accept}
along with other basic definitions from probability theory.

\begin{subequations}\label{eq:stats:sampling_prob_dist:reject:proof}
\begin{align}
P\left(x \mid \text{Accept}\right) &= \frac{P\left(\text{Accept} \mid x\right) P\left(x\right)}{P\left(\text{Accept}\right)} = \frac{1}{P\left(\text{Accept}\right)} \frac{f\left(x\right)}{M \cancel{g\left(x\right)}} \cancel{g\left(x\right)}, \label{eq:stats:sampling_prob_dist:reject:proof:x_given_Accept} \\
P\left(\text{Accept}\right) &= \int_{-\infty}^{\infty} P\left(\text{Accept} \mid x\right) g\left(x\right) \, \dd{x} \label{eq:stats:sampling_prob_dist:reject:proof:P_Accept1} \\
&= \int_{-\infty}^{\infty} \frac{f\left(x\right)}{M g\left(x\right)} g\left(x\right) \, \dd{x} = \frac{1}{M} \int_{-\infty}^{\infty} f\left(x\right) \, \dd{x} = \frac{A}{M}, \label{eq:stats:sampling_prob_dist:reject:proof:P_Accept2} \\
\implies P\left(x \mid \text{Accept}\right) &= \frac{f\left(x\right)/M}{A/M} = \frac{1}{A} f\left(x\right) = P\left(x\right) = X. \label{eq:stats:sampling_prob_dist:reject:proof:conclusion}
\end{align}
\end{subequations}

%%%%%%%%%%%%%%%%%%%%%%%%%%%%%%%%%%%%%%%%%%%%%%%%%%%%%%%%
\subsection{Metropolis-Hastings Algorithm}
\label{misc:sampling_prob_dist:metropolis_hastings}
% TODO
}
%%%%%%%%%%%%%%%%%%%%%%%%%%%%%%%%%%%%%%%%%%%%%%%%%%%%%%%%
%%%%%%%%%%%%%%%%%%%%%%%%%%%%%%%%%%%%%%%%%%%%%%%%%%%%%%%%
%%%%%%%%%%%%%%%%%%%%%%%%%%%%%%%%%%%%%%%%%%%%%%%%%%%%%%%%
\chapter{Hypothesis Testing}
\label{chap:hypo}

%%%%%%%%%%%%%%%%%%%%%%%%%%%%%%%%%%%%%%%%%%%%%%%%%%%%%%%%
%%%%%%%%%%%%%%%%%%%%%%%%%%%%%%%%%%%%%%%%%%%%%%%%%%%%%%%%
\section{\texorpdfstring{$Z$}{Z}-Test}
\label{hypo:Z_test}
% TODO

%%%%%%%%%%%%%%%%%%%%%%%%%%%%%%%%%%%%%%%%%%%%%%%%%%%%%%%%
%%%%%%%%%%%%%%%%%%%%%%%%%%%%%%%%%%%%%%%%%%%%%%%%%%%%%%%%
\section{Student's \texorpdfstring{$t$}{t}-Test}
\label{hypo:t_test}
Student's $t$-test can be used to
statistically compare the means of two samples via the $t$-distribution
given a $t$-statistic and degrees of freedom $\nu$.
The $t$-test is appropriate when each sample is small\footnote{For
larger sample sizes the $t$-distribution approaches the normal distribution which should be used instead.}, $n \lesssim 30$,
and drawn from a larger normally distributed population with an unknown standard deviation.
The \pvalue returned by the test estimates the probability of obtaining the sample means
assuming the null hypothesis, typically that the samples share the same mean, is true.
We can compute two-sided, \ie the means are not equal, or one-sided, \ie mean 1 is $>$ or $<$ mean 2,
forms of the $t$-test in three basic variations.
See \cref{fig:two_sided_t_test} for an illustration of a two-sided test's null hypothesis rejection regions.
The \texttt{scipy.stats.ttest\_*}
\href{https://docs.scipy.org/doc/scipy/reference/stats.html#statistical-tests}{family of functions}
to easily compute $t$-statistic and \pvalue in practice.

% TODO may want to switch to a similar figure in the Z-test section...
\begin{figure}
\centering
\includegraphics[width=0.7\textwidth]{figures/stats/one_side_t_test_rejection_regions.png}
\caption{
Illustration of a two-tail $t$-test's null hypothesis, $H_{0}$ rejection regions,
by \href{https://www.machinelearningplus.com/statistics/t-test-students-understanding-the-math-and-how-it-works/}{Selva Prabhakaran}.
In a one-tail $t$-test we would only consider the tail on a single side.
$\alpha$ is the significance level, typically \num{0.05} or lower.
}
\label{fig:two_sided_t_test}
\end{figure}

%%%%%%%%%%%%%%%%%%%%%%%%%%%%%%%%%%%%%%%%%%%%%%%%%%%%%%%%
\subsection{One-Sample}
\label{hypo:t_test:one}
In a one-sample $t$-test we have one sample of size $n$ with mean $\expval{x}$ and standard deviation $s$,
and test the null hypothesis that it belongs to a parent population with mean $\mu_{0}$.
In this case the parent population does not need to be normally distributed, but the distribution of possible $\expval{x}$ is assumed to be normal.

The $t$-statistic \cref{eq:hypo:t:one} can be used with $\nu = n-1$ to find a $\pvalue$.

\begin{equation}\label{eq:hypo:t:one}
t = \frac{\expval{x} - \mu_{0}}{s / \sqrt{n}}
\end{equation}

%%%%%%%%%%%%%%%%%%%%%%%%%%%%%%%%%%%%%%%%%%%%%%%%%%%%%%%%
\subsection{Two-Sample: Unpaired}
\label{hypo:t_test:two:unpaired}
In a two-sample unpaired $t$-test we have two independent samples, each of size $n$,
but with their own sample means $\expval{x_{i}}$ and standard deviations $s_{i}$.
Provided that we can assume that the two parent distributions of $x_{1}$ and $x_{2}$ have the same variance,
we can test the null hypothesis that the two parent distribution means are equal.

The $t$-statistic \cref{eq:hypo:t:two:unpaired} can be used with $\nu = 2n-2$ to find a $\pvalue$.

\begin{subequations}\label{eq:hypo:t:two:unpaired}
\begin{align}
t &= \frac{\expval{x_{1}} - \expval{x_{2}}}{s_{p} \sqrt{2/n}} \label{eq:hypo:t:two:unpaired:t} \\
s_{p} &= \sqrt{\left(s^{2}_{x_{1}} + s^{2}_{x_{2}}\right)/2} \label{eq:hypo:t:two:unpaired:s_p}
\end{align}
\end{subequations}

%%%%%%%%%%%%%%%%%%%%%%%%%%%%%%%%%%%%%%%%%%%%%%%%%%%%%%%%
\subsubsection{Different Sample Sizes}
\label{hypo:t_test:two:unpaired:diff_n}
If we relax the sample size condition and let $n_{1} \neq n_{2}$
we can still compute a $t$-statistic,
provided the parent distribution's variances are equal\footnote{A useful rule of thumb is $1/2 < s_{x_{1}} / s_{x_{2}} < 2$.}.

The $t$-statistic \cref{eq:hypo:t:two:unpaired:diff_n} can be used with $\nu = n_{1} + n_{2} - 2$ to find a $\pvalue$.

\begin{subequations}\label{eq:hypo:t:two:unpaired:diff_n}
\begin{align}
t &= \frac{\expval{x_{1}} - \expval{x_{2}}}{s_{p} \sqrt{\frac{1}{n_{1}} + \frac{1}{n_{2}}}} \label{eq:hypo:t:two:unpaired:diff_n:t} \\
s_{p} &= \sqrt{\left(\left(n_{1} - 1\right)s^{2}_{x_{1}} + \left(n_{2} - 1\right)s^{2}_{x_{2}}\right)/\left(n_{1} + n_{2} -2\right)} \label{eq:hypo:t:two:unpaired:diff_n:s_p}
\end{align}
\end{subequations}

%%%%%%%%%%%%%%%%%%%%%%%%%%%%%%%%%%%%%%%%%%%%%%%%%%%%%%%%
\subsubsection{Different Sample Sizes and Variances (Welch's \texorpdfstring{$t$}{t}-test)}
\label{hypo:t_test:two:unpaired:diff_n_diff_var}
If we further relax the assumptions and also let the population variances differ
we arrive at Welch's $t$-test which approximates\footnote{The
true distribution of $t$ depends somewhat on the unknown population variances,
see the \href{https://en.wikipedia.org/wiki/Behrens\%E2\%80\%93Fisher_problem}{Behrens--Fisher problem}
for more.} the $t$-distribution.

The $t$-statistic \cref{eq:hypo:t:two:unpaired:diff_n_diff_var:t} can be used with $\nu$ \cref{eq:hypo:t:two:unpaired:diff_n_diff_var:dof} to find a $\pvalue$.

\begin{subequations}\label{eq:hypo:t:two:unpaired:diff_n_diff_var}
\begin{align}
t &= \frac{\expval{x_{1}} - \expval{x_{2}}}{\sqrt{\frac{s^{2}_{1}}{n_{1}} + \frac{s^{2}_{2}}{n_{2}}}} \label{eq:hypo:t:two:unpaired:diff_n_diff_var:t} \\
\nu &= \frac{\left(s^{2}_{1}/n_{1} + s^{2}_{2}/n_{2}\right)^{2}}{\frac{\left(s^{2}_{1}/n_{1}\right)^{2}}{n_{1}-1} + \frac{\left(s^{2}_{2}/n_{2}\right)^{2}}{n_{2}-1}} \label{eq:hypo:t:two:unpaired:diff_n_diff_var:dof}
\end{align}
\end{subequations}

%%%%%%%%%%%%%%%%%%%%%%%%%%%%%%%%%%%%%%%%%%%%%%%%%%%%%%%%
\subsection{Two-Sample: Paired}
\label{hypo:t_test:two:paired}
In a two-sample paired $t$-test we have two dependent samples,
such as two sets of measurements from the same $n$ individuals taken at different times.
In this case, we are testing the null hypothesis that the difference in means of the two dependent samples is $\mu_{0}$.
Note, we can set $\mu_{0} = 0$ if we simply want to test for a statistically significant difference, and not an \apriori degree of difference.
Defining the difference between paired observations as $x_{\Delta}$, we compute the mean difference $\expval{x_{\Delta}}$ and $s_{\Delta}$ standard deviation on the sample.

The $t$-statistic \cref{eq:hypo:t:two:paired} can be used with $\nu = n-1$ to find a $\pvalue$.

\begin{equation}\label{eq:hypo:t:two:paired}
t = \frac{\expval{x_{\Delta}} - \mu_{0}}{s_{\Delta} / \sqrt{n}}
\end{equation}

%%%%%%%%%%%%%%%%%%%%%%%%%%%%%%%%%%%%%%%%%%%%%%%%%%%%%%%%
%%%%%%%%%%%%%%%%%%%%%%%%%%%%%%%%%%%%%%%%%%%%%%%%%%%%%%%%
\section{\texorpdfstring{$\chi^{2}$-Test}{Chi-Squared Test}}
\label{hypo:chi2_test}

Pearson's $\chi^{2}$-test can be used to
statistically compare a set of observations in
$n$ variables, $x_{i}$, to prior expectations via the $\chi^{2}$-distribution.
The \pvalue returned by the test estimates the probability of obtaining the observations
assuming the null hypothesis, \ie the expectations, is true.
The $\chi^{2}$-test statistic, $X^{2}$, is created with the assumption that
the data are normally distributed and independent,
which often is the case due to the central limit theorem (CLT).
It is constructed by squaring the difference\footnote{Yates's
correction for continuity $\left(x^{\text{obs}}_{j} - x^{\text{exp}}_{j}\right)^{2} \to \left(\abs{x^{\text{obs}}_{j} - x^{\text{exp}}_{j}}-0.5\right)^{2}$ may
also be applied in some low statistics cases.} between
an expected value, $x^{\text{exp}}_{i}$, and its corresponding observation, $x^{\text{obs}}_{i}$,
and dividing by the expectation:

\begin{equation}\label{eq:hypo:chi2_score}
X^{2} = \sum_{i=1}^{n} \frac{\left(x^{\text{obs}}_{i} - x^{\text{exp}}_{i}\right)^{2}}{x^{\text{exp}}_{i}}\,.
\end{equation}

In the limit that each $x^{\text{obs}}_{i}$ is normally distributed and $n$ is large, $X^{2} \to \chi^{2}$.
We can then use the $\chi^{2}$ distribution with $\nu = n-1$ degrees of freedom to find the \pvalue as the area to the right of $X^{2}$.
An easy way to compute $X^{2}$ and the \pvalue is to use the \texttt{scipy.stats.chisquare(f\_obs, f\_exp)}
\href{https://docs.scipy.org/doc/scipy/reference/generated/scipy.stats.chisquare.html}{function}.

The $\chi^{2}$-test can also be used to test of the data's independence, or homogeneity,
for $m$ samples of $n$ variables with $\nu = \left(n-1\right)\left(m-1\right)$ and\footnote{Note this is really the same as \cref{eq:hypo:chi2_score} if we reindex, just with a different $\nu$.}

\begin{equation}\label{eq:hypo:chi2_score_ind}
X^{2} = \sum_{i=1}^{n} \sum_{j=1}^{m} \frac{\left(x^{\text{obs}}_{i,j} - x^{\text{exp}}_{i,j}\right)^{2}}{x^{\text{exp}}_{i,j}}\,.
\end{equation}

%%%%%%%%%%%%%%%%%%%%%%%%%%%%%%%%%%%%%%%%%%%%%%%%%%%%%%%%
%%%%%%%%%%%%%%%%%%%%%%%%%%%%%%%%%%%%%%%%%%%%%%%%%%%%%%%%
\section{Analysis of Variance (ANOVA)}
\label{hypo:ANOVA}
% TODO

%%%%%%%%%%%%%%%%%%%%%%%%%%%%%%%%%%%%%%%%%%%%%%%%%%%%%%%%
%%%%%%%%%%%%%%%%%%%%%%%%%%%%%%%%%%%%%%%%%%%%%%%%%%%%%%%%
\section{Binomial Proportion Test}
\label{hypo:binomial_test}
When dealing with samples of $n$ binary events we can perform hypothesis testing
on the number of observed positive events $k$
using test statistics built on the binomial distribution.

%%%%%%%%%%%%%%%%%%%%%%%%%%%%%%%%%%%%%%%%%%%%%%%%%%%%%%%%
\subsection{Exact Binomial Test}
\label{hypo:binomial_test:exact}
For small $n$ it is possible to compute the \pvalue
explicitly from the binomial distribution \cref{eq:stats:binomial}.
We test the null hypothesis that the probability of success is $\pi_{0}$
having actually observed $k$ successes, $k = n \pi$.

The \pvalue is then the sum

\begin{equation}\label{eq:hypo:binomial_test:exact}
\pvalue = \sum_{i \in \, \mathcal{I}} {n \choose i}\pi_{0}^{i} \left(1-\pi_{0}\right)^{n-i},
\end{equation}

\noindent where $\mathcal{I}$ depends on the type of test:

\begin{table}[H]
\centering
\begin{tabular}{l|l}
$\pi < \pi_{0}$ & $\mathcal{I} = \left\{0, 1, \ldots, k\right\}$, \\
$\pi > \pi_{0}$ & $\mathcal{I} = \left\{k, k+1, \ldots, n\right\}$, \\
$\pi \neq \pi_{0}$ & $\mathcal{I} = \left\{\forall \, i: P\left(x=i\right) \leq P\left(x = k\right)\right\}$, with binomial $P\left(x\right)$ \cref{eq:stats:binomial}.
\end{tabular}
\end{table}

%%%%%%%%%%%%%%%%%%%%%%%%%%%%%%%%%%%%%%%%%%%%%%%%%%%%%%%%
\subsection{One-Sample}
\label{hypo:binomial_test:one}
For large sample sizes the binomial distribution is approximated by the normal distribution
and we can use a form of the $Z$-test to produce {\pvalue}s.
We require the observations to be independent,
\ie we may only sample $< \SI{10}{\percent}$ of the parent population,
the sampling distribution of $\pi$ to be approximately normal,
and that there are $\geq 10$ successes and $\geq 10$ failures,
$n \pi_{0} \geq 10$ and $n \left(1-\pi_{0}\right) \geq 10$, \ie the success-failure condition.

The $Z$-score is then:

\begin{equation}\label{eq:hypo:binomial_test:one}
Z = \frac{\pi - \pi_{0}}{\sqrt{\pi_{0}\left(1-\pi_{0}\right)/n}} = \frac{k - n \pi_{0}}{\sqrt{n\pi_{0}\left(1-\pi_{0}\right)}}.
\end{equation}

Note that the one-sample test is provided by the
\texttt{scipy.stats.binomtest} \href{https://docs.scipy.org/doc/scipy/reference/generated/scipy.stats.binomtest.html}{function}.

%%%%%%%%%%%%%%%%%%%%%%%%%%%%%%%%%%%%%%%%%%%%%%%%%%%%%%%%
\subsection{Two-Sample}
\label{hypo:binomial_test:two}
In the case of two samples, we can test the null hypothesis
that the difference\footnote{Again, we set $\pi_{\Delta} = 0$ if we want to test for any difference.} in the sample's probabilities is $\pi_{\Delta}$.
We require that the $\pi$ from the two samples are uncorrelated,
have approximately normal sampling distributions,
and that their difference $\pi_{1} - \pi_{2}$ is an unbiased estimator.

The $Z$-score is then:

\begin{subequations}\label{eq:hypo:binomial_test:two}
\begin{align}
Z &= \frac{\pi_{1} - \pi_{2} - \pi_{\Delta}}{\pi_{p} \left(1-\pi_{p}\right)\left(1/n_{1} + 1/n_{2}\right)} \label{eq:hypo:binomial_test:two:Z} \\
\pi_{p} &= \frac{k_{1} + k_{2}}{n_{1} + n_{2}} \label{eq:hypo:binomial_test:two:pi_p}
\end{align}
\end{subequations}

%%%%%%%%%%%%%%%%%%%%%%%%%%%%%%%%%%%%%%%%%%%%%%%%%%%%%%%%
%%%%%%%%%%%%%%%%%%%%%%%%%%%%%%%%%%%%%%%%%%%%%%%%%%%%%%%%
\section{Mann-Whitney U Test}
\label{hypo:mann_whitney_U_test}
% TODO

%%%%%%%%%%%%%%%%%%%%%%%%%%%%%%%%%%%%%%%%%%%%%%%%%%%%%%%%
%%%%%%%%%%%%%%%%%%%%%%%%%%%%%%%%%%%%%%%%%%%%%%%%%%%%%%%%
\section{Kolmogorov-Smirnov Test}
\label{hypo:KS_test}
% TODO

%%%%%%%%%%%%%%%%%%%%%%%%%%%%%%%%%%%%%%%%%%%%%%%%%%%%%%%%
%%%%%%%%%%%%%%%%%%%%%%%%%%%%%%%%%%%%%%%%%%%%%%%%%%%%%%%%
\section{Hypothesis Test Error Types and Power Analysis}
\label{hypo:power}

In hypothesis testing, like binary classification, we can suffer from two types of errors;
Type I or false positives, and Type II or false negatives.
The probabilities of these errors are functions of the experimental design
and are important to understand before undertaking a study.
We label $\alpha$ as the probability of rejecting a true null hypothesis, \ie a type I error,
and $\beta$ as the probability of failing to reject a false null hypothesis, \ie a type II error.
See \cref{table:CM} for a graphical representation in the context of binary classification.

$\alpha$ is the easier parameter to understand and improve,
as it is just the \pvalue threshold we select before the study.
It is typical to use $\alpha \leq \num{0.05}$.
$\beta$ depends on many factors including
$\alpha$,
the magnitude of the underlying effect,
the measurement variance,
the model being utilized,
and the sample size $n$.
Instead of using $\beta$ directly we often talk about the statistical power of a hypothesis test, $1-\beta$,
\ie the probability of correctly rejecting a false null hypothesis.
$\num{0.8} < 1-\beta$ is a commonly used target for the power.
As experimenters, we can
try to improve our methods to reduce the measurement variance,
select a more appropriate model\footnote{Parametric
models tend to have higher powers than the equivalent non-parametric model.
% https://youtu.be/diRX_NesFkA?t=558
In particular, comparing
the Mann-Whitney U and unpaired $t$-test gives $\frac{\left(1-\beta\right)_{\text{MWU}}}{\left(1-\beta\right)_{t}} = \frac{3}{\pi} = \num{0.955}$,
%while comparing the Wilcoxon signed-rank and paired $t$-test gives $\frac{\left(1-\beta\right)_{\text{WSR}}}{\left(1-\beta\right)_{t}} = \frac{2}{\pi} = \num{0.6437}$, % TODo add if I cover the Wilcoxon signed-rank test in the future
for large $n$.},
or begrudgingly accept some combination of a larger $\alpha$ or larger minimal detectable difference.
However, the primary lever for improving an experiment's power is by increasing $n$,
at the cost of additional time and money to complete the study.

%%%%%%%%%%%%%%%%%%%%%%%%%%%%%%%%%%%%%%%%%%%%%%%%%%%%%%%%
\subsection{\texorpdfstring{$Z$}{Z}-Test Power Example}
\label{hypo:power:Z_example}

Calculating $\beta$ can be challenging and is commonly done in software\footnote{The \texttt{statsmodels.stats.power}
\href{https://www.statsmodels.org/dev/stats.html?highlight=statsmodels\%20stats\%20power\#power-and-sample-size-calculations}{module}
provides many useful functions, see \href{https://machinelearningmastery.com/statistical-power-and-power-analysis-in-python/}{here} for one example implementation.} particular to the model being utilized.
For a simple example, we can consider a $Z$-test of a null hypothesis $\mu_{0}$
and compute the power of the test at a specific value of the alternative hypothesis $\mu_{a}$, with $\mu_{0} < \mu_{a}$.
For the given $\alpha$ being used we can look up the corresponding critical $Z$-score, $Z_{\alpha}$.
Then assuming a standard deviation of $s_{\text{est}}$ from prior work,
and knowing $n$, we can estimate the sample mean
which would put us at the critical $Z$-score, $\expval{x_{c}}$ \cref{eq:hypo:power_ex:x_c}.
We then calculate the $Z$-score again, this time assuming the alternative hypothesis $\mu_{a}$ is true
and we have observed $\expval{x_{c}}$ from our sample, \ie we are right at the edge of rejecting a false null hypothesis.
This $Z$-score, $Z_{a}$ \cref{eq:hypo:power_ex:Z_a}, can finally be used to find the power $1-\beta = P\left(Z_{a} < Z\right)$.

\begin{subequations}\label{eq:hypo:power_ex}
\begin{align}
Z_{\alpha} &= \frac{\expval{x_{c}} - \mu_{0}}{s_{\text{est}} / \sqrt{n}} \implies
\expval{x_{c}} = \frac{Z_{\alpha} s_{\text{est}}}{\sqrt{n}} + \mu_{0} \label{eq:hypo:power_ex:x_c} \\
Z_{a} &= \frac{\expval{x_{c}} - \mu_{a}}{s_{\text{est}} / \sqrt{n}}
= Z_{\alpha} + \sqrt{n}\,\frac{\mu_{0} - \mu_{a}}{s_{\text{est}}} \label{eq:hypo:power_ex:Z_a}
\end{align}
\end{subequations}

Note that as advertised $Z_{a}$ depends on
the choice of $\alpha$ via $Z_{\alpha}$,
the magnitude of the underlying effect $\mu_{0} - \mu_{a}$,
the measurement variance $s_{\text{est}}$,
the sample size $n$,
and is particular to this hypothesis test.
Plugging in numbers, if
$\alpha = \num{0.05} \to Z_{\alpha} = \num{1.645}$,
$\mu_{0} = \num{10}$,
$s_{\text{est}} \approx \num{2}$,
$n = \num{100}$,
and we want to find the power of the test for an alternative hypothesis of $\mu_{a} = \num{10.5}$,
we have $Z_{a} = \num{-0.8551} \to P\left(\num{-0.8551} < Z\right) = \num{0.8038}$
and thus the power is an acceptable $1-\beta = \num{0.8038} = \SI{80.38}{\percent}$.
We could use a similar line of reasoning to estimate
the $n$ necessary to obtain a desired $\alpha$ and $\beta$ before running the experiment.

% import numpy as np
% import scipy.stats
% norm = scipy.stats.norm
% Z_a = norm.ppf(1-0.05) + np.sqrt(100)*(10-10.5)/2
% print(f'Z_a = {Z_a:.4f}')
% print(f'Power = 1-beta = {1-norm.cdf(Z_a):.4f}')

%%%%%%%%%%%%%%%%%%%%%%%%%%%%%%%%%%%%%%%%%%%%%%%%%%%%%%%%
\subsection{Lehr's Rule of Thumb for \texorpdfstring{$t$}{t}-Tests}
\label{hypo:power:rule_of_thumb}

As a rough rule of thumb for one-sided (two-sided) $t$-tests,
Lehr argues that to have a power of $1-\beta \sim \num{0.8}$ with $\alpha = \num{0.05}$,
$n$ \cref{eq:hypo:power_rule_of_thumb} should be set to 8\footnote{This
factor is $8 \approx \left(Z_{\alpha/2} + Z_{\beta}\right)^{2}$, for $1-\beta = \num{0.8}$, $\alpha = \num{0.05}$.} (16)
times the ratio of the estimated population variance, $s^{2}$,
and the desired detectable difference squared, $\Delta^{2} = \left(\mu_{1} - \mu_{2}\right)^{2}$.
Note that we can also rearrange this rule of thumb to estimate $\Delta^{2}$ given a particular $n$.

\begin{subequations}\label{eq:hypo:power_rule_of_thumb}
\begin{align}
n &\approx \hphantom{1}8 \frac{s^{2}}{\Delta^{2}}\quad \left(\text{One-Sided}\right), \label{eq:hypo:power_rule_of_thumb:one} \\
n &\approx 16 \frac{s^{2}}{\Delta^{2}}\quad \left(\text{Two-Sided}\right). \label{eq:hypo:power_rule_of_thumb:two}
\end{align}
\end{subequations}

%%%%%%%%%%%%%%%%%%%%%%%%%%%%%%%%%%%%%%%%%%%%%%%%%%%%%%%%
%%%%%%%%%%%%%%%%%%%%%%%%%%%%%%%%%%%%%%%%%%%%%%%%%%%%%%%%
\section{Bonferroni Correction}
\label{hypo:bonferroni_correction}
When conducting multiple hypothesis tests on the same set of data
we run the risk of underreporting $\alpha$ for the whole analysis.
In particle physics this is known as the look elsewhere effect\footnote{See the discussion in Appendix C of the dissertation \cite{mepland_dissertation}.}
\cite{Demortier:2007zz,lyons2008,Gross2010,Ranucci:2012ed}.
For example, if we set $\alpha = \num{0.05}$ for any individual test
on a set of data with many features, but then run $\num{20}$
tests on it, by chance we'd expect $\approx \num{1}$ tests
to erroneously reject a true null hypothesis.
To quantify this concept, we can construct
the family-wise $\alpha$ across all of the $N$ tests done on a dataset\footnote{There is
disagreement on the best way to treat $\alpha_{\text{FW}}$,
\eg are we even talking about the right null hypothises \cite{Perneger1236},
what if the different tests are use correlated variables -- then they are not wholly independent tests in the context of $\alpha_{\text{FW}}$.
The $\alpha_{\text{FW}}$ of \cref{eq:hypo:alpha_fw} is just one simple definition.},
$\alpha_{\text{FW}}$ \cref{eq:hypo:alpha_fw}.
In our earlier example, we would have $\alpha_{\text{FW}} = \num{0.642}$,
or a \SI{64.2}{\percent} chance of at least one test rejecting the null hypothesis in error.

\begin{equation}\label{eq:hypo:alpha_fw}
\alpha_{\text{FW}} = 1 - \left(1 - \alpha\right)^{N}
\end{equation}

To address this issue we can apply the Bonferroni correction,
and simply divide our nominal $\alpha$ by $N$
before conducting the tests\footnote{Or equivalently multiply the observed {\pvalue}s by $N$.}.

%%%%%%%%%%%%%%%%%%%%%%%%%%%%%%%%%%%%%%%%%%%%%%%%%%%%%%%%
\subsection{Sequential Bonferroni Correction, \ie Holm--Bonferroni Correction}
\label{hypo:bonferroni_correction:sequential}
We can control $\alpha_{\text{FW}}$ more efficiently in terms of the cost imposed on $\alpha$, and hence decreased power,
by using the sequential Bonferroni correction, \ie the Holm--Bonferroni correction.
As the name suggests, in the sequential correction
we iterate through the $i \in \left\{0, 1, \ldots, N-1\right\}$ tests
in order of their {\pvalue}s, from smallest to largest,
checking that $i$th test's \pvalue is $< \alpha / \left(N-i\right)$.
When we come to the first $i+1$ test with a $\pvalue \geq \alpha / \left(N-\left(i+1\right)\right)$
we stop iterating and say the previous $i$ tests reject the null hypothesis,
while the remaining $N-i$ tests fail to reject the null hypothesis.
In this way we can constrain $\alpha_{\text{FW}} \leq \alpha$,
while checking most tests against a less stringent condition
than $\alpha / N$ of the normal Bonferroni correction.
}
\chapter{Regression}
\label{chap:regression}

%%%%%%%%%%%%%%%%%%%%%%%%%%%%%%%%%%%%%%%%%%%%%%%%%%%%%%%%
\section{Linear Regression}
\label{regression:linear}

Linear regression fits the best hyperplane, or line in 1D,
to a collection of $m$ points $\mathbf{x}_{i}, y_{i}$,
typically via the method of least squares.
If $\mathbf{x}$ has $n$ features we can represent the
linear relationship between $\mathbf{x}$ and $y$ as:

\begin{equation}\label{eq:linear:onepoint}
y_{i} = \beta_{0} + \sum_{j=1}^{n}\, \beta_{j} x_{ij} + \epsilon_{i}\,,
\end{equation}

\noindent where $\beta_{j}$ are the parameters of the regression
and $\epsilon$ represent random errors.
Transitioning to matrix notation\footnote{Note
that \textit{linear} regression refers to the linearity in the model parameters
$\bm{\beta}$, not $\mathbf{X}$.
The components of $\mathbf{X}_{i}$ can be, and often are,
non-linear functions of other input features.}, this is simply:

\begin{equation}\label{eq:linear:matrix}
\mathbf{y} = \mathbf{X} \bm{\beta} + \bm{\epsilon}\,,
\end{equation}

\noindent where we have set $X_{i0} =1$.

The ordinary least squares (OLS) estimate of the parameters $\hat{\bm{\beta}}$
can be found by minimizing the squares of the residuals,
\ie the objective function $S\left(\bm{\beta}\right)$:

\begin{subequations} \label{eq:linear:ols}
\begin{align}
\hat{\bm{\beta}} &= \argmin_{\bm{\beta}} S\left(\bm{\beta}\right)\,, \label{eq:linear:argmin} \\
S\left(\bm{\beta}\right) &= \sum_{i=1}^{m} \, \abs{y_{i} - \sum_{j=0}^{n} \, \beta_{j} x_{ij}}^{2} = \norm{\mathbf{y} - \mathbf{X} \bm{\beta}}^{2}\,. \label{eq:linear:S}
\end{align}
\end{subequations}

\noindent The optimal $\hat{\bm{\beta}}$ of \cref{eq:linear:ols} has a closed form solution:

\begin{equation}\label{eq:linear:betahat}
\hat{\bm{\beta}} = \left(\mathbf{X}\transpose\mathbf{X}\right)^{-1}\mathbf{X}\transpose \mathbf{y}\,,
\end{equation}

\noindent provided the following assumptions hold:

\begin{enumerate}[noitemsep]
\item The underlying relationship between $\mathbf{x}$ and $y$ is linear.
\item The columns of $\mathbf{X}$ are linearly independent, \ie $\mathrm{rank}\left(\mathbf{X}\right) = n$ (no multicollinearity).
\item The errors $\epsilon$ have conditional mean 0, $E\left(\epsilon \mid \mathbf{X}\right) = 0$ (exogeneity). The errors thus:
\begin{enumerate}[noitemsep]
\item Have a mean of zero, $E\left(\epsilon\right) = 0$.
\item Are not correlated with the input features, $E\left(\mathbf{X}\transpose\epsilon\right) = 0$.
\end{enumerate}
\item The errors are spherical, $\mathrm{var}\left(\epsilon \mid \mathbf{X}\right) = \sigma^{2} \identity$. Thus:
\begin{enumerate}[noitemsep]
\item Each observation $\mathbf{x}_{i}$ has the same variance $\sigma^{2}$ (homoscedasticity).
\item The errors are uncorrelated between observations, $E\left(\epsilon_{i}\epsilon_{j \neq i} \mid \mathbf{X}\right) = 0$ (no autocorrelation).
\end{enumerate}
\item The errors are normally distributed (multivariate normality)\footnote{This is not strictly required, but if true the OLS is the MLE and hypothesis testing works.}.
\end{enumerate}
% TODO what goes wrong exactly when each assumption is violated

% TODO weighted version
% TODO R2, reduced chi square values, F-test, t-test
% TODO Ridge regression
% TODO Lasso regression

%%%%%%%%%%%%%%%%%%%%%%%%%%%%%%%%%%%%%%%%%%%%%%%%%%%%%%%%
\section{Logistic Regression}
\label{regression:logistic}

Logistic regression is a simple method to create a classifier,
typically on two classes $y = 0,1$, though multinomial extensions exist.
Its name comes from the use of the logit, or log-odds, function

\begin{equation}\label{eq:logistic:logic}
l = \text{logit}\left(p\right) = \log\left(\frac{p}{1-p}\right)
\end{equation}

\noindent on the probability $p$ of class $1$.
$l$ is estimated linearly from $n$ input features $x_{j}$ with $n+1$ parameters $\beta_{j}$ as:

\begin{equation}\label{eq:logistic:logicBeta}
l = \beta_{0} + \sum_{j=1}^{n} \, \beta_{j}\,x_{j}\,.
\end{equation}

\noindent The probability $p$ is then

\begin{equation}\label{eq:logistic:p}
p = \frac{e^l}{e^l + 1} = \frac{1}{1+e^{-l}} = \text{logit}^{-1}\left(l\right)
\end{equation}

\noindent which can be turned into a predicted class through the choice of a suitable decision threshold.

The model parameters $\bm{\beta}$ are chosen by maximizing
the log of the likelihood $L$ \cref{eq:logistic:L} over $m$ known example points $\mathbf{x}_{i}, y_{i}$.
Note that $P\left(y \mid x\right)$ \cref{eq:logistic:Pr} is simply the Bernoulli distribution.
In practice the log-likelihood $\log\left(L\right)$ is maximized via gradient descent.
An example of logistic regression can be found in \cref{fig:logistic_regression_ex}.

\begin{subequations} \label{eq:logistic:L_Pr}
\begin{align}
L\left(\bm{\beta} \mid \mathbf{x}\right) &= \prod_{i=1}^{m} \, P\left(y_{i} \mid \mathbf{x}_{i}; \bm{\beta}\right) \label{eq:logistic:L} \\
P\left(y \mid \mathbf{x}\right) &= p^y\left(1-p\right)^{1-y}, \quad y \in \{0, 1\} \label{eq:logistic:Pr}
\end{align}
\end{subequations}

\begin{figure}
\centering
\includegraphics[width=0.8\textwidth]{figures/regression/Exam_pass_logistic_curve.jpeg}
\caption{
Example logistic regression curve on one input feature, by \href{https://en.wikipedia.org/wiki/File:Exam_pass_logistic_curve.jpeg}{Michaelg2015}.
}
\label{fig:logistic_regression_ex}
\end{figure}

Some assumptions of the logistic regression approach are:
\begin{enumerate}[noitemsep]
\item $y$ is either present or absent (dichotomous).
\item There are minimal correlations between the $x_{j}$ features (no multicollinearity).
\item There are no major outliers in the data.
\end{enumerate}

% TODO pseudo R2, Wald statistic

%%%%%%%%%%%%%%%%%%%%%%%%%%%%%%%%%%%%%%%%%%%%%%%%%%%%%%%%
\section{Principal Component Regression (PCR)}
\label{Regression:PCR}
% TODO

%%%%%%%%%%%%%%%%%%%%%%%%%%%%%%%%%%%%%%%%%%%%%%%%%%%%%%%%
\section{Gaussian Process Regression}
\label{Regression:kriging}
% TODO also known as kriging

}
%%%%%%%%%%%%%%%%%%%%%%%%%%%%%%%%%%%%%%%%%%%%%%%%%%%%%%%%
%%%%%%%%%%%%%%%%%%%%%%%%%%%%%%%%%%%%%%%%%%%%%%%%%%%%%%%%
%%%%%%%%%%%%%%%%%%%%%%%%%%%%%%%%%%%%%%%%%%%%%%%%%%%%%%%%
% \chapter{Machine Learning}
% \label{ml}

%%%%%%%%%%%%%%%%%%%%%%%%%%%%%%%%%%%%%%%%%%%%%%%%%%%%%%%%
%%%%%%%%%%%%%%%%%%%%%%%%%%%%%%%%%%%%%%%%%%%%%%%%%%%%%%%%
\section{General Concepts}
\label{ml:general}

%%%%%%%%%%%%%%%%%%%%%%%%%%%%%%%%%%%%%%%%%%%%%%%%%%%%%%%%
\subsection{Evaluating Performance}
\label{ml:general:eval}

\subsubsection{Confusion Marix}
\label{ml:general:eval:cm}

The confusion matrix is a simple table of the number of actual, or truth, class instances
versus the number of a model's predicted class instances.
A two class example is provided in \cref{table:CM}.
Multi-class confusion matrices are straight forward extensions,
with correctly classified instances appearing along the diagonal.

\begin{table}[H]
  \centering
  \begin{tabular}{c | c | c | c |}
  \multicolumn{2}{c}{} & \multicolumn{2}{c}{\textbf{Actual}} \\ \cline{3-4}
  \multicolumn{1}{c}{} & & Positive & Negative \\ \cline{2-4}
  \multirow{4}{*}{\rotatebox{90}{\textbf{Predicted}}} & \multirow{2}{*}{Positive} & \multirow{2}{*}{TP} & FP \\[-8pt]
   & & & (Type I) \\ \cline{2-4}
   & \multirow{2}{*}{Negative} & FN & \multirow{2}{*}{TN} \\[-8pt]
   & & (Type II) & \\ \cline{2-4}
  \end{tabular}
  \caption{Two class confusion matrix.}
  \label{table:CM}
\end{table}

\subsubsection{TPR \& TNR -- Sensitivity \& Specificity}
\label{ml:general:eval:TPR_TNR}
The true positive rate (TPR) and true negative rate (TNR) are
relatively straight forward to compute and understand, along with their complements,
the false negative rate (FNR) and false positive rate (FPR).

\begin{enumerate}[noitemsep]
\item True positive rate (TPR), \ie sensitivity, recall, hit rate.
\begin{equation} \label{eq:TPR}
\text{TPR} = \frac{\text{TP}}{\text{P}} = \frac{\text{TP}}{\text{TP}+\text{FN}} = 1 - \text{FNR} = P\left(\hat{+} \mid + \right)
\end{equation}

\item True negative rate (TNR), \ie specificity, selectivity.
\begin{equation} \label{eq:TNR}
\text{TNR} = \frac{\text{NP}}{\text{N}} = \frac{\text{TN}}{\text{TN}+\text{FP}} = 1 - \text{FPR} = P\left(\hat{-} \mid - \right)
\end{equation}
\end{enumerate}

% TODO precision vs recall

% TODO cite in text
\begin{figure}
\centering
  \begin{subfigure}[c]{0.48\textwidth}\centering
  \includegraphics[width=\textwidth]{figures/ml/precision_recall.pdf}
  \caption{Precision \& Recall}
  \label{fig:graphical_CM_quantities:precision_recall}
  \end{subfigure}
  ~
  \begin{subfigure}[c]{0.48\textwidth}\centering
  \includegraphics[width=\textwidth]{figures/ml/sensitivity_and_specificity.pdf}
  \caption{Sensitivity \& Specificity}
  \label{fig:graphical_CM_quantities:sensitivity_specificity}
  \end{subfigure}
\caption{
Graphical representation of
precision versus recall, by \href{https://commons.wikimedia.org/wiki/File:Precisionrecall.svg}{Walber},
and
sensitivity versus specificity, by \href{http://en.wikipedia.org/wiki/File:Sensitivity_and_specificity.svg}{FeanDoe}.
}
\label{fig:graphical_CM_quantities}
\end{figure}

% TODO other scores: F1, etc

\subsubsection{ROC Curves}
\label{ml:general:eval:ROC}
% TODO include small figure with TPR vs FPR (better in the upper left), and version with better in lower right

\subsubsection{Selecting a Decision Threshold}
\label{ml:general:eval:decision_threshold}

% include \cref to significance section, if eventually added

In physics we may try to maximize the significance $Z$ of a classifier\footnote{And or
work with the \href{https://en.wikipedia.org/wiki/Neyman\%E2\%80\%93Pearson\_lemma}{Neyman-Pearson framework}.} by
picking an optimal point along the ROC curve to set the decision threshold.
However in data science it is often better to create a payoff matrix of the anticipated
benefits associated with a TP or TN, and costs associated with a FP or FN,
for the particular business case at hand.
The expected value of any decision threshold can quickly be computed
from the payoff matrix elements, $E\left( \hat{A} \mid B\right)$, as

\begin{equation} \label{eq:E_profit}
E\left(\text{profit}\right) = \sum_{A,B} E\left( \hat{A} \mid B\right) P\left(\hat{A} \mid B \right) P\left(B\right),
\end{equation}

\noindent where $A$ and $B$ are any two cases.
The optimal decision threshold can then be found by maximizing $E\left(\text{profit}\right)$.

%%%%%%%%%%%%%%%%%%%%%%%%%%%%%%%%%%%%%%%%%%%%%%%%%%%%%%%%
\subsection{Bias-Variance Tradeoff}
\label{ml:general:biasVar}
% TODO add graphs too

\begin{enumerate}[noitemsep]
\item Bias: Errors due to a model not learning about relationships between features, \ie underfitting.
\item Variance: Errors due to an overly complex model failing to generalize beyond the training data, \ie overfitting.
\end{enumerate}

%%%%%%%%%%%%%%%%%%%%%%%%%%%%%%%%%%%%%%%%%%%%%%%%%%%%%%%%
\subsection{Gradient Decent}
\label{ml:general:gradDec}
% TODO

%%%%%%%%%%%%%%%%%%%%%%%%%%%%%%%%%%%%%%%%%%%%%%%%%%%%%%%%
\subsection{Normalization}
\label{ml:general:normalization}
% TODO normalization of input features (for faster training, more equal regularization), batch renormalization in neural networks

%%%%%%%%%%%%%%%%%%%%%%%%%%%%%%%%%%%%%%%%%%%%%%%%%%%%%%%%
\subsection{Regularization}
\label{ml:general:reg}
% TODO

\begin{subequations} \label{eq:L1_L2}
\begin{align}
\Omega_{\text{L1}}\left(\bm{\beta}\right) &= \lambda \norm{\bm{\beta}}\,,     \label{eq:L1} \\
\Omega_{\text{L2}}\left(\bm{\beta}\right) &= \lambda \norm{\bm{\beta}}^{2}\,. \label{eq:L2}
\end{align}
\end{subequations}

\subsubsection{L1 -- Lasso}
\label{ml:general:reg:L1}
% TODO taxi cab distance, many model parameters are reduced to 0 (sparsity) - built in feature selection

\subsubsection{L2 -- Ridge}
\label{ml:general:reg:L2}
% TODO computationally fast
% particularly useful when variance of data is high(?)

\subsubsection{Elastic Net}
\label{ml:general:reg:EN}

\begin{equation} \label{eq:elastic_net}
\Omega_{\text{EN}}\left(\bm{\beta}\right) = \lambda_{1} \norm{\bm{\beta}} + \lambda_{2} \norm{\bm{\beta}}^{2}
\end{equation}

% TODO connections to SVM, use cases

%%%%%%%%%%%%%%%%%%%%%%%%%%%%%%%%%%%%%%%%%%%%%%%%%%%%%%%%
%%%%%%%%%%%%%%%%%%%%%%%%%%%%%%%%%%%%%%%%%%%%%%%%%%%%%%%%
%%%%%%%%%%%%%%%%%%%%%%%%%%%%%%%%%%%%%%%%%%%%%%%%%%%%%%%%
\chapter{Unsupervised Learning}
\label{ml:unsupervised}

%%%%%%%%%%%%%%%%%%%%%%%%%%%%%%%%%%%%%%%%%%%%%%%%%%%%%%%%
%%%%%%%%%%%%%%%%%%%%%%%%%%%%%%%%%%%%%%%%%%%%%%%%%%%%%%%%
\section{\texorpdfstring{$k$}{k}-Means}
\label{ml:unsupervised:kMean}
% TODO

%%%%%%%%%%%%%%%%%%%%%%%%%%%%%%%%%%%%%%%%%%%%%%%%%%%%%%%%
\subsection{Metrics}
\label{ml:unsupervised:kMean:metrics}
% TODO

\subsubsection{Cartesian Radius}
\label{ml:unsupervised:kMean:metrics:cartesian}
% TODO

% \begin{equation}\label{eq:cartesian}
% \end{equation}

\subsubsection{Cosine Simularity}
\label{ml:unsupervised:kMean:metrics:cos}
% TODO

% \begin{equation}\label{eq:cos}
% \end{equation}


%%%%%%%%%%%%%%%%%%%%%%%%%%%%%%%%%%%%%%%%%%%%%%%%%%%%%%%%
%%%%%%%%%%%%%%%%%%%%%%%%%%%%%%%%%%%%%%%%%%%%%%%%%%%%%%%%
\section{Supervised Learning}
\label{ml:supervised}

In supervised learning a model is trained over many known examples
to use input features $\mathbf{X}$ to make a prediction \yhat about the true value $y$.
During the training process parameters $\beta$ of
the model are adjusted to minimize a two-part objective function,
$\text{obj}\left(\beta\right) = L\left(\beta\right) + \Omega\left(\beta\right)$.
The training loss $L\left(\beta\right)$ measures the model's predictive performance
while $\Omega\left(\beta\right)$ is a regularization term to penalize model complexity.
Note that $L$ is a measure of the model's bias and $\Omega$ is a measure of its variance,
so $\text{obj}\left(\beta\right)$ captures both parts of the bias-variance tradeoff \cite{HastieTF09}.

%%%%%%%%%%%%%%%%%%%%%%%%%%%%%%%%%%%%%%%%%%%%%%%%%%%%%%%%
\subsection{(Gaussian) N{a\"i}ve Bayes Classification (GNB)}
\label{ml:supervised:Bayes}
% TODO
% TODO maximum a posteriori probability (MAP) estimator

%%%%%%%%%%%%%%%%%%%%%%%%%%%%%%%%%%%%%%%%%%%%%%%%%%%%%%%%
\subsection{\texorpdfstring{$k$}{k}-Nearest Neighbors (\texorpdfstring{$k$}{k}-NN)}
\label{ml:supervised:kNN}
% TODO
% TODO also talk about collaborative filtering

%%%%%%%%%%%%%%%%%%%%%%%%%%%%%%%%%%%%%%%%%%%%%%%%%%%%%%%%
\subsection{Support Vector Machines (SVM)}
\label{ml:supervised:SVM}
% TODO

%%%%%%%%%%%%%%%%%%%%%%%%%%%%%%%%%%%%%%%%%%%%%%%%%%%%%%%%
\subsection{Decision Trees, \texorpdfstring{\ie}{ie} Classification and Regression Trees (CART)}
\label{ml:supervised:CART}

A basic classifier can be created from a tree of selections on $\mathbf{X}$ designed to
separate the classes at each branch.
Such a model is known as a classification and regression tree (CART) \cite{Breiman:2253780}
and a simple example can be found in \cref{ml:supervised:CART:small_example_CART}.
As the splits are just selections on the input variables,
they are --- somewhat --- possible to understand,
and conveniently do not need any kind of feature scaling, unlike other methods.
To make a prediction for an event the tree and its branches are traversed
until the event lands in one of the weighted leaves.
The weight of the leaf $w$ is positive (negative) for signal-like (background-like) events.
A logistic function is used to properly transform $w$ into an output score
$\yhat = 1 /\left(1+e^{-w}\right)$ within $0 < \yhat < 1$.

\begin{figure}[H]
\centering
\includegraphics[width=0.4\textwidth]{figures/ml/tree7_g2000_n1200.pdf}
\caption{
Simple classification and regression tree (CART).
Signal-like (background-like) events receive positive (negative) weights in the leaves.
}
\label{ml:supervised:CART:small_example_CART}
\end{figure}

% TODO include gini impurity / importance

%%%%%%%%%%%%%%%%%%%%%%%%%%%%%%%%%%%%%%%%%%%%%%%%%%%%%%%%
\subsection{Boosted Decision Trees (BDT)}
\label{ml:supervised:BDT}

Individual CARTs are rather poor and limited models
in terms of the behaviors they can successfully predict.
However, by taking an ensemble of $K$ complementary trees, \ie boosting \cite{FREUND1997119,friedman2000},
and summing each CART's individual weight $w_{k}$ a much more flexible BDT\footnote{As the leaf weights
are reals rather than integer classes this approach may be better described as a boosted regression tree,
and can indeed handle regression problems without the logistic function.} is formed.
The component trees of a BDT are generated by iteratively adding new trees $f_{k}\left(x_{i}\right)$ to those which came before \cite{XGBoost},

\begin{equation} \label{eq:boosting}
\begin{aligned}
\hat{y}^{\left(0\right)} &= 0\,, \\
\hat{y}^{\left(1\right)} &= f_1\left(\mathbf{X}\right) = \hat{y}^{\left(0\right)} + f_1\left(\mathbf{X}\right), \\
\hat{y}^{\left(2\right)} &= f_1\left(\mathbf{X}\right) + f_2\left(\mathbf{X}\right)= \hat{y}^{\left(1\right)} + f_2\left(\mathbf{X}\right), \\
                           &\vdotswithin{\displaystyle =} \\
\hat{y}^{\left(t\right)} &= \sum_{k=1}^t f_k\left(\mathbf{X}\right)= \hat{y}^{\left(t-1\right)} + f_t\left(\mathbf{X}\right),
\end{aligned}
\end{equation}

\noindent where each tree $f_{k}$ is grown from zero branches while minimizing $\text{obj}\left(\theta\right)$.
Through the ingenious use of a second order Taylor expansion this process can
be recast as a form of gradient descent, and thus is known as
stochastic gradient boosting \cite{10.2307/2699986,FRIEDMAN2002367}.
The number of boosting rounds, and thus trees, $K$ can be chosen in advance
but is better optimized during the training process via early stopping.

\subsubsection{\xgboost}% would rather have the \textsc caps than italics
\label{ml:supervised:BDT:xgboost}
% TODO see https://towardsdatascience.com/boosting-algorithm-xgboost-4d9ec0207d

The \xgboost\footnote{\xgboost: eXtreme Gradient Boosting, \href{https://github.com/dmlc/xgboost}{github.com/dmlc/xgboost}.} library \cite{XGBoost}
is a modern open source implementation of gradient boosted decision tree methods.
Through various algorithmic and memory optimizations \xgboost demonstrates good performance\footnote{\xgboost has lost
its lead in recent years to newer libraries such as LightGBM \cite{LightGBM}
and CatBoost \cite{CatBoost}.}.
L1 and L2 regularization is incorporated via

\begin{equation}
\Omega\left(f\right) = \alpha T + \frac{1}{2}\lambda \sum_{j=1}^T w_j^2\,,
\end{equation}

\noindent where $T$ is the number of leaves in a tree and $w_{j}$ are the leaf weights;
however, the default hyperparameters $\alpha=0$ and $\lambda=1$ only enable L2 regularization.
Other important hyperparameters in \xgboost include the
learning rate $\eta$, which scales the corrections added by each new tree,
maximum tree depth, which sets a limit on the complexity of any tree via its depth,
and the early stopping validation threshold.
For reference $\eta=0.3$ and a maximum depth of 6 are the default values.

\subsubsection{AdaBoost}
\label{ml:supervised:BDT:AdaBoost}
% TODO

%%%%%%%%%%%%%%%%%%%%%%%%%%%%%%%%%%%%%%%%%%%%%%%%%%%%%%%%
\subsection{Random Forest}
\label{ml:supervised:RF}
% TODO

%%%%%%%%%%%%%%%%%%%%%%%%%%%%%%%%%%%%%%%%%%%%%%%%%%%%%%%%
\subsection{Artificial Neural Networks (NN)}
\label{ml:supervised:ANN}
% TODO

%%%%%%%%%%%%%%%%%%%%%%%%%%%%%%%%%%%%%%%%%%%%%%%%%%%%%%%%
\subsection{Recursive Neural Networks (RNN)}
\label{ml:supervised:RNN}
% TODO

\subsubsection{Long Short Term Memory (LSTM)}
\label{ml:supervised:RNN:LSTM}
% TODO

%%%%%%%%%%%%%%%%%%%%%%%%%%%%%%%%%%%%%%%%%%%%%%%%%%%%%%%%
\subsection{Convolutional Neural Networks (CNN)}
\label{ml:supervised:CNN}
% TODO

}
\chapter{Miscellaneous}
\label{chap:misc}

%%%%%%%%%%%%%%%%%%%%%%%%%%%%%%%%%%%%%%%%%%%%%%%%%%%%%%%%
\section{Dimensionality Reduction}
\label{misc:m_reduction}
% TODO

%%%%%%%%%%%%%%%%%%%%%%%%%%%%%%%%%%%%%%%%%%%%%%%%%%%%%%%%
\section{Factor Analysis}
\label{misc:factor_ana}
% TODO

}

%==============================================================================

%-----------------------------------------------------------------------------%
% APPENDICES -- OPTIONAL. These are just chapters enumerated by Appendix A, Appendix B, Appendix C...
%-----------------------------------------------------------------------------%
% Start each appendix tex file with '\chapter{Title}'
\appendix
%%%%%%%%%%%%%%%%%%%%%%%%%%%%%%%%%%%%%%%%%%%%%%%%%%%%%%%%
%%%%%%%%%%%%%%%%%%%%%%%%%%%%%%%%%%%%%%%%%%%%%%%%%%%%%%%%
\chapter{\pandas}
\label{pandas}

%%%%%%%%%%%%%%%%%%%%%%%%%%%%%%%%%%%%%%%%%%%%%%%%%%%%%%%%
%%%%%%%%%%%%%%%%%%%%%%%%%%%%%%%%%%%%%%%%%%%%%%%%%%%%%%%%
\section{Basic Commands}
\label{pandas:basic}

\begin{lstlisting}[language=Python]
# IO
df = pd.read_csv('file.csv', header=None)
df['col'] = df['col'].astype(int)
df.to_csv('out.csv')

# descriptive commands
df.describe(); df.columns; df.shape;

# aggregation commands
df.sum(); df.cumsum();
df.min(); df.max(); df.idxmin(); df.idxmax();
df.mean(); df.std(); df.median(); df.mode();

# extract all rows from one column
df_y = df.loc[:, ['y']]

# select by value
df_selection = df.loc[( ((df['x'] == x_value) & (df['y'] == y_value)) | (df['z'] > z_value))]

# select by value and assign new value
df.loc[(df['x'] == x_value), 'y'] = y_value

# select with query
df_selection = df.query('x > 0')

# select with isin
df_selection = df.isin({'x': [x_value1, x_value2], 'y': [y_value]})

# select row by index
series = df.iloc[0]

# apply an arbitrary function
def func(x, y):
  return x*np.sin(y)
df['z'] = np.vectorize(func)(df['x'], df['y'])

# iterate through rows (slow!)
for index, row in df.iterrows():
  x_value = row['x']

# construct from rows
rows_list = []
for nrow in range(nrows):
  rows_list.append({'x':x_value, 'y':y_value})
df = pd.DataFrame(rows_list)
df = df[['x', 'y']]

# rename columns
df = df.rename({'old': 'new'}, axis='columns')

# sort
df = df.sort_values(by=['x', 'y'], ascending=[True, False]).reset_index(drop=True)

# group by, while dropping new count column and duplicates
df = df.groupby(['x', 'y', 'z']).size().to_frame(name = 'count').reset_index().drop(['count'], axis=1).drop_duplicates()

# return duplicate rows
columns_to_check_for_duplicates = ['x', 'y']
df_duplicates = df[df.duplicated(subset=columns_to_check_for_duplicates, keep=False)]

# shuffling
df = df.sample(frac=1., replace=False, random_state=rnd_seed).reset_index(drop=True)

# drop columns
df = df.drop(['col_to_drop1', 'col_to_drop2'], axis=1)

# fill nans, for all columns and per column
df = df.fillna(0.0)
df = df.fillna(value={'x': x_nan_value, 'z': y_nan_value})
\end{lstlisting}

\clearpage

%%%%%%%%%%%%%%%%%%%%%%%%%%%%%%%%%%%%%%%%%%%%%%%%%%%%%%%%
%%%%%%%%%%%%%%%%%%%%%%%%%%%%%%%%%%%%%%%%%%%%%%%%%%%%%%%%
\section{Joining}
\label{pandas:join}

\noindent See the \href{https://pandas.pydata.org/pandas-docs/stable/user_guide/merging.html}{documentation} and this \href{http://chrisalbon.com/python/data_wrangling/pandas_join_merge_dataframe/}{useful guide}.

\begin{lstlisting}[language=Python]
df = pd.merge(df_l, df_r, left_on='id_left', right_on='id_right', how='left')
\end{lstlisting}

%%%%%%%%%%%%%%%%%%%%%%%%%%%%%%%%%%%%%%%%%%%%%%%%%%%%%%%%
%%%%%%%%%%%%%%%%%%%%%%%%%%%%%%%%%%%%%%%%%%%%%%%%%%%%%%%%
\section{Pivoting}
\label{pandas:pivoting}

\subsubsection{pivot}
\label{pandas:pivoting:pivot}

\noindent \href{http://pandas.pydata.org/pandas-docs/stable/reference/api/pandas.DataFrame.pivot.html}{\texttt{pivot} documentation}.

\begin{lstlisting}[language=Python]
DataFrame.pivot(index=None, columns=None, values=None)
\end{lstlisting}

\begin{figure}[H]
\centering
\includegraphics[width=0.85\textwidth]{figures/pandas/reshaping_pivot.png}
\caption{
Example \pandas \texttt{pivot} operation, from the package \href{http://pandas.pydata.org/pandas-docs/stable/user_guide/reshaping.html}{documentation}.
}
\label{fig:pandas:pivot}
\end{figure}

\subsubsection{pivot\_table}
\label{pandas:pivoting:pivot_table}

\noindent \href{https://pandas.pydata.org/pandas-docs/stable/reference/api/pandas.pivot_table.html}{\texttt{pivot\_table} documentation}.

\begin{lstlisting}[language=Python]
pandas.pivot_table(data, values=None, index=None, columns=None, aggfunc='mean', fill_value=None, margins=False, dropna=True, margins_name='All')
\end{lstlisting}

\begin{figure}[H]
\centering
\includegraphics[width=0.95\textwidth]{figures/pandas/pivot-table-datasheet.png}
\caption{
Example \pandas \texttt{pivot\_table} operation, by \href{http://pbpython.com/pandas-pivot-table-explained.html}{Chris Moffitt}.
}
\label{fig:pandas:pivot_table}
\end{figure}
}
%%%%%%%%%%%%%%%%%%%%%%%%%%%%%%%%%%%%%%%%%%%%%%%%%%%%%%%%
%%%%%%%%%%%%%%%%%%%%%%%%%%%%%%%%%%%%%%%%%%%%%%%%%%%%%%%%
%%%%%%%%%%%%%%%%%%%%%%%%%%%%%%%%%%%%%%%%%%%%%%%%%%%%%%%%
\chapter{\sql}
\label{sql}

%%%%%%%%%%%%%%%%%%%%%%%%%%%%%%%%%%%%%%%%%%%%%%%%%%%%%%%%
%%%%%%%%%%%%%%%%%%%%%%%%%%%%%%%%%%%%%%%%%%%%%%%%%%%%%%%%
\section{Introduction}
\label{sql:intro}

\sql, or Structured Query Language, is way to
communicate with a Relational Database Management System, RDBMS.
There data is stored as collection of tables with
at least one common column to allow relational operators
to join information across tables.
In some SQL implementations, such as MySQL, each table must have an unique primary key for each row.
When a column in one table relates to the primary key of another, it is known as a foreign key.
A SQL query returns information from the database as a result set, and may contain subqueries.

%%%%%%%%%%%%%%%%%%%%%%%%%%%%%%%%%%%%%%%%%%%%%%%%%%%%%%%%
%%%%%%%%%%%%%%%%%%%%%%%%%%%%%%%%%%%%%%%%%%%%%%%%%%%%%%%%
\section{Basic Commands}
\label{sql:basic}

\begin{lstlisting}[language=SQL]
-- create a new table, define columns & types
create table t (id int primary key
	, name varchar(20) not null, state char(2), dob date);

-- manually insert new rows
insert into t values (0, 'Matt', 'NJ', '1990-1-2');
insert into t (id, name, state, dob) values
                     (1, 'Jamie','NJ', '1990-3-4');
insert into t values (2,'Mary','IL', '1970-5-6')
	, (3,'Eddie','IL','2010-7-8'),(4,'Bob','ND','1980-9-10');

-- insert rows from another table
insert into t select * from other_t where zip = 11111;

-- update row
update t set name = 'Mar' where id = 2;

/* select with where, order by, limit
comparison ops: >, >=, =, <> or !=, between, like, in
logical ops: and, or, not */
select id,name from t where 0 < id order by name limit 10;
select id,name from t where id between 1 and 3;
select id,name from t where name like 'M%';--starts with M
select id,name from t where state like 'N_';-- N + 1 char
select id,name from t where (id in (0,3)) or (name='Bob');
select name,dob from t where 1985<year(dob) order by dob;

-- aggregation commands
select count(*) as n_rows from t;
select count(id) as n_rows from t; -- only counts rows with non-null id values
select count(distinct id) as n_distinct_ids from t;
select min(id) from t; -- avg, mode, sum, ...

-- select unique / distinct values
select distinct name from t;

-- when finding nulls, use is; can't use =, <>
select id, name from t where state is null;

-- group by, use having, not where
select state, count(*) as n_rows
from t
group by state
having 1 < n_rows
order by n_rows desc;

-- CTEs
with state_counts as (
	select count(distinct id) as n_distinct_ids, state
	from t group by state
)
select avg(n_distinct_ids)
from state_counts where state != 'NC';

-- case
select name
	, case
		when state = 'IL' then 'bears fan'
		when state = 'ND' then 'bison fan'
		else concat(state, ' (unknown fan)')
	end as fandom
from t;

-- fill nulls
select ifnull(state, 'Unknown') as state from t;

-- group by cube
with b as (
	-- taking distinct before the group by can help reduce some duplicated computation later in count(distinct id), but is not required
	select distinct id
		, ifnull(left(upper(trim(name)), 1), ' ') as first_letter
		-- must not have any nulls in the field to be group by cube(), or there will be multiple null rows in output
		, ifnull(state, 'Unknown') as state
	from t
)
-- format output
select ifnull(state, 'All') as "State"
	, first_letter as "First Letter"
	, n_patients as "# Patients"
from (
	select state, first_letter
		, count(distinct id) as n_patients
	from b
	-- can mix regular group by and cube
	group by cube (state), first_letter
)
order by "State", "First Letter";

-- group by, get mode of state per person - deterministically (alphabetical order)
-- can be expanded to additional columns, each with their own g_state CTEs
with p as ( select distinct patient from p0 ),
g_state as (
	-- note count(state) will ignore null values in state
	select patient, state, count(state) as c
	from p0 group by patient, state
	-- qualify is the where statement for windows functions
	-- can use rank() instead of row_number() to allow ties
	qualify row_number() over (partition by patient order by c desc, state asc) = 1
)
select p.patient, state
from p
left join g_state on p.patient = g_state.patient;

-- working with arrays
with b0 as (
	select id, state
		, array_construct_compact(D1,D2,D3,D4,D5) as Dx_array
	from e
	where arrays_overlap(Dx_array, array_construct('W5602', 'W5551XD', 'W5803XA'))
),
b1 as (
	select id, state, Dx.value as Dx
	from b
	, lateral flatten(input => Dx_array, outer => True) as Dx
)
select state, Dx, count(distinct id) as n_patients
from b1
group by 1,2
order by 1,2;

-- return duplicate rows
select a.*, n_rows as n_duplicates
from t as a
inner join (
	select state, count(*) as n_rows
	from t
	group by state
	having 1 < n_rows
) as b
	on a.state = b.state;
\end{lstlisting}

%%%%%%%%%%%%%%%%%%%%%%%%%%%%%%%%%%%%%%%%%%%%%%%%%%%%%%%%
%%%%%%%%%%%%%%%%%%%%%%%%%%%%%%%%%%%%%%%%%%%%%%%%%%%%%%%%
\section{Joining}
\label{sql:join}

\begin{lstlisting}[language=SQL]
select column1, column2
from table1
left join table2
	on table1.column_nameA = table2.column_nameB;

-- coalesce appropriately when using a full outer join
select coalesce(a.id, b.id) as id
	, a.field as field_a
	, b.field as field_b
from a
full outer join b
	on a.id = b.id;
\end{lstlisting}

The three common join types are the standard
\texttt{inner join}, \texttt{left join},
and \texttt{full outer join}\footnote{Note that \texttt{full join} can also be used as an equivalent to \texttt{full outer join}.} as
shown in \cref{fig:sql:joins}.

\begin{figure}[H]
\centering
  \begin{subfigure}[c]{0.3\textwidth}\centering
  \includegraphics[width=\textwidth]{figures/sql/left_join}
  %\caption{}
  \label{fig:sql:joins:left_join}
  \end{subfigure}
  ~
  \begin{subfigure}[c]{0.3\textwidth}\centering
  \includegraphics[width=\textwidth]{figures/sql/inner_join}
  %\caption{}
  \label{fig:sql:joins:inner_join}
  \end{subfigure}
  ~
  \begin{subfigure}[c]{0.3\textwidth}\centering
  \includegraphics[width=\textwidth]{figures/sql/full_outer_join}
  %\caption{}
  \label{fig:sql:joins:full_outer_join}
  \end{subfigure}
\caption{
Illustration of common types of joins, adapted from \href{http://stevestedman.com/2015/03/sql-server-join-types-poster-version-2}{Steve Stedman}.
}
\label{fig:sql:joins}
\end{figure}

%%%%%%%%%%%%%%%%%%%%%%%%%%%%%%%%%%%%%%%%%%%%%%%%%%%%%%%%
%%%%%%%%%%%%%%%%%%%%%%%%%%%%%%%%%%%%%%%%%%%%%%%%%%%%%%%%
\section{Pivoting}
\label{ssql:pivoting}

\noindent \href{https://docs.snowflake.com/en/sql-reference/constructs/pivot.html}{\texttt{pivot} documentation}.

\begin{lstlisting}[language=SQL]
with b as (
	select left(name, 1) as first_letter, state
		, datediff(day, dob, current_date)/365 as age
	from t
)
select first_letter as "First Letter"
	, "'NJ'" as "Avg Age NJ"
	, "'IL'" as "Avg Age IL"
	, "'ND'" as "Avg Age ND"
from b
pivot(avg(age) for state in ('NJ','IL','ND'))
order by first_letter;
\end{lstlisting}

%%%%%%%%%%%%%%%%%%%%%%%%%%%%%%%%%%%%%%%%%%%%%%%%%%%%%%%%
%%%%%%%%%%%%%%%%%%%%%%%%%%%%%%%%%%%%%%%%%%%%%%%%%%%%%%%%
\section{IO Commands (MySQL)}
\label{sql:io}

\begin{lstlisting}[language=SQL]
-- admin, setup new user
grant all privileges on *.* to 'user'@'localhost'
	identified by 'pw';

-- create a new database
create database mydb; use mydb;

-- create a new schema
create schema myschema; use schema myschema;

-- load a SQL dump
set autocommit=0; source dump.sql; commit;

-- load csv file into table
load data local infile 'in.csv' into table tnew
	fields terminated by ',' lines terminated by '\n';

-- rename column, change type, add new column
alter table t rename column old_col_name to new_col_name;
alter table t modify col_name new_type;
alter table t add new_col double;

-- delete all rows of a table, keep structure
truncate table t;

-- delete a row, column, table, database, schema, ...
delete from t where id = 4;

show columns from t; /* or, also for MySQL */ describe t;
show columns from t where type like 'Varchar%';
alter table t drop column name;

show tables;
drop table if exists 't';

show databases;
drop database mydb;
\end{lstlisting}

\begin{lstlisting}[language=bash]
# export selection to csv (from shell, no file perms)
mysql -u user --password=pw --database=mydb
 --execute='select ...;' -q -n -B -r > out.csv
 && sed -i '/\t/ s//,/g' out.csv

# load table from csv (from shell)
mysqlimport --ignore-lines=1 --fields-terminated-by=,
 --verbose --local -u user -p mydb /path/to/in.csv
\end{lstlisting}
}
%%%%%%%%%%%%%%%%%%%%%%%%%%%%%%%%%%%%%%%%%%%%%%%%%%%%%%%%
%%%%%%%%%%%%%%%%%%%%%%%%%%%%%%%%%%%%%%%%%%%%%%%%%%%%%%%%
%%%%%%%%%%%%%%%%%%%%%%%%%%%%%%%%%%%%%%%%%%%%%%%%%%%%%%%%
\chapter{\pyspark}
\label{pyspark}

%%%%%%%%%%%%%%%%%%%%%%%%%%%%%%%%%%%%%%%%%%%%%%%%%%%%%%%%
%%%%%%%%%%%%%%%%%%%%%%%%%%%%%%%%%%%%%%%%%%%%%%%%%%%%%%%%
\section{Basic Commands}
\label{pyspark:basic}

\begin{lstlisting}[language=Python]
import pyspark.sql.functions as F
from pyspark.sql.window import Window
from pyspark.sql.types import IntegerType, DoubleType

# IO
df = spark.read.parquet('s3a://bucket/table')
df = df.withColumn('col', F.col('col').cast(IntegerType()))
df.write.save('s3a://bucket/table.parquet')

# descriptive commands
df.describe().show(); df.printSchema();
df.show(10); df.limit(10).toPandas();

# select columns
df_y = df.select('x', 'y')

# get distinct values
df_y_distinct = df.select('y').distinct()

# select by value
df_selection = df.where( ((F.col('x') == x_value) & (F.col('y') == y_value)) | (F.col('z') < z_value) )
df_selection = df.where( F.col('x').isNotNull() )

# select by value and assign new value
df = df.withColumn('y', F.when( F.col('x') == x_value, F.lit(y_value) ).otherwise(F.col('y')))

# select where isin, and not (~) isin, some_list
df_selection = df.where(F.col('x').isin(some_list))
df_selection = df.where(~F.col('x').isin(some_list))

# apply an arbitrary function - slow unless written in scala!
def func(x, y):
	return x*np.sin(y)
func_udf = F.udf(func, DoubleType())
df = df.withColumn('z', func_udf('x', 'y'))

# rename columns
df = df.withColumnRenamed('old', 'new')

# order by
df = df.orderBy(['x', 'y'], ascending=[True, False])

# group by, while counting rows and aggregating max date
df = df.groupBy('x', 'y', 'z').agg(F.count('*').alias('count'), F.max('date_col').alias('max_date'))

# group by, get mode of state per patient - deterministically (alphabetical order)
# can be expanded to additional columns, each with their own .join(df.groupBy()...) statements
df.select('patient').distinct().join(
df.where(F.col('state').isNotNull())
	.groupBy(['patient', 'state']).count().alias('c')
	.withColumn('row_num', F.row_number().over(Window().partitionBy('patient').orderBy(F.col('c').desc(), F.col('state'))))
	.where(F.col('row_num') == 1)
	.select('patient', 'state')
, 'patient', 'left')

# return duplicate rows
df.join(df, df.groupBy('x', 'y').agg(F.count('*').alias('c')).where(1 < F.('c')), ['x', 'y'], 'left_semi')

# drop columns
df = df.drop('col_to_drop1', 'col_to_drop2')

cols_to_drop = ['col1', 'col2']
df = df.drop(*cols_to_drop)

# fill nans, for all columns and per column - other syntaxes are also available
df = df.fillna(0.0)
df = df.fillna({'x': x_nan_value, 'z': y_nan_value})

# run a SQL query, note you must register the needed dataframes as tables first
df.registerTempTable('df')
spark.sql('select * from df limit 10').show(10)
\end{lstlisting}

%%%%%%%%%%%%%%%%%%%%%%%%%%%%%%%%%%%%%%%%%%%%%%%%%%%%%%%%
%%%%%%%%%%%%%%%%%%%%%%%%%%%%%%%%%%%%%%%%%%%%%%%%%%%%%%%%
\section{Joining}
\label{pyspark:join}

\noindent See the \href{http://spark.apache.org/docs/2.1.0/api/python/pyspark.sql.html?highlight=join#pyspark.sql.DataFrame.join}{documentation} and this \href{http://www.learnbymarketing.com/1100/pyspark-joins-by-example/}{useful guide}. In addition to the standard types of joins \texttt{left\_semi}, or \texttt{leftsemi}, is very useful for filtering the left table to the matching rows in the join condition, without actually joining any columns from the right table.

\begin{lstlisting}[language=Python]
df = df_l.join(df_r, df_l['id_left'] == df_r['id_right'], 'left')
\end{lstlisting}

%%%%%%%%%%%%%%%%%%%%%%%%%%%%%%%%%%%%%%%%%%%%%%%%%%%%%%%%
%%%%%%%%%%%%%%%%%%%%%%%%%%%%%%%%%%%%%%%%%%%%%%%%%%%%%%%%
\section{Pivoting}
\label{pyspark:pivoting}

\noindent \href{https://spark.apache.org/docs/2.1.0/api/python/pyspark.sql.html?highlight=pivot#pyspark.sql.GroupedData.pivot}{\texttt{pivot} documentation}, and \href{https://sparkbyexamples.com/pyspark/pyspark-pivot-and-unpivot-dataframe/}{examples}.

\begin{lstlisting}[language=Python]
df.groupBy('product').pivot('state').sum('cost')
\end{lstlisting}
}
%%%%%%%%%%%%%%%%%%%%%%%%%%%%%%%%%%%%%%%%%%%%%%%%%%%%%%%%
%%%%%%%%%%%%%%%%%%%%%%%%%%%%%%%%%%%%%%%%%%%%%%%%%%%%%%%%
%%%%%%%%%%%%%%%%%%%%%%%%%%%%%%%%%%%%%%%%%%%%%%%%%%%%%%%%
\chapter{Coding Concepts}
\label{coding}

%%%%%%%%%%%%%%%%%%%%%%%%%%%%%%%%%%%%%%%%%%%%%%%%%%%%%%%%
%%%%%%%%%%%%%%%%%%%%%%%%%%%%%%%%%%%%%%%%%%%%%%%%%%%%%%%%
\section{Sorting Algorithms}
\label{coding:sorts}

\begin{table}[H]
\centering
\begingroup
\renewcommand*{\arraystretch}{1}
\begin{tabular}{c c c c c c}
\hline
Name & Best & Average & Worst & Memory & Links \\
\hline
\hline
\multirow{2}{*}{Quicksort} & \multirow{2}{*}{$\order{n \log{n}}$} & \multirow{2}{*}{$\order{n \log{n}}$} & \multirow{2}{*}{$\order{n^{2}}$} & $\order{\log{n}}$, avg & \multirow{2}{*}{\href{https://youtu.be/XE4VP_8Y0BU}{Computerphile}, \href{https://youtu.be/SLauY6PpjW4}{HR}} \\
 & & & & $\order{n}$, worst & \\
Merge Sort & $\order{n \log{n}}$ & $\order{n \log{n}}$ & $\order{n \log{n}}$ & $\order{n}$ & \href{https://youtu.be/kgBjXUE_Nwc}{Computerphile}, \href{https://youtu.be/KF2j-9iSf4Q}{HR} \\
Bubble Sort & $\order{n}$ & $\order{n^{2}}$ & $\order{n^{2}}$ & $\order{1}$ & \href{https://youtu.be/kgBjXUE_Nwc}{Computerphile}, \href{https://youtu.be/6Gv8vg0kcHc}{HR} \\
Insertion Sort & $\order{n}$ & $\order{n^{2}}$ & $\order{n^{2}}$ & $\order{1}$ & \href{https://youtu.be/pcJHkWwjNl4}{Computerphile} \\
Bogosort & $\order{n}$ & $\order{\left(n+1\right)!}$ & $\infty$ & $\order{1}$ & --- \\
\hline
\end{tabular}

% TODO add Heapsort, Shellsort, Tree Sort

\endgroup
\caption{
A collection of sorting algorithms with time complexities.
}
\label{tab:sorting_table}
\end{table}

% TODO add heapsort

%%%%%%%%%%%%%%%%%%%%%%%%%%%%%%%%%%%%%%%%%%%%%%%%%%%%%%%%
%%%%%%%%%%%%%%%%%%%%%%%%%%%%%%%%%%%%%%%%%%%%%%%%%%%%%%%%
\section{Other}
\label{coding:other}

%%%%%%%%%%%%%%%%%%%%%%%%%%%%%%%%%%%%%%%%%%%%%%%%%%%%%%%%
\subsection{Binary Tree Search}
\label{coding:other:binary_tree_search}
% TODO
% TODO should probably be moved under BFS or DFS
% https://youtu.be/P3YID7liBug

%%%%%%%%%%%%%%%%%%%%%%%%%%%%%%%%%%%%%%%%%%%%%%%%%%%%%%%%
\subsection{Breadth First Search (BFS)}
\label{coding:other:bfs}
% TODO

\subsubsection{Dijkstra's Algorithm}
\label{coding:other:bfs:dijkstra}
% TODO
% https://youtu.be/GazC3A4OQTE

%%%%%%%%%%%%%%%%%%%%%%%%%%%%%%%%%%%%%%%%%%%%%%%%%%%%%%%%
\subsection{Depth First Search (DFS)}
\label{coding:other:dfs}
% TODO

%%%%%%%%%%%%%%%%%%%%%%%%%%%%%%%%%%%%%%%%%%%%%%%%%%%%%%%%
\subsection{Linked Lists}
\label{coding:other:lls}
% TODO

}
%%%%%%%%%%%%%%%%%%%%%%%%%%%%%%%%%%%%%%%%%%%%%%%%%%%%%%%%
%%%%%%%%%%%%%%%%%%%%%%%%%%%%%%%%%%%%%%%%%%%%%%%%%%%%%%%%
\chapter{Additional Concepts}
\label{additional}

While interesting, the concepts covered here are less likely
to be relevant during the interview process.

%%%%%%%%%%%%%%%%%%%%%%%%%%%%%%%%%%%%%%%%%%%%%%%%%%%%%%%%
%%%%%%%%%%%%%%%%%%%%%%%%%%%%%%%%%%%%%%%%%%%%%%%%%%%%%%%%
\section{Statistics}
\label{additional:stats}

%%%%%%%%%%%%%%%%%%%%%%%%%%%%%%%%%%%%%%%%%%%%%%%%%%%%%%%%
\subsection{Analysis of Variance (ANOVA)}
\label{additional:stats:ANOVA}
% TODO

%%%%%%%%%%%%%%%%%%%%%%%%%%%%%%%%%%%%%%%%%%%%%%%%%%%%%%%%
\subsection{Time Series Analysis}
\label{additional:stats:time_series_ana}
% TODO

%%%%%%%%%%%%%%%%%%%%%%%%%%%%%%%%%%%%%%%%%%%%%%%%%%%%%%%%
\subsection{Kolmogorov-Smirnov Test}
\label{additional:misc:KS_test}
% TODO

%%%%%%%%%%%%%%%%%%%%%%%%%%%%%%%%%%%%%%%%%%%%%%%%%%%%%%%%
\subsection{Kalman Filters}
\label{additional:misc:kalman_filters}
% TODO

%%%%%%%%%%%%%%%%%%%%%%%%%%%%%%%%%%%%%%%%%%%%%%%%%%%%%%%%
\subsection{Markov Chains}
\label{additional:misc:markov_chains}
% TODO

%%%%%%%%%%%%%%%%%%%%%%%%%%%%%%%%%%%%%%%%%%%%%%%%%%%%%%%%
%%%%%%%%%%%%%%%%%%%%%%%%%%%%%%%%%%%%%%%%%%%%%%%%%%%%%%%%
\section{Regression}
\label{additional:Regression}

%%%%%%%%%%%%%%%%%%%%%%%%%%%%%%%%%%%%%%%%%%%%%%%%%%%%%%%%
\subsection{Principal Component Regression (PCR)}
\label{additional:Regression:PCR}
% TODO

%%%%%%%%%%%%%%%%%%%%%%%%%%%%%%%%%%%%%%%%%%%%%%%%%%%%%%%%
\subsection{Generalized Linear Models (GLM)}
\label{additional:Regression:GLM}
% TODO

%%%%%%%%%%%%%%%%%%%%%%%%%%%%%%%%%%%%%%%%%%%%%%%%%%%%%%%%
\subsection{Poisson Regression}
\label{additional:Regression:poisson}
% TODO

%%%%%%%%%%%%%%%%%%%%%%%%%%%%%%%%%%%%%%%%%%%%%%%%%%%%%%%%
%%%%%%%%%%%%%%%%%%%%%%%%%%%%%%%%%%%%%%%%%%%%%%%%%%%%%%%%
\section{Unsupervised Learning}
\label{additional:unsupervised}

%%%%%%%%%%%%%%%%%%%%%%%%%%%%%%%%%%%%%%%%%%%%%%%%%%%%%%%%
\subsection{Gaussian Mixture Model (GMM)}
\label{additional:unsupervised:GMM}
% TODO

%%%%%%%%%%%%%%%%%%%%%%%%%%%%%%%%%%%%%%%%%%%%%%%%%%%%%%%%
\subsection{\texorpdfstring{$\epsilon$}{epsilon}-Means}
\label{additional:unsupervised:epsilonMean}
% TODO

%%%%%%%%%%%%%%%%%%%%%%%%%%%%%%%%%%%%%%%%%%%%%%%%%%%%%%%%
\subsection{Louvain Method}
\label{additional:unsupervised:louvain}
% TODO

%%%%%%%%%%%%%%%%%%%%%%%%%%%%%%%%%%%%%%%%%%%%%%%%%%%%%%%%
%%%%%%%%%%%%%%%%%%%%%%%%%%%%%%%%%%%%%%%%%%%%%%%%%%%%%%%%
\section{Supervised Learning}
\label{additional:supervised}

%%%%%%%%%%%%%%%%%%%%%%%%%%%%%%%%%%%%%%%%%%%%%%%%%%%%%%%%
\subsection{Adversarial Networks} % TODO (AN?)
\label{additional:supervised:AN}
% TODO

%%%%%%%%%%%%%%%%%%%%%%%%%%%%%%%%%%%%%%%%%%%%%%%%%%%%%%%%
\subsection{Variational Autoencoders (VAE)}
\label{additional:supervised:VAE}
% TODO

\subsection{Learning Vector Quantization (LVQ)}
\label{additional:supervised:kNN:LVQ}
% TODO

%%%%%%%%%%%%%%%%%%%%%%%%%%%%%%%%%%%%%%%%%%%%%%%%%%%%%%%%
%%%%%%%%%%%%%%%%%%%%%%%%%%%%%%%%%%%%%%%%%%%%%%%%%%%%%%%%
\section{Miscellaneous}
\label{additional:misc}

%%%%%%%%%%%%%%%%%%%%%%%%%%%%%%%%%%%%%%%%%%%%%%%%%%%%%%%%
\subsection{Checking \texorpdfstring{$n$}{N} for the Model}
\label{additional:misc:enough_data}
% TODO
% TODO ref in text \cref{fig:additional:misc:enough_data}

% TODo Replace with pdf version from HastieTF09 figure 7.8 someday
\begin{figure}
\centering
\includegraphics[width=0.8\textwidth]{figures/ml/acc_vs_n.png}
\caption{
Illustration of the decrease in classification error rate
with larger sets of training data \cite{HastieTF09}.
By artificially limiting the available number of data points $n$,
one can create a similar curve to verify that
there is enough statistics in the training data
for the complexity of the model under consideration.
Ideally the curve should reach a clear asymptote before the maximum $n$ is used.
}
\label{fig:additional:misc:enough_data}
\end{figure}

%%%%%%%%%%%%%%%%%%%%%%%%%%%%%%%%%%%%%%%%%%%%%%%%%%%%%%%%
\subsection{Factor Analysis}
\label{additional:misc:factor_ana}
% TODO

%%%%%%%%%%%%%%%%%%%%%%%%%%%%%%%%%%%%%%%%%%%%%%%%%%%%%%%%
\subsection{Optimization and Lagrange Multipliers}
\label{additional:misc:opt}
% TODO

%%%%%%%%%%%%%%%%%%%%%%%%%%%%%%%%%%%%%%%%%%%%%%%%%%%%%%%%
\subsection{Information Theory}
\label{additional:misc:info_theory}
% TODO
% TODO Entropy, significance of bits

%%%%%%%%%%%%%%%%%%%%%%%%%%%%%%%%%%%%%%%%%%%%%%%%%%%%%%%%
\subsection{Modulo Operations}
\label{additional:misc:modulo}

\begin{subequations}\label{eq:additional:misc:modulo}
\begin{align}
\left(A~\text{mod}~C\right)~\text{mod}~C &= A~\text{mod}~C, \label{eq:misc:additional:modulo:basic} \\
\left(A \pm B\right)~\text{mod}~C &= \left(A~\text{mod}~C \pm B~\text{mod}~C\right)~\text{mod}~C, \label{eq:misc:additional:modulo:pm} \\
\left(A \times B\right)~\text{mod}~C &= \left(A~\text{mod}~C \times B~\text{mod}~C\right)~\text{mod}~C, \label{eq:misc:additional:modulo:multiplication} \\
A^{B}~\text{mod}~C &= \left(\left(A~\text{mod}~C\right)^{B}\right)~\text{mod}~C. \label{eq:misc:additional:modulo:exp}
\end{align}
\end{subequations}

}
%%%%%%%%%%%%%%%%%%%%%%%%%%%%%%%%%%%%%%%%%%%%%%%%%%%%%%%%
%%%%%%%%%%%%%%%%%%%%%%%%%%%%%%%%%%%%%%%%%%%%%%%%%%%%%%%%
\chapter{Probability Distributions}
\label{dist}

\begin{figure}
  \centering
  \savebox{\largestimage}{
    \includegraphics[width=0.47\textwidth]{figures/stats/dist/poisson_pmf.pdf}
  }% Store largest image in a box

  \begin{subfigure}[b]{\wd\largestimage}\centering
    \raisebox{\dimexpr.5\ht\largestimage-.5\height}{% Adjust vertical height of smaller image
      \includegraphics[width=\textwidth]{figures/stats/dist/binomial_pmf.pdf}}
  \caption{Binomial Distribution PMF}
  \label{fig:dist:binomial}
  \end{subfigure}
  ~
  \begin{subfigure}[b]{0.48\textwidth}\centering
    \usebox{\largestimage}
  \caption{Poisson Distribution PMF}
  \label{fig:dist:poisson}
  \end{subfigure}
\caption{
Binomial and Poisson distribution PMFs,
by \href{https://en.wikipedia.org/wiki/File:Binomial_distribution_pmf.svg}{Tayste}
and \href{https://en.wikipedia.org/wiki/File:Poisson_pmf.svg}{Skbkekas}.
Both plots have $k$ on the $x$-axis and $P$ on the $y$-axis, and curves for various $n$ and $p$, or $\lambda$.
\label{fig:dist:binomial_poisson}
}
\end{figure}

\begin{figure}
  \centering
  \savebox{\largestimage}{
    \includegraphics[width=0.47\textwidth]{figures/stats/dist/student_t_pdf.pdf}
  }% Store largest image in a box

  \begin{subfigure}[b]{0.48\textwidth}\centering
    \raisebox{\dimexpr.5\ht\largestimage-.5\height}{% Adjust vertical height of smaller image
      \includegraphics[width=\textwidth]{figures/stats/dist/gaussian_pdf.pdf}}
  \caption{Gaussian Distribution PDF}
  \label{fig:dist:gaus}
  \end{subfigure}
  ~
  \begin{subfigure}[b]{\wd\largestimage}\centering
    \usebox{\largestimage}
  \caption{Student's $t$-Distribution PDF}
  \label{fig:dist:student_t}
  \end{subfigure}
\caption{
Gaussian and Student's $t$-distribution PDFs,
by \href{https://en.wikipedia.org/wiki/File:Normal_Distribution_PDF.svg}{Inductiveload}
and \href{https://en.wikipedia.org/wiki/File:Student_t_pdf.svg}{Skbkekas}.
Note that in the limit $\nu \to \infty$ the Student's $t$-distribution
approaches the standard normal distribution shown in red.
\label{fig:dist:gaus_student_t}
}
\end{figure}


\begin{figure}
\centering
\includegraphics[width=0.7\textwidth]{figures/stats/dist/chi2_pdf.pdf}
\caption{
$\chi^{2}$-distribution PDF,
by \href{https://en.wikipedia.org/wiki/File:Chi-square_pdf.svg}{Geek3}.
Here $k$ is being used in lieu of $\nu$ for the number of degrees of freedom.
}
\label{fig:dist:chi2}
\end{figure}
}
%%%%%%%%%%%%%%%%%%%%%%%%%%%%%%%%%%%%%%%%%%%%%%%%%%%%%%%%%
%%%%%%%%%%%%%%%%%%%%%%%%%%%%%%%%%%%%%%%%%%%%%%%%%%%%%%%%
%%%%%%%%%%%%%%%%%%%%%%%%%%%%%%%%%%%%%%%%%%%%%%%%%%%%%%%%
\chapter{Concepts for Finance}
\label{finance}

Note, I have not studied what is covered in interviews for quantitative finance roles,
but have heard that these are worthwhile topics to review.

%%%%%%%%%%%%%%%%%%%%%%%%%%%%%%%%%%%%%%%%%%%%%%%%%%%%%%%%
%%%%%%%%%%%%%%%%%%%%%%%%%%%%%%%%%%%%%%%%%%%%%%%%%%%%%%%%
\section{Stochastic Processes}
\label{finance:sp}
% TODO

%%%%%%%%%%%%%%%%%%%%%%%%%%%%%%%%%%%%%%%%%%%%%%%%%%%%%%%%
%%%%%%%%%%%%%%%%%%%%%%%%%%%%%%%%%%%%%%%%%%%%%%%%%%%%%%%%
\section{Martingale}
\label{finance:martingale}
% TODO

%%%%%%%%%%%%%%%%%%%%%%%%%%%%%%%%%%%%%%%%%%%%%%%%%%%%%%%%
%%%%%%%%%%%%%%%%%%%%%%%%%%%%%%%%%%%%%%%%%%%%%%%%%%%%%%%%
\section{Wiener Processes}
\label{finance:wiener}
% TODO

%%%%%%%%%%%%%%%%%%%%%%%%%%%%%%%%%%%%%%%%%%%%%%%%%%%%%%%%
%%%%%%%%%%%%%%%%%%%%%%%%%%%%%%%%%%%%%%%%%%%%%%%%%%%%%%%%
\section{Brownian Motion}
\label{finance:brownian}
% TODO

%%%%%%%%%%%%%%%%%%%%%%%%%%%%%%%%%%%%%%%%%%%%%%%%%%%%%%%%
%%%%%%%%%%%%%%%%%%%%%%%%%%%%%%%%%%%%%%%%%%%%%%%%%%%%%%%%
\section{Random Walks}
\label{finance:rand_walk}
% TODO

%%%%%%%%%%%%%%%%%%%%%%%%%%%%%%%%%%%%%%%%%%%%%%%%%%%%%%%%
%%%%%%%%%%%%%%%%%%%%%%%%%%%%%%%%%%%%%%%%%%%%%%%%%%%%%%%%
\section{It\^o's Lemma}
\label{finance:ito_lemma}
% TODO

%%%%%%%%%%%%%%%%%%%%%%%%%%%%%%%%%%%%%%%%%%%%%%%%%%%%%%%%
%%%%%%%%%%%%%%%%%%%%%%%%%%%%%%%%%%%%%%%%%%%%%%%%%%%%%%%%
\section{Black-Scholes Model}
\label{finance:black_scholes}
% TODO

}

%-----------------------------------------------------------------------------%
% BIBLIOGRAPHY -- Change the style to match your discipline's standards.
%-----------------------------------------------------------------------------%
\bibliographystyle{./bib/atlasBibStyleWithTitle}
\cleardoublepage
\normalbaselines %Fixes spacing of bibliography
% \addcontentsline{toc}{chapter}{Bibliography} % not needed on my system
\bibliography{./bib/bib}
%-----------------------------------------------------------------------------%

%-----------------------------------------------------------------------------
\end{document}
